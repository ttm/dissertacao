% 	Modelo de Tese/Dissertação segundo as normas da Pós-graduação do IFSC-USP
% 	Criado por Alexandre de Castro Maciel - Grupo de Polimeros Bernhard Gross
% 	Este trabalho está licenciado sob uma Licença Creative Commons Atribuição-Uso Não-Comercial-Compartilhamento pela mesma Licença 2.5 Brasil
% 	Modelo criado e testado em ambiente Linux/Debian
% 	Software utilizados: Kile,Kbibtex,Kpdf,Gimp
%
%
%-----------------TIPO DE DOCUMENTO------------------
\documentclass[espaco=duplo,twoside,openany,tocpage=plain]{abnt}
%-----------------NORMAS IFSC------------------------
\usepackage{estiloifsc}
%-----------------INICIO DO DOCUMENTO----------------
\begin{document}
%-----------------PARTE PRE-TEXTUAL------------------
%	Preencha os campos abaixo para fornecer as informações necessárias para confecção da capa e folha de rosto
\autor{RENATO FABBRI}

% \titulo{Código Aberto em áudio, web e cultura digital}
\titulo{Áudio, Código Aberto e Cultura Digital}
% \titulo{Desenvolvimentos em áudio e web: suprindo demandas em código Aberto e cultura digital}

\instituicao{Universidade de São Paulo \hspace{10cm} Instituto de Física de São Carlos \hspace{10cm}  Departamento de Física e Informática \hspace{10cm} Grupo de Física Computacional e Instrumentação Aplicada}
% OU GRUPO DE Conputação Mutidisciplinar, veja em: http://cyvision.ifsc.usp.br/~luciano/

%	Alunos da Física deixem as três linhas abaixo não comentadas
\comentario{Dissertação apresentada ao Programa de Pós-graduação em Física do Instituto de Física de São Carlos da Universidade de São Paulo, para a obtenção do título de Mestre em Ciência.}
\area{Física Aplicada} %Física Aplicada ou Básica
% \opcao{Física Computacional} %Usado para quem faz parte da Física Computacional ou Biomolecular
%	Alunos da Interunidades deixem as duas linhas abaixo descomentadas
% \comentario{Tese apresentada ao Programa de Pós-graduação Interunidades em Ciência e Engenharia de Materiais da Universidade de São Paulo, para a obtenção do título de Doutor em Ciência e Engenharia de Materiais.}
% \area{Desenvolvimento, Caracterização e Aplicação dos Materiais} %Física Aplicada ou Básica
%-------
\orientador{Prof. Dr. Osvaldo Novais de Oliveira Junior}
\coorientador{Prof. Dr. Luciano da Fontoura Costa}
\local{São Carlos}
\data{2011}
%	Este comando inclui o arquivo pretextual.tex onde estão todas as informações pretextuais.
%	Nesse arquivo você poderá alterar as informações e formatação das seguintes paginas:
%	* Capa e folha de rosto
%	* Dedicatória
%	* Agradecimentos
%	* Epígrafe
%	* Resumo
%	* Lista de figuras (automatizada)
%	* Lista de tabelas (automatizada)
%	* Lista de abreviaturas (manual)
%	* Lista de símbolos (manual)
%	* Sumário (automatizado)
\capa

% \pretextualchapter{}

\folhaderosto
% 
\pretextualchapter{}
	\thispagestyle{plain}
	\noindent \parbox{5.7in}{\centering AUTORIZO A REPRODUÇÃO E DIVULGAÇÃO TOTAL OU PARCIAL DESTE TRABALHO, POR QUALQUER MEIO CONVENCIONAL OU ELETRÔNICO, PARA FINS DE ESTUDO E PESQUISA, DESDE QUE CITADA A FONTE.}

\pretextualchapter{}
	Lado dedicado à folha de aprovação. Apagar isso antes de imprimir a versão oficial

\pretextualchapter{Dedicatória} %Título da página dedicatória
	Dedico este trabalho à minha família e às pessoas próximas, especialmente
à minha mulher Thaís Teixeira Fabbri e ao meu filho Antônio Anzoategui Fabbri. %Para incluir o texto de dedicatória,use o arquivo dedicatória.tex

\pretextualchapter{Agradecimentos} %Título da página de agradecimentos
	    Agradeço aos meus pais e familiares por todo o suporte e companhia. Aos
meus amigos pelo carinho. Aos professores por tanta atenção e compreensão.
Aos colegas colecionadores de entendimentos por ter-nos construído até aqui.
Agradeço a Felipe Machado pela visão, a Fabiana 'Goa' Sherine pelo entendimento,
a Glerm Soares pela transcendência, a Fabianne Baveldi pelo modelo, a Daniel
Marostegan e Carneiro pelas lições. Um profundo agradecimento aos meus orientadores de IC
e MSC incompleto na FEEC/Unicamp. Não posso deixar de agradecer nominalmente
ao Prof. Luciano da Fontoura Costa, que esteve presente na quase totalidade deste
trabalho e notavelmente elevou-o a outro patamar. Finalmente, agradeço ao meu 
orientator Prof. Osvaldo Novais de Oliveira Júnior pelo inestimável acompanhamento
e notável paciência e dedicação em todos os momentos.
 %Para incluir o texto de agradecimentos,use o arquivo agradecimentos.tex

\pretextualchapter{}
	\begin{epigrafetop}
		{from Pantheon import Obatalá\, \, \, \\
                from World import sharing, kitten \\
                \vspace{.1in}
                while True: \; \; \; \; \; \; \; \; \; \; \; \; \; \; \; \,\\
                    if sharing == 0: \; \; \; \; \;  \; \; \; \; \; \\
                        Obatalá.kill( kitten ) \; \; \;}
        {Autoria coletiva e anônima (2010)}
	\end{epigrafetop}
% \,      a small space
% \:      a medium space
% \;      a large space
% \quad   a really large space
% \qquad  a huge space
% \!      a negative space (moves things back to the left) 
%	\begin{epigrafemid}
%		{They must find it difficult ... \\ Those who have taken authority as the truth, \\ rather than truth as the authority.} %frase
%		{Gerald Massey (1828 - 1907)} %referência
%	\end{epigrafemid}
	\begin{epigrafemid}
		{If I have seen further, \\it is by standing on the shoulders of giants.} %frase
		{Sir Isaac Newton (1642 - 1727)} %referência
	\end{epigrafemid}
	\vspace{-1cm}
	\begin{epigrafebot}
		{Electronic music used pure sounds, completely calibrated. You had to think digitally, as it were, in a way that allowed you to extend serial ideas into other parameters through technology.} %frase
		{Luc Ferrari (1929 - 2005)} %referência
	\end{epigrafebot}
\resumoeabstract %Para incluir o texto do resumo e abstract,use o arquivo resumoeabstract.tex

\listadefiguras %Comando que gera a lista de figuras (AUTOMÁTICO)

\listadetabelas %Comando que gera a lista de tabelas (AUTOMÁTICO)

\pretextualchapter{Lista de Abreviaturas}
	\begin{listaespecial}[BIGNAMEWIDTH]
		\item[OSS] Programas de Código Aberto \emph{(Open Source Software)}
		\item[GNU] \emph{GNU is Not UNIX}
		\item[GPL] \emph{General Public Licence}
                \item[AA] \emph{Algorithmic Autoregularion ou Acrônimo Ambíguo}
                \item[AC] \emph{Autogestão Coletiva ou Ágora Communs}
                \item[CC] \emph{Creative Commons}
                \item[ABT] \emph{ABeatTracker}
                \item[ABD] \emph{ABeatDetector}
                \item[PD] \emph{Pure Data}
                \item[CMDCA] \emph{Conselho Municipal de defesa dos Direitos da Cirança e do Adolescente}
                \item[SOS] \emph{Saúde Olha Sabedoria: uma proposta de coleta e difusão de conhecimentos relacionados à saúde}
                \item[LADSPA] \emph{Linux Audio Developers Simple Plugin API}
                \item[LV2] \emph{LADSPA Version 2}
                \item[JACK] \emph{Jack Audio Connection Kit}
                \item[SIP] \emph{Scilab Imaging Processing toolbox}
                \item[MIT] \emph{Massachusetts Institute of Technology}
                \item[NUMPY] \emph{Numerical Python}
                \item[SCIPY] \emph{Scientific Python}
                \item[ONU] \emph{Organização das Nações Unidas}
                \item[SL] \emph{Software Livre}
                \item[EKP] \emph{Emotional Kernel Panic}
                \item[SOS] \emph{Saúde Olha Sabedoria}
                \item[FISL] \emph{Festival Internacional de Software Livre}
                \item[BPM] \emph{Batidas Por Minuto (medida de andamento musical)}
                \item[API] \emph{Application Programming Interface ou Interface de Programação de Aplicativos}
                \item[WP] \emph{Wavelet Packet}

		\item[EL] \emph{Estúdio Livre}
		\item[AE] \emph{AudioExperiments}
		\item[PCM] \emph{Pulse Code Modulation (modulação por código de pulsos)}
		\item[DTI] \emph{Diferença de Tempo Interaural (\emph{Interaural Time Difference} - ITD)}
		\item[DII] \emph{Diferença de Intensidade Interaural (\emph{Interaural Intensity Difference)} - IID ou \emph{Interaural Level Difference} - ILD}
	\end{listaespecial} 

\pretextualchapter{Lista de Símbolos}
	\begin{listaespecial}[BIGNAMEWIDTH]
		\item[\$] Indica que o resto da linha é um comando bash$\Lambda_{DTI}=DTI . f_a$
		\item[\#] Resto da linha é comentario em código Bash ou Python
		\item[] Resto da linha é comentario em código C/C++
	\end{listaespecial} 

\pretextualchapter{Glossário de Termos}
	\begin{listaespecial2}[BIGNAMEWIDTH]
		\item[Montagem musical :] algumas ideias musicais rascunhadas, sem a pretensão de ser uma peça ou uma obra musical. Sinônimo de Sequência musical.
		\item[Sequência musical :] sinônimo de montagem musical. 
		\item[Tessitura :] âmbito de altura ($\log(freq)$) do objeto em discussão (instrumento, passagem musical, etc).

		\item[Delay :] embora a palavra signifique simplesmente atraso, no contexto de música a áudio o \emph{delay} comumente se refere a um efeito causado por diversas repetições de um som original, com atrasos diferentes.
	\end{listaespecial2} 


\sumario


%-----------------PARTE TEXTUAL----------------------
%	Comandos que incluem os elementos textuais. A variável cap[j] é referente ao arquivo cap[j].tex. Pode ser qualquer nome, desde que exista no mesmo diretorio do tese.tex um capítulo referente.
%	Ex: Se você usa %% ------------------------------------------------------------------------- %%
\chapter{Introdução} %Nome do capítulo.
\setcounter{page}{33}
\label{cap:intro} 
\epigraph{"Tradicionalmente a notação musical é vista como um código através do qual sons, ideias musicais ou indicações para execução musical são registrados sob forma escrita."}{Edson S. Zampronha.\cite{Zampronha} \\}


Representar estruturas e artifícios musicais através das características do som discretizado
é a proposta deste trabalho. Os resultados são relações matemáticas e suas implementações computacionais. Uma descrição teórica está no capítulo~\ref{cap:resultados} e o conjunto de \emph{scripts} disponibilizados no Apêndice~\ref{cap:codigoProc} e \emph{online}. A caixa de ferramentas (\emph{toolbox}) recebeu o nome \massa\ (música e áudio em sequências e séries amostrais) e foi utilizada para fazer pequenas peças e montagens focadas nos princípios expostos. O Apêndice~\ref{cap:codigoPecas} possui uma relação destas montagens assim como o diretório \emph{exemplos\_de\_uso} da \massa.\cite{MASSA}

    \section{Som em áudio digital}\label{sec:audio}

O som é uma onda mecânica longitudinal de pressão. A banda de frequências compreendida entre $20Hz$ e $20 kHz$ é apreciada pelo aparelho auditivo humano com variações dependentes da pessoa, das condições climáticas e do som em si.
 Considerada a velocidade do som no ar $\approx 343.2\,m/s$,
estes limites correspondem respectivamente aos comprimentos de onda $\frac{343.2}{20} = 17.16\,m$ e $\frac{343.2}{20000}=17.16\,mm$.\cite{Roederer}


A percepção humana do som envolve captações pelos ossos, estômago e orelha, funções de transferência da cabeça e dorso e processamento pelo sistema nervoso. Além disso, o ouvido é um órgão dedicado à captura destas ondas. Seu funcionamento decompõe o som em seu espectro senoidal e passa para o sistema nervoso.\cite{Roederer} Estas componentes senoidais são cruciais para os fenômenos musicais, como se pode observar tanto na composição dos sons de interesse para a música quanto nas afinações e escalas.\cite{floEsp} A subseção~\ref{sec:notaDisc} expõe a presença de senoides no som discretizado e caracteriza a nota musical básica.

A representação do som é o áudio\footnote{Os termos
som e áudio são muitas vezes usados de forma intercambiável.\cite{Everest}} e este pode provir da captura do som por microfones ou da síntese. Muitas vezes, o áudio digital é especificado através de protocolos que facilitam o armazenamento e transferência dos arquivos. A representação digital do som pode consistir em amostras igualmente espaçadas no tempo e cujas amplitudes individuais são registradas com um mesmo número de \emph{bits}. Estas amostras separadas por intervalos regulares $\lambda_a$ constituem a forma padrão de representação do som em tempo discreto, chamada de modulação por código de pulsos (PCM do inglês \emph{Pulse Code Modulation}).
Um som digital PCM é caracterizado pela frequência de amostragem $f_a=\frac{1}{\lambda_a}$, também chamada de taxa de amostragem, e a profundidade de \emph{bit} que é o número de \emph{bits} utilizados para representar a amplitude de cada amostra. A figura~\ref{fig:PCM} exibe $25$ amostras de um áudio PCM com $4$ \emph{bits} cada. Os $2^4=16$ grados para a amplitude de cada amostra junto ao espaçamento regular $\lambda_a$ introduzem um erro de quantização. O ruído causado por estes erros diminuem com a diminuição destes espaçamentos.\cite{audioMedia} 


\begin{figure}[h!]
    \centering
        \includegraphics[width=\textwidth]{figuras/pcm}
        \caption{Som digital em modulação por código de pulsos (PCM): 25 amostras representadas por 4 bits cada uma.}
        \label{fig:PCM}
\end{figure}

Pelo teorema de Nyquist, constata-se que a metade da frequência de amostragem é a frequência máxima do sinal. Assim, para apreender as frequências audíveis, é necessária uma taxa de amostragem que seja ao menos o dobro da frequência mais aguda $f_a \geq 2\times 20kHz=40kHz$. Este raciocínio está na base da utilização das frequências de amostragem $f_a=44.1kHz$ e $f_a=48kHz$, ambas padrão em \emph{Compact Disks} (CDs) e em sistemas de Rádio e TV, respectivamente.\cite{audioMedia}


    \section{Arte sonora e teoria musical}

    A música é definida como a arte manifesta pelos sons e silêncios. Para um ouvinte comum - e boa parte dos especialistas - uma 'música que seja música' pressupõe também uma métrica rítmica e organizações de alturas que formem melodias e harmonias como explicadas na seção~\ref{notasMusica}. A música do século XX ampliou esta concepção tradicional de música. Isso ocorreu na música de concerto, especialmente nas correntes concreta, eletrônica e eletroacústica. Já na década de 90 era evidente que também a música popular, especialmente as músicas eletrônicas de dança, tinham incorporado sons sem altura definida e organizações temporais fora de métricas simples. Mesmo assim, a nota permanece paradigmática como 'unidade fundamental' das estruturas musicais e, na prática, pode se desdobrar em sons que contemplam estes desenvolvimentos recentes. A definição e expansão da nota como unidade fundamental da música são abordados nas seções~\ref{varInternas} e~\ref{sec:notaDisc}, respectivamente. A seção~\ref{notasMusica} trata da organização das notas em estruturas de alto nível.\cite{Wisnick,Webern,Lerdahl,Cook,Lacerda} 

A teoria musical engloba assuntos tão diversos quanto psico-acústica, manifestações culturais e formalismos. O texto do capítulo~\ref{cap:resultados} aborda estes assuntos mediante necessidade e assinala complementos externos.\cite{Zamacois,Schoenberg,microsound}



    \section{Implementação computacional}
Os resultados apresentados desta dissertação incluem \emph{scripts}, i.e. pequenos programas para melhor disponibilidade e validação das tecnologias. Estes constituem a caixa de ferramentas \massa, disponibilizada em domínio público através de repositórios Git abertos.\cite{gitBook}
Os \emph{scripts} estão em Python e fazem uso das bibliotecas externas Numpy e Scikits/Audiolab que realizam chamadas à linguagem Fortran para maior eficiência computacional. Parte deste código foi transcrita para JavaScript e Python nativos com facilidade, o que aponta para um uso destas contribuições em navegadores como o Firefox e o Chromium.\cite{numpy,audiolab,tutPython,python}

Estas tecnologias são todas abertas, i.e. estão publicadas em licenças que permitem o uso, cópia, distribuição e utilização de quaisquer partes para estudo e geração de produtos derivados. Desta forma, o trabalho aqui descrito está disponível e facilita os processos de co-autoria\footnote{A comunidade e movimento chamada \emph{'Open Source'} entende a publicação de código computacional (e outras tecnologias) em licenças abertas como uma vantagem pragmática que facilita o desenvolvimento de \emph{software} e apresenta vantagens pedagógicas e mercadológicas. A comunidade e movimento chamada \emph{'Free Software'} engloba este entendimento, mas adiciona a abordagem filosófica da liberdade e compartilhamento, dando ênfase a isso. Ambas as correntes reforçam o entendimento de que o código computacional é o 'bem mais precioso produzido atualmente' pois consiste em tecnologia condensada, reativa (executa, processa ou gera resultados), modular (partes são copiadas e reutilizadas eficientemente) e replicada sem custo adicional (a cópia de texto tem custo baixíssimo).\cite{Raymond,Lessig}}.

    \section{Objetivos}
   \label{sec:objetivos}
   O objetivo principal desta dissertação é apresentar de forma unificada relações entre elementos básicos da música e as sequências amostrais do áudio PCM. O capítulo seguinte é um texto conciso em que os elementos musicais são apresentados junto às amostras temporais resultantes. Para validação e compartilhamento, as implementações em código computacional destas relações e de pequenas peças musicais\footnote{Ou, de forma menos pretensiosa, montagens musicais, sequências musicais.} foram reunidas em uma \emph{toolbox} chamada \massa\ e disponibilizadas \emph{online} e parcialmente nos Apêndices~\ref{cap:codigoProc} e~\ref{cap:codigoPecas}. 

Dos objetivos secundários, destaca-se a difusão da compreensão do código computacional através de práticas lúdicas, no caso a música. Outro objetivo considerado é a apresentação de um arcabouço de síntese sonora e musical com controle amostral, para o qual há potenciais usos em experimentos psico-acústicos e síntese em alta definição (\emph{hi-fi}). Também é considerada a apresentação destes conteúdos de forma didática, quase um tutorial, o que possibilita compreensão e uso facilitados. Esta exposição amistosa faz-se significativa pois os assuntos tratados são de reconhecida complexidade: processamento de sinais, música, psico-acústica, para citar somente alguns exemplos. Deste ponto de vista pedagógico, também se presta a apresentação destes resultados na forma de hipertexto, em que cada \emph{script} e exemplo sonoro/musical seja acessível junto ao material teórico.

\section{Trabalhos relacionados}

Dado o interesse humano pela música e a multidisciplinaridade inerente a esta dissertação, os trabalhos relacionados são numerosos. Assim, o Apêndice~\ref{cap:trabalhosRelacionados} é dedicado aos livros e implementações computacionais de interesse ou que apresentem similaridades com a descrição de elementos musicais em termos do áudio digital. Não há ênfase em artigos pois foram poucos os encontrados. A visita indica aspectos inéditos deste trabalho, em especial a descrição analítica de elementos musicais básicos em termos das amostras sonoras e a descrição natural, formal e concisa de técnicas tradicionais da música. 
, deverá existir um arquivo tex com o nome introducao.tex no mesmo diretorio de tese.tex.
%% ------------------------------------------------------------------------- %%
\chapter{Introdução} %Nome do capítulo.
\setcounter{page}{33}
\label{cap:intro} 
\epigraph{"Tradicionalmente a notação musical é vista como um código através do qual sons, ideias musicais ou indicações para execução musical são registrados sob forma escrita."}{Edson S. Zampronha.\cite{Zampronha} \\}


Representar estruturas e artifícios musicais através das características do som discretizado
é a proposta deste trabalho. Os resultados são relações matemáticas e suas implementações computacionais. Uma descrição teórica está no capítulo~\ref{cap:resultados} e o conjunto de \emph{scripts} disponibilizados no Apêndice~\ref{cap:codigoProc} e \emph{online}. A caixa de ferramentas (\emph{toolbox}) recebeu o nome \massa\ (música e áudio em sequências e séries amostrais) e foi utilizada para fazer pequenas peças e montagens focadas nos princípios expostos. O Apêndice~\ref{cap:codigoPecas} possui uma relação destas montagens assim como o diretório \emph{exemplos\_de\_uso} da \massa.\cite{MASSA}

    \section{Som em áudio digital}\label{sec:audio}

O som é uma onda mecânica longitudinal de pressão. A banda de frequências compreendida entre $20Hz$ e $20 kHz$ é apreciada pelo aparelho auditivo humano com variações dependentes da pessoa, das condições climáticas e do som em si.
 Considerada a velocidade do som no ar $\approx 343.2\,m/s$,
estes limites correspondem respectivamente aos comprimentos de onda $\frac{343.2}{20} = 17.16\,m$ e $\frac{343.2}{20000}=17.16\,mm$.\cite{Roederer}


A percepção humana do som envolve captações pelos ossos, estômago e orelha, funções de transferência da cabeça e dorso e processamento pelo sistema nervoso. Além disso, o ouvido é um órgão dedicado à captura destas ondas. Seu funcionamento decompõe o som em seu espectro senoidal e passa para o sistema nervoso.\cite{Roederer} Estas componentes senoidais são cruciais para os fenômenos musicais, como se pode observar tanto na composição dos sons de interesse para a música quanto nas afinações e escalas.\cite{floEsp} A subseção~\ref{sec:notaDisc} expõe a presença de senoides no som discretizado e caracteriza a nota musical básica.

A representação do som é o áudio\footnote{Os termos
som e áudio são muitas vezes usados de forma intercambiável.\cite{Everest}} e este pode provir da captura do som por microfones ou da síntese. Muitas vezes, o áudio digital é especificado através de protocolos que facilitam o armazenamento e transferência dos arquivos. A representação digital do som pode consistir em amostras igualmente espaçadas no tempo e cujas amplitudes individuais são registradas com um mesmo número de \emph{bits}. Estas amostras separadas por intervalos regulares $\lambda_a$ constituem a forma padrão de representação do som em tempo discreto, chamada de modulação por código de pulsos (PCM do inglês \emph{Pulse Code Modulation}).
Um som digital PCM é caracterizado pela frequência de amostragem $f_a=\frac{1}{\lambda_a}$, também chamada de taxa de amostragem, e a profundidade de \emph{bit} que é o número de \emph{bits} utilizados para representar a amplitude de cada amostra. A figura~\ref{fig:PCM} exibe $25$ amostras de um áudio PCM com $4$ \emph{bits} cada. Os $2^4=16$ grados para a amplitude de cada amostra junto ao espaçamento regular $\lambda_a$ introduzem um erro de quantização. O ruído causado por estes erros diminuem com a diminuição destes espaçamentos.\cite{audioMedia} 


\begin{figure}[h!]
    \centering
        \includegraphics[width=\textwidth]{figuras/pcm}
        \caption{Som digital em modulação por código de pulsos (PCM): 25 amostras representadas por 4 bits cada uma.}
        \label{fig:PCM}
\end{figure}

Pelo teorema de Nyquist, constata-se que a metade da frequência de amostragem é a frequência máxima do sinal. Assim, para apreender as frequências audíveis, é necessária uma taxa de amostragem que seja ao menos o dobro da frequência mais aguda $f_a \geq 2\times 20kHz=40kHz$. Este raciocínio está na base da utilização das frequências de amostragem $f_a=44.1kHz$ e $f_a=48kHz$, ambas padrão em \emph{Compact Disks} (CDs) e em sistemas de Rádio e TV, respectivamente.\cite{audioMedia}


    \section{Arte sonora e teoria musical}

    A música é definida como a arte manifesta pelos sons e silêncios. Para um ouvinte comum - e boa parte dos especialistas - uma 'música que seja música' pressupõe também uma métrica rítmica e organizações de alturas que formem melodias e harmonias como explicadas na seção~\ref{notasMusica}. A música do século XX ampliou esta concepção tradicional de música. Isso ocorreu na música de concerto, especialmente nas correntes concreta, eletrônica e eletroacústica. Já na década de 90 era evidente que também a música popular, especialmente as músicas eletrônicas de dança, tinham incorporado sons sem altura definida e organizações temporais fora de métricas simples. Mesmo assim, a nota permanece paradigmática como 'unidade fundamental' das estruturas musicais e, na prática, pode se desdobrar em sons que contemplam estes desenvolvimentos recentes. A definição e expansão da nota como unidade fundamental da música são abordados nas seções~\ref{varInternas} e~\ref{sec:notaDisc}, respectivamente. A seção~\ref{notasMusica} trata da organização das notas em estruturas de alto nível.\cite{Wisnick,Webern,Lerdahl,Cook,Lacerda} 

A teoria musical engloba assuntos tão diversos quanto psico-acústica, manifestações culturais e formalismos. O texto do capítulo~\ref{cap:resultados} aborda estes assuntos mediante necessidade e assinala complementos externos.\cite{Zamacois,Schoenberg,microsound}



    \section{Implementação computacional}
Os resultados apresentados desta dissertação incluem \emph{scripts}, i.e. pequenos programas para melhor disponibilidade e validação das tecnologias. Estes constituem a caixa de ferramentas \massa, disponibilizada em domínio público através de repositórios Git abertos.\cite{gitBook}
Os \emph{scripts} estão em Python e fazem uso das bibliotecas externas Numpy e Scikits/Audiolab que realizam chamadas à linguagem Fortran para maior eficiência computacional. Parte deste código foi transcrita para JavaScript e Python nativos com facilidade, o que aponta para um uso destas contribuições em navegadores como o Firefox e o Chromium.\cite{numpy,audiolab,tutPython,python}

Estas tecnologias são todas abertas, i.e. estão publicadas em licenças que permitem o uso, cópia, distribuição e utilização de quaisquer partes para estudo e geração de produtos derivados. Desta forma, o trabalho aqui descrito está disponível e facilita os processos de co-autoria\footnote{A comunidade e movimento chamada \emph{'Open Source'} entende a publicação de código computacional (e outras tecnologias) em licenças abertas como uma vantagem pragmática que facilita o desenvolvimento de \emph{software} e apresenta vantagens pedagógicas e mercadológicas. A comunidade e movimento chamada \emph{'Free Software'} engloba este entendimento, mas adiciona a abordagem filosófica da liberdade e compartilhamento, dando ênfase a isso. Ambas as correntes reforçam o entendimento de que o código computacional é o 'bem mais precioso produzido atualmente' pois consiste em tecnologia condensada, reativa (executa, processa ou gera resultados), modular (partes são copiadas e reutilizadas eficientemente) e replicada sem custo adicional (a cópia de texto tem custo baixíssimo).\cite{Raymond,Lessig}}.

    \section{Objetivos}
   \label{sec:objetivos}
   O objetivo principal desta dissertação é apresentar de forma unificada relações entre elementos básicos da música e as sequências amostrais do áudio PCM. O capítulo seguinte é um texto conciso em que os elementos musicais são apresentados junto às amostras temporais resultantes. Para validação e compartilhamento, as implementações em código computacional destas relações e de pequenas peças musicais\footnote{Ou, de forma menos pretensiosa, montagens musicais, sequências musicais.} foram reunidas em uma \emph{toolbox} chamada \massa\ e disponibilizadas \emph{online} e parcialmente nos Apêndices~\ref{cap:codigoProc} e~\ref{cap:codigoPecas}. 

Dos objetivos secundários, destaca-se a difusão da compreensão do código computacional através de práticas lúdicas, no caso a música. Outro objetivo considerado é a apresentação de um arcabouço de síntese sonora e musical com controle amostral, para o qual há potenciais usos em experimentos psico-acústicos e síntese em alta definição (\emph{hi-fi}). Também é considerada a apresentação destes conteúdos de forma didática, quase um tutorial, o que possibilita compreensão e uso facilitados. Esta exposição amistosa faz-se significativa pois os assuntos tratados são de reconhecida complexidade: processamento de sinais, música, psico-acústica, para citar somente alguns exemplos. Deste ponto de vista pedagógico, também se presta a apresentação destes resultados na forma de hipertexto, em que cada \emph{script} e exemplo sonoro/musical seja acessível junto ao material teórico.

\section{Trabalhos relacionados}

Dado o interesse humano pela música e a multidisciplinaridade inerente a esta dissertação, os trabalhos relacionados são numerosos. Assim, o Apêndice~\ref{cap:trabalhosRelacionados} é dedicado aos livros e implementações computacionais de interesse ou que apresentem similaridades com a descrição de elementos musicais em termos do áudio digital. Não há ênfase em artigos pois foram poucos os encontrados. A visita indica aspectos inéditos deste trabalho, em especial a descrição analítica de elementos musicais básicos em termos das amostras sonoras e a descrição natural, formal e concisa de técnicas tradicionais da música. 

%% ------------------------------------------------------------------------- %%
\chapter{Fundamentos teóricos e técnicos} %Nome do capítulo.
\label{cap:intro} %Rótulo para futura referência ao capítulo. Em qualquer lugar da tese, você poderá citar este capítulo através de ~\ref{cap:introducao}. Você escolhe o argumento de \label e pode ser qualquer coisa (Ex: \label{Procedimento_Experimental})

\section{Áudio}

\subsection{O som como fenômeno físico}

O som é o fenômeno físico de propagação de ondas mecânicas. Ele pode
somente ocorrer na presença da matéria, ou seja, o vácuo é um
isolante acústico perfeito.

\subsection{A audição e a psicofísica}

A audição é definida com a percepção de vibrações mecânicas com
ou sem a contribuição direta do tato. Em diversos
seres vivos, encontramos um órgão dedicado a captar e transmitir
estas vibrações mecânicas (som) para o sistema nervoso.

O ouvido humano cumpre exatamente este papel. Um esquema básico
do ouvido humano conta com 3 divisões. No ouvido externo tem a orelha.
No médio, tem os martelinhos e o tímpano. No ouvido interno tem a cóclea
e os cíclios. Uma visualização simplificada do ouvido humano está abaixo.


FIGURA MOSTRANDO ENTRADA DE VIBRAÇÃO E TRANSMISSÃO AO CÉREBRO

Vale notar que a audição conta em grande parte com o tato e potencialmente
outros órgãos, como o estômago.

Já a psicofísica busca relacionar características físicas do fenômeno sonoro
com as percepções individuais causadas\footnote{PsycoPy para escrever experimentos psicofísicos em Python: http://www.psychopy.org/}.
Importantes exemplos destas relações
é a escala logarítimica que a percepção linear das alturas assumem no
domínio da frequência e as curvas iso-audíveis (conhecidas como curvas de Fletcher-Münsen).

Outro importante aspecto da psicofísica do som - e da música - é a resolução da percepção
e a capacidade de notar características do espectro em contextos específicos. O mascaramento
de frequências é a base na qual se faz compressores de tamanhos de arquivos de áudio. Sabe-se
que, por exemplo, no extremo grave, não conseguimos distinguir duas senóides a quase um tom de
diferença. Outros fenômenos psicoacústicos estudados com fins tecnológicos, científicos e artísticos são: espacialidade,
relação da percepção de volume com espectro agudo e reverberação, compreensibilidade de sinais de fala, etc.


\subsection{Som digitalizado}

Ver livros em casa.

Representação digital do som.

Padrões PCM. Padrões compactados.

\subsection{Processamento de áudio}

Ver livros em casa.

Procedimentos básicos. Processamento
espectral. Outros processamentos.

Vincular a um apêndice com excertos de código.

\section{Linguagens e modularidade}

\subsection{Breve histórico}

Discursar brevemente sobre as diversas linguagens,
inclusive voltadas para áudio. E sobre a reutilização
de trechos de código.

\subsection{C/C++}

\subsection{Python}


\section{Sistema Operacional e Web}

\subsection{Breve histórico}

O hardware estava ficando mais ambundante, sendo replicado em 
vários lugares. Modelos eram vendidos com diferenças de hardware 
de mínimas a discrepantes. Neste contexto, começou-se a
pensar em uma camada de código que abstraísse o hardware, assim
como a linguagem C se propôs.

Iniciou-se assim o desenvolvimento de Sistemas Operacionais (SO).
Um SO é um software que gerencia os recursos do computador e ajuda
também em abstrair o hardware. Orientamos a esta excelênte linha do
tempo da Wikipédia em língua inglesa \cite{solinhadotempo}:



\subsection{Unix e GNU/LINUX}

Resumo de utilização. No apêndice o resumo de:
The Unix programing envinoment
e
Running Linux

\subsection{Web e Browsers}

Dados do livrinho do google.

Possibilidades de uso.

Breve Histórico.

No apêndice resumo de:
http://www.html5rocks.com/en/tutorials/internals/howbrowserswork/




%% ------------------------------------------------------------------------- %%
\chapter{Desenvolvimentos e resultados} %Nome do capítulo.
\label{cap:resultados} %Rótulo para futura referência ao capítulo. Em qualquer lugar da tese, você poderá citar este capítulo através de ~\ref{cap:introducao}. Você escolhe o argumento de \label e pode ser qualquer coisa (Ex: \label{Procedimento_Experimental})

\section{Materiais didáticos}

\subsection{Tutoriais em texto e código}

Os vários materiais didáticos produzidos constam no apêndice
deste trabalho. Um único destes será exposto a seguir por sua
capacidade de agregar os conteúdos dos capítulos anterioes:
os tutoriais de filtros e amostragem.

\begin{itemize}
    \item {\bf Tutorial de python para áudio e som}

Este tutorial foi levado para Berlim no LAC 2007 e sofreu melhoras desde entao. Esta
primeira versao ficou resumida em forma de texto no EL\footnote{http://estudiolivre.org/python-e-som-tutorial}. Em 2010
a Associacao Python Brasil escolheu este trabalho, então já mais amadurecido, para ser apresentado no
FISL em Porto Alegre. Como consequencia, foi feita uma série de video-tutoriais bastante utilizados\footnote{http://estudiolivre.org/tiki-index.php?page=Video+Tutoriais}.
Este tutorial foi comentado em listas em que o autor não participa (e outras em que o autor participa).

    \item {\bf Tutoriais de filtros e amostragem via python}

Voltados para explicitar principios fundamentais de áudio, estes tutoriais
são baseados código Python e o equivalente em C. Pequenas explicações são
dadas com o intuito de orientar a exploração inteligente destes \emph{snippets}.

\emph{Teorema de Amostragem}: estes scripts visam a experimentacao inteligente com
o Teorema de Nyquist. (descricao)

\emph{Filtros}: alem da explicitação sobre as diferenças entre filtros FIR e IIR,
duas utilizações clássicas destes filtros estão implementadas: Wavelets (FIR) e Quad (IIR)

Estes códigos podem ser baixados no repositório SVN do AudioExperiments. E os textos estão
na wiki (nos digitais ou EL, recriar pois os CDTL foram apagados)

    \item {\bf Tutorial de plugins lv2}

Dadas as dificuldades que o desenvolvimento dos \emph{plugins} de áudio apresenta,
desenvolvi um tutorial passo a passo com plugins que rodam em todas as etapas.
Ele é baseado em uma interface C++ para este padrão de plugin que eh implementado
em C. Os códigos e os textos estão todos em repositório.

    \item {\bf Microtutoriais Django ~\cite{dmicrotuts}}

Estes 'microtutoriais' são baseados nos conceitos de \emph{scripts mínimos} e
\emph{alterações puntuais}. O primeiro conjunto de microtutoriais é dedicado
a reconstruir o tutorial oficial do django de forma condensada e não prolixa.
O segundo destes conjuntos é dedicado a instrumentalizar de fato o leitor com
o entendimento do funcionamento dos princípios fundamentais deste framework.

    \item {\bf Philosometrics}

Embora este não seja um trabalho didático propriamente dito, ele tem este intuito
no cerne de sua concepção e surgimento. Em decorrência dele, surgiu o  Musimetrics,
o Cinemetrics e o Literametrics. Além disso, ele é um belo exemplo da
utilização das ciências duras para a análise de ciências humanas e foi acolhido
como tal em alguns momentos.

    \item {\bf Carta mídias livres}

Texto criado em decorrência da participação da comissão de seleção no
'Prêmio Mídias Livres', a convite do Ministério da Cultura por 'notório saber'.
Esta carta é um documento único no seu conteúdo, deixando às claras
o conceito de Mídias Livres como mídias não aprisionadas pelo conceito
de propriedade, ou seja, que priorizam a sua livre circulação e a possibilidade
de geração de materiais derivados, assim como sua geração aberta ao colaborativo e comunitário.

    \item {\bf Textos de cunho sociológico, transformador}

Produção mais numerosa que as anteriores, se caracteriza por métodos não convencionais
de abordagem dos assuntos e de escrita. Em especial utiliza-se pseudônimos para
auxiliar a despersonificação, gerando textos menos presos à satisfação da auto-imagem, dente
outras qualidades. A utiliação de psudônimos é um costume muito apreciado em diversos meios,
e as pesquisas tem confirmado as vantagens que a prática apresenta e confirma\footnote{http://disqus.com/research/pseudonyms/}.

O autor destes textos se dá ao direito de não revelar seus psudônimos - embora muitos deles
sejam publicamente conhecidos - para conservar as consequências desta prática na
forma mais pura. Como comprovante desta produção, deixamos uma mensagem sobre a publicação
de textos em mídia impressa com autores internacionais,
confirmando a participação do autor desta dissertação, mas cujo
nome não consta na publicação.

!!!!!!!!!!!!!!!!!!!!!!!!!!!!!!!!!!!!!!!

de      fabi borges catadores@gmail.com por  riseup.net 
responder a     submidialogia@lists.riseup.net
para    submidialogia@lists.riseup.net
data    23 de agosto de 2011 12:26
assunto Re: [submidialogia] livro sub- publicação
lista de e-mails        <submidialogia.lists.riseup.net> Filtrar as mensagens dessa lista de e-mails
enviado por     lists.riseup.net
assinado por    riseup.net
cancelar inscrição      Cancelar a inscrição para essa lista de e-mails
        Importante principalmente porque você frequentemente lê mensagens com esse marcador.
ocultar detalhes 12:26 (6 minutos atrás)
entao, eu fui recebendo textos durante esse tempo,
alguns tao atrazados como dos sem satelites, mas muita gente mandou;

aqui os autores:

os internacionais nao sao muitos, o joni kempf (o que bebe ouro do hardware), o barbrook (futuros imaginarios),
Hamdy heda (da revolucao egipcia), o pedro soller (summerlab), maria llopis (pos porno),  talvez a bronac, nao entregou ainda.

dos brasileiros, renato fabri, ruiz, pasteur, morgana e caio, ju dornelles, mari marcassa, coletivo errorista, adriana velozo-drica, tiago pimentel, felipe fonseca, thiago novaes, lelex, felipe ribeiro(?), maira, verenilde,  bartolina silva, poro, vitoria amaro, fabib (eu),

entao, precisa publicar agora,
uma equipe para publicacao e,,, 

bjs
f

!!!!!!!!!!!!!!!!!!!!!!!!!!!!!!!!!!!!!!!

\end{itemize}

\subsection{Screencasts}

\begin{itemize}
    \item Python para áudio e música

    \item Canal Macambira
No coletivo Macambira estão sendo produzidos materiais em screencasts sobre
diversas.
    \begin{itemize}
	\item Live-Coding
	\item Raspagem de dados
    \end{itemize}
\end{itemize}



\subsection{Figusdevpack (FDP)}

Um ambiente de interação da comunidade de Python e Música 
para compartilhamento de códigos e excertos. Baseado principalmente
em documentação organizada sobre as práticas e as bibliotecas
existentes para python. Assim como scripts compartilhados que
fazem uso de objetos e módulos específicos. Este trabalho foi aceito na
maior conferência de áudio em linux, a Linux Audio Conference de 2008
(LAC2008) e está sendo reativado por mim em conjunto com Vilson Vieira
e outros desenvolvedores de áudio. Este projeto está em desenvolvimento
do site do Estúdio Livre [estudio livre] com repositório no sourceforge [source force].

As principais fontes sobre estre trabalho é a página de desenvolvimento da ideia
que está em constante mudança ~\cite{http://estudiolivre.org/tiki-index.php?page=fdp&highlight=fdp fdpel}
e, o artigo que foi aceito no LAC de 2008 ~\cite{http://www.estudiolivre.org/el-gallery_view.php?arquivoId=8221 fdplac2008}
e o repositório ~\cite{http://sourceforge.net/projects/fdpack/develop fdpsf}.

\section{Áudio e Música}

\subsection{ABeatTracker (ABT)}

Manual ABT

Linkar com ABD

\subsection{Emotional Kernel Panic (EKP)}

Em 2008, colaborando intensamente com o CDTL
(Centro de Desenvolvimento de Tecnologias Livres)\footnote{foi uma associação civil formada e desmembrada em 2008 e sediada em Recife, PE}
foi lançada a ideia de utilizar o estado do sistema operacional - especialmente o kernel linux - para
geração de sons. Surge o Emotional Kernel Panic (EKP). Desde o inicio, foram definidos três finalidades
para esta exploração do SO:

\begin{itemize}
    \item Didáticos
    \item Artísticos
    \item Monitoramento do SO
\end{itemize}

http://trac.assembla.com/audioexperiments/browser/ekp-base


\subsection{Finite Groups in Granular Synthesis (FIGGS)}

\subsection{Plugins LADSPA e lv2}

\subsection{Protocolo de compactação de áudio via wavelets, polinômios e permutações}

\subsection{Processamento de voz via redes complexas}

\subsection{AirHackTable}

\subsection{Livecoding}

\subsection{Experimentos abertos}

Audioexperiments. Estúdio livre.

\section{Web}

Difusão de informação com ênfase na facilitação
da apropriação de tecnologias e de instancias políticas.

\subsection{Tecnologias sociais de alta demanda}

\subsection{Sítios}

FDDCA

Ferramenta de comunicação

(Cadastro dos pontos?)

AA, SOS, Catalogo de Ideias, etc

Meu site pessoal


\subsection{Conteúdos}

Wiki?

\subsection{Articulação}

IRC, Emails

\section{Disponibilização e Desenvolvimento}

\subsection{Wiki}

\subsection{Trac}

\subsection{Screencasts - Vimeo}

\subsection{AA}

\subsection{Audio Experiments (Æ)}

\subsection{IRC}

\subsection{Etherpads}

\subsection{Outras fontes}
%% ------------------------------------------------------------------------- %%
\chapter{Conclusões e trabalhos futuros} %Nome do capítulo.
\label{cap:conclusao}

No capítulo anterior está um sistema conciso
que relaciona elementos musicais ao som digital. \emph{Scripts} 
implementam estas relações, e em conjunto foram nomeados \massa\ (Música
e Áudio em Sequências e Séries Amostrais). 
A exposição didática destes desenvovimentos no
capítulo anterior destina-se a facilitar a utilização
do arcabouço.

As possibilidades abertas por estes resultados envolvem a criação de interfaces de geração de ruídos e outros sons em alta fidelidade (\emph{hi-fi}), experimentos psicoacústicos e a utilização destes resultados para fins artísticos e didáticos. A incorporação de conhecimentos
em programação é bastante facilitada através de recursos audiovisuais, o que já realizamos por práticas de \emph{livecoding} e cursos focados em ferramentas especializadas, como o Puredata e o ChucK.
Está prevista a utilização destes resultados com
métodos generativos para geração de materiais artísticos.

A disposição online destes conteúdos na forma de hipertexto junto aos códigos e exemplos sonoros, todos em licenças livres, facilita colaborações e geração de subprodutos em co-autoria, e com isso a expansão da \massa\ com novas implementações e desenvolvimentos das montagens musicais.
Explorações sistemáticas de parametrizações (dos tremolos, da ADSR, etc) em alta fidelidade tem utilidade artística e é possibilidata por este trabalho com controle amostral. 

Este trabalho também teve resultados não previstos, como a formação de grupos
de interesse em torno da questão criativa aliada à computação.
Neste contexto, destaca-se o grupo
labMacambira.sf.net, que reúne colaboradores de todo o Brasil e alguns fora do país.
Este grupo
já apresentou contribuições relevantes em diferentes áreas
como Democracia Direta Digital, ferramentas de georeferenciamento e
atividades artísticas e educacionais, como cursos, workshops e apresentações artísticas. Vários destes resultados estão apresentados no Apêndice~\ref{cap:musicaExtra} e no acervo online criado, que ultrapassa 700 vídeos, diversos software originais e contribuições em software externos utilizados no mundo todo, como o Firefox, Scilab, LibreOffice, GEM/Puredata, para citar somente alguns exemplos~\cite{siteLM,wikiLM,vimeoLM}.









%-----------------PARTE PÓS-TEXTUAL------------------
%	Comando para incluir o banco de dados de bibliografia. O arquivo é o bibdatabase.bib e deve estar no mesmo diretório deste arquivo (tese.tex).

\apendice
%% ------------------------------------------------------------------------- %%
\chapter{Mapa organizacional do sítio} %Nome do capítulo.
\label{cap:mapa-sitio}



\renewcommand\bibname{Referências}
\bibliographystyle{abnt-num}
\bibliography{bibdatabase}
%-----------------FIM DO DOCUMENTO-------------------
\end{document}