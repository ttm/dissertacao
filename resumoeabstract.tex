\begin{resumo}

A representação dos elementos básicos da música, tais como notas musicais, ornamentos e estruturas intervalares, em termos do sinal digital do som é muito utilizada em softwares e rotinas para criação musical e tratamento sonoro. Não há, entretanto, um sistema unificado e conciso que relacione matematicamente esses elementos a sequências temporais derivadas do som digitalizado. Nesta dissertação, implementamos um sistema computacional, em que cada elemento musical é representado por equações que resultam claramente nas sequências temporais do som na sua representação discretizada. O primeiro elemento, a nota musical básica com duração, volume, altura e timbre, é relacionado quantitativamente às características do sinal digital. Dado este alicerce, apreendemos a realização de suas variações internas, como trêmolos, vibratos e flutuações espectrais. Isso permite sintetizar notas e sonoridades que dispertam interesse estético e com inspiração nos instrumentos musicais reais, mas cujas possibilidades são expandidas por se tratar de uma prática artificial com resultados sintéticos. A partir dessa representação das notas musicais, expande-se para sua organização em música, com o que é possível gerar estruturas musicais através de recursos consagrados, como a métrica rítmica, os intervalos de altura e os ciclos. Todas as equações estão implementadas na linguagem Python, com scripts que resultam em figuras e exemplos sonoros claros e simples. Com esse paradigma de implementação em código aberto, esta contribuição pode ser expandida em processos de co-autoria e utilizada livremente por músicos, engenheiros e outros interessados. A realidade e eficácia destas contribuições foi comprovada com a síntese de montagens musicais em cada uma das etapas assinaladas: nota básica, nota incrementada, notas em música. É possível, também, sintetizar albuns inteiros através de alguma parametrização dada por um usuário. Subprodutos deste arcabouço já foram disponibilizados na Internet na forma de músicas geradas por scripts, experimentos psicoacústicos e iniciações hackers pelo grupo labMacambira.sf.net / \#labmacambira @ Freenode.




$\phantom{linha em branco}$\\
Palavras-chave: som, acústica, áudio, código aberto, cultura digital, cultura hacker, programação

\end{resumo}

\begin{abstract}

tradução de:
Nesta dissertação, expomos os fundamentos do manuseio do som digitalizado através de recursos em código aberto como
uma forma de manifestação cultural.
Após uma brevíssima contextualização sobre o status que o código e o áudio digital possuem na cultura atual, procuramos abordar como um todo o fenômeno
físico (e psicofísico) do som, a sua representação digital e os procedimentos clássicos de manuseio. As linguagens de programação C/C++ e Python
são então abordadas já com vistas ã tarefa de lidar com o áudio digital. Um capítulo especial é dedicado às linguagens dedicadas ao áudio e à música. Ainda no campo do software,
o ambiente de desenvolvimento Unix é descrito de um ponto
de vista bem resumido e ferramental, pois é a base para todos os desenvolvimentos expostos nesta dissertação.
Os resultados antingidos com os fundamentos expostos são então sistematizados com ênfase nas ferramentas desenvolvidas.
Apresentamos também uma série de repercussões destes desenvolvimentos, em especial uma frente de ação criada: o LabMacambira.sf.net.
Por fim, como um produto adicional relacionado a esta dissertação, é delineado o mapa de exposição e comunicação no qual
os desenvolvimentos deste trabalho estão agregados. Esta estrutura é dedicada a manter as contribuições
de forma continuada e sistematizada através de webpages, wikis, repositórios, um canal IRC, aplicativos de acompanhamento de projetos
e uma das tecnologias desenvolvidas neste trabalho: a Autoregulação Algorítmica.


$\phantom{linha em branco}$\\
Keywords: audio, programming, acoustics, open source, digital culture, hacker culture

\end{abstract}
