\begin{resumo}

Este texto expõe de forma sistemática os elementos básicos da música em termos do som em sua representação
discreta. Esta abordagem é de utilidade
já comprovada pela ampla utilização feita em diversos softwares
para criação musical e tratamento sonoro. Há interesse especial
de músicos com aproximação das áreas tecnológicas
de por engenheiros
ou programadores com interesses em música ou mesmo áudio em geral.
Mesmo assim, a literatura relacionada
não expõe a teoria e realização de forma concisa, resultando em conteúdos
espalhados em diversos títulos de diferentes autores. Além disso, pouco se expõe sobre a
efetiva implementação destes recursos. Em um primeiro momento, caracterizamos
a nota musical no som digitalizado e relacionamos de forma quantitativa
as suas características musicais básicas aos aspectos do sinal digital. Em seguida,
adentramos a nota básica e expomos formas de realizar as variações internas - como trêmolos, vibratos e
variações espectrais - tão
caras às empreitadas artísticas e típicas de notas realizadas por instrumentos reais.
Por se encontrar em um nível sintético, as possibilidades são expandidas e, com elas,
os recursos artísticos disponíveis para criação. Por fim, saímos da canônica
unidade musical mínima que é a nota e lidamos com a organização de notas musicais
em música. Neste momento são apresentados recursos musicais consagrados para geração
de estruturas musicais através das notas, como a métrica rítmica, os intervalos de altura e os
ciclos.
Como resultados, além desta exposição sistemática com equações de aplicação explícita,
está disponibilizado um conjunto de scripts
com as implementações computacionais destes procedimentos e exemplos sonoros audíveis em equipamentos
comuns. Dentre estes exemplos sonoros, encontramos sons com ornamentos e outros exemplos simples. Também
dispomos montagens musicais com fins de explorar os recursos apresentados. Ao final, apresentamos
uma implementação capaz de sintetizar discos inteiros através de alguma parametrização.
Este arcabouço dispertou interesses e foi responsável por empreitadas coletivas diversas,
que fogem ao escopo deste trabalho e estão dispostas em apêndices sumplementares.



Nesta dissertação, expomos os fundamentos do manuseio do som digitalizado através de recursos em código aberto como
uma forma de manifestação cultural.
Após uma brevíssima contextualização sobre o status que o código e o áudio digital possuem na cultura atual, procuramos abordar como um todo o fenômeno
físico (e psicofísico) do som, a sua representação digital e os procedimentos clássicos de manuseio. As linguagens de programação C/C++ e Python
são então abordadas já com vistas ã tarefa de lidar com o áudio digital. Um capítulo especial é dedicado às linguagens dedicadas ao áudio e à música. Ainda no campo do software,
o ambiente de desenvolvimento Unix é descrito de um ponto
de vista bem resumido e ferramental, pois é a base para todos os desenvolvimentos expostos nesta dissertação.
Os resultados antingidos com os fundamentos expostos são então sistematizados com ênfase nas ferramentas desenvolvidas.
Apresentamos também uma série de repercussões destes desenvolvimentos, em especial uma frente de ação criada: o LabMacambira.sf.net.
Por fim, como um produto adicional relacionado a esta dissertação, é delineado o mapa de exposição e comunicação no qual
os desenvolvimentos deste trabalho estão agregados. Esta estrutura é dedicada a manter as contribuições
de forma continuada e sistematizada através de webpages, wikis, repositórios, um canal IRC, aplicativos de acompanhamento de projetos
e uma das tecnologias desenvolvidas neste trabalho: a Autoregulação Algorítmica.


$\phantom{linha em branco}$\\
Palavras-chave: som, acústica, áudio, código aberto, cultura digital, cultura hacker, programação

\end{resumo}

\begin{abstract}

tradução de:

Nesta dissertação, expomos os fundamentos do manuseio do som digitalizado através de recursos em código aberto como
uma forma de manifestação cultural.
Após uma brevíssima contextualização sobre o status que o código e o áudio digital possuem na cultura atual, procuramos abordar como um todo o fenômeno
físico (e psicofísico) do som, a sua representação digital e os procedimentos clássicos de manuseio. As linguagens de programação C/C++ e Python
são então abordadas já com vistas ã tarefa de lidar com o áudio digital. Um capítulo especial é dedicado às linguagens dedicadas ao áudio e à música. Ainda no campo do software,
o ambiente de desenvolvimento Unix é descrito de um ponto
de vista bem resumido e ferramental, pois é a base para todos os desenvolvimentos expostos nesta dissertação.
Os resultados antingidos com os fundamentos expostos são então sistematizados com ênfase nas ferramentas desenvolvidas.
Apresentamos também uma série de repercussões destes desenvolvimentos, em especial uma frente de ação criada: o LabMacambira.sf.net.
Por fim, como um produto adicional relacionado a esta dissertação, é delineado o mapa de exposição e comunicação no qual
os desenvolvimentos deste trabalho estão agregados. Esta estrutura é dedicada a manter as contribuições
de forma continuada e sistematizada através de webpages, wikis, repositórios, um canal IRC, aplicativos de acompanhamento de projetos
e uma das tecnologias desenvolvidas neste trabalho: a Autoregulação Algorítmica.

$\phantom{linha em branco}$\\
Keywords: audio, programming, acoustics, open source, digital culture, hacker culture

\end{abstract}
