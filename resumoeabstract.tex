\begin{resumo2}
\vspace{-10mm}
FABBRI, R. \textbf{\ABNTtitulodata}. 2012. 151p. Dissertação (Mestrado) - Instituto de Física de São Carlos, Universidade de São Paulo, 2012.
\vspace{15mm}

A representação dos elementos básicos da música - tais como notas, ornamentos e estruturas intervalares - em termos do som discretizado é bastante utilizada em software e rotinas para criação musical e tratamento sonoro. Não há, entretanto, uma abordagem concisa que relacione estes elementos às amostras sonoras. Nesta dissertação, cada elemento musical é descrito por equações que resultam diretamente nas sequências temporais do som em sua representação discretizada. O elemento fundamental, a nota musical básica com duração, volume, altura e timbre, é relacionado quantitativamente às características do sinal digital. As variações internas, como tremolos, vibratos e flutuações espectrais, também são contempladas, o que permite sintetizar notas com inspiração nos instrumentos musicais reais além de sonoridades novas. A partir desta representação das notas, dispomos de recursos para a geração de estruturas musicais, como a métrica rítmica, os intervalos de altura e os ciclos. As equações deram origem a uma caixa de ferramentas computacionais: \emph{scripts} que realizam cada equação e geram exemplos sonoros simples. Nomeamos \massa, Música e Áudio em Sequências e Séries Amostrais, este ferramental e sua eficácia foi comprovada com a síntese de pequenas peças usando notas básicas, notas incrementadas e notas em música. É possível, também, sintetizar álbuns inteiros através de colagens dos scripts e parametrização especificada pelo usuário. Com o paradigma de implementação em código aberto, a (caixa de ferramentas) \massa\  pode ser expandida em processos de co-autoria e usada livremente por músicos, engenheiros e outros interessados. De fato, o sistema já foi empregado por usuários externos para a produção de músicas, apresentações artísticas, experimentos psico-acústicos e a difusão da linguagem computacional através do apelo lúdico dos artefatos audiovisuais.


$\phantom{linha em branco}$\\
Palavras-chave: Som. Áudio. Áudio digital. Acústica. Música. Música digital. Psicofísica. Alta fidelidade. Hi-fi. Hifi. Síntese sonora. Síntese musical. Python. Livecoding. Arte e tecnologia. Código aberto. Cultura hacker. Cultura digital. Cultura livre.

\end{resumo2}


\afterpage{\blankpage}

\begin{abstract2}
\vspace{-10mm}
FABBRI, R. \textbf{Music on digital audio: psychophysical description and toolbox}. 2012. 151p. Dissertação (Mestrado) - Instituto de Física de São Carlos, Universidade de São Paulo, 2012.
\vspace{15mm}

The representation of the basic elements of music - such as notes, ornaments and intervalar structures - in terms of discrete audio signal is often used in software for music creation and design. Nevertheless, there is no unified approach that relates these elements to the sound discrete samples. In this dissertation, each musical element is described by equations that represent the sonic time samples, which are then implemented in scripts within a software toolbox, referred to as \massa\ (Music and Audio in Sequences and Samples). The fundamental element, the musical note with duration, volume, pitch and timbre, is related quantitatively to the characteristics of the digital signal. Internal variations, such as tremolos, vibratos and spectral fluctuations, are also considered, which enables the synthesis of notes inspired by real instruments and new sonorities. With this representation of notes, resources are provided for the generation of musical structures, such as rhythmic meter, pitch intervals and cycles. The efficacy of \massa\ was confirmed by the synthesis of small musical pieces using basic notes, incremented notes and notes in music. It is possible to synthesize whole albums through collage of the scripts and parameterization specified by the user. With the paradigm of open source implementation, \massa toolbox can be expanded in co-authorship processes and used freely by musicians, engineers and other interested parties. In fact, \massa has already been employed by external users for diverse purposes which include music production, artistic presentations, psychoacoustic experiments and computer language diffusion where the appeal of audiovisual artifacts is exploited.


$\phantom{linha em branco}$\\
Keywords: Sound. Audio. Digital audio. Acoustics. Music. Digital music. Psicophysics. High fidelity. Hi-fi. Hifi. Sound synthesis. Musical synthesis. Python. Livecoding. Art and technology. Open source software. Hacker culture. Digital culture. Free culture.


\end{abstract2}
