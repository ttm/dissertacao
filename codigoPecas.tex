\chapter{Código Computacional das Peças Musicais}
\label{cap:codigoPecas}

\subsection{Reduced-fi}

\subsection{Quadros Sonoros}

