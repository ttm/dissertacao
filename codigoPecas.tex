\chapter{Código Computacional das Peças Musicais}
\label{cap:codigoPecas}
Todas as peças a seguir se propõem a exemplificar as relações apresentadas no capítulo~\ref{cap:resultados} e são disponibilizadas online junto ao toolbox \massa.\cite{MASSA}


\subsection{Quadros sonoros}\label{ap:quadros}
Montagem musical 'quadros sonoros' para demonstração da mixagem pela soma direta de sequências amostrais. Resultam em 5 pequenas peças de sonoridades estáticas. Peça demonstrativa dos conceitos apresentados na seção~\ref{sec:notaDisc}.

\code{Quadros sonoros 1-5}{scripts/pecas2.1/quadrosSonoros.py}

\subsection{Reduced-fi}\label{ap:reduced}
Pequena peça musical para demonstração da concatenação de sequências amostrais como notas musicais. Resulta em uma pequena peça de 25 segundos. Peça demonstrativa dos conceitos apresentados na seção~\ref{sec:notaDisc} e disponibilizada online junto ao toolbox \massa.

\code{reduced-fi}{scripts/pecas2.1/reduced-fi-limpo.py}
