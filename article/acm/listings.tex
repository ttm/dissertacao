\documentclass{scrreprt}
% \documentclass[12pt]{article}
\usepackage{graphicx}

\usepackage{xr}
\externaldocument{sample-acmsmall}

\usepackage[usenames,dvipsnames]{xcolor}
\usepackage{hyperref}
\hypersetup{
        colorlinks,
        linkcolor={red!50!black},
        citecolor={blue!50!black},
        urlcolor={blue!80!black}
}
%
\usepackage{graphicx} %allows for including images
\usepackage{biblatex}

\newcommand{\massa}{{\large \textsc{mass}}}

\makeatletter
\newcommand{\usetocfromothersource}[1]{%
  \begingroup
  \IfFileExists{#1}{%
    \@input{#1}% 
    \@nobreakfalse
  }{}%
  \endgroup
}
\makeatother

\def\thesection{SI-\arabic{section}}

\begin{document}

\title{Listings of equations, figures, tables and sections of the
article '{\it \bf Musical elements in the discrete-time representation of
sound}' and of the scripts in the MASS toolbox}

\author{\\\\\\ Renato Fabbri (USP),\\
Vilson Vieira da Silva Junior (The Grid),\\
Ant\^onio Carlos Silvano Pessotti (Unimep),\\
D\'ebora Cristina Corr\^ea,\\
Osvaldo N. Oliveira Jr. (USP)
}

\maketitle

\begin{abstract}
The article is the main document of the \massa\ toolbox.
Being it of considerable length and complexity,
this document contains listings of its elements to facilitate
its navigation, apprehension and general usage.
\end{abstract}

\tableofcontents

\clearpage
\section{Table of Contents of the article}
\makeatletter
\let\Hy@linktoc\Hy@linktoc@none
\makeatother
\usetocfromothersource{sample-acmsmall.toc}

\clearpage
\section{Equations}

\begin{table*}[htp!]
\centering
\caption{Equation numbers and their descriptions.
All these equations are implemented in file \texttt{src/sections/2.py}.}
\begin{tabular}{ c | p{12cm} }
   Number & Description \\\hline
 \ref{eq:dur} & relation between number of samples and duration \\
 \ref{eq:potencia} & power of the wave \\
 \ref{decibels} & decibels by difference by means of the power of each wave \\
 \ref{eq:ampVol} & double amplitude implies $\approx 6dB$ \\
 \ref{eq:potVol} & double power implies $\approx 3dB$ \\
 \ref{eq:dobraVol} & double volume ($10dB$) implies a factor of $\approx 3.16$ for amplitude \\
 \ref{ampDec} & direct relation between variations in amplitude and decibels \\
 \ref{periodicidade} & equivalences in a periodic sound with respect to wavelength, frequency and sample rate \\
 \ref{sinusoid} & sample amplitudes in a sinusoid \\
 \ref{sawTooth} & sample amplitudes in a sawtooth wave \\
 \ref{triangular} & sample amplitudes in a triangular wave \\
 \ref{square} & sample amplitude in a square wave \\
 \ref{sampleandoFormaDeOnda} & samples in a sound derived from a sampled wave period \\
 \ref{recomposicaoFourier} & reconstruction of samples from the Fourier components \\
 \ref{moduloEfase} & reconstruction of real samples (e.g. for audio) from Fourier components \\
 \ref{coefsPareados} & number of pairs of Fourier coefficients which are related to the same frequency \\
 \ref{equivalenciasFreqs} & indexes of equivalent frequencies and coefficients for real signals \\
 \ref{equivalenciasModulos} & equal modules between samples of real signals \\
 \ref{equivalenciasFases} & equivalence in phases between samples of real signals \\
 \ref{eq:reconsCompleta} & complete reconstruction of the signal using Fourier components and previous equations \\
 \ref{eq:notaBasica} & sample sequence related to a basic note \\
 \ref{periodoUnico} & sample sequence of a period of an arbitrary waveform \\
 \ref{eq:notaBasicaTimbre} & note samples derived from a sampled waveform \\
 \ref{eq:distOuvidos} & distance of a (sound) source to each ear given the distance between the ears and an (x,y) position of the source \\
 \ref{eq:dti} & Interaural Time Difference (ITD), the time difference of a sound reaching each ear \\
 \ref{eq:dii} & Interaural Intensity Difference (IID), the intensity difference (in decibels) of a sound reaching each ear \\
 \ref{eq:locImpl} & ITD and IID in terms of sample delays and their amplitudes \\
 \ref{eq:angulo} & azimuthal angle of a (x, y) source \\
 \ref{eq:mixagem} & samples that result from mixing sounds \\
 \ref{eq:concatenacao} & samples that result from concatenating sounds \\
\end{tabular}
\end{table*}

\begin{table*}[htp!]
\centering
\caption{Equation numbers and their descriptions.
All these equations are implemented in file \texttt{src/sections/3.py}.}
\begin{tabular}{ c | p{12cm} }
   Number & Description \\\hline
 \ref{eq:lut} & sample sequence generated by means of a lookup table (LUT) \\
 \ref{freqLinear} & frequency at each sample in a linear transition of frequency \\
 \ref{indiceLinear} & indices of a LUT in a linear transition of frequency \\
 \ref{serieAmostralLin} & sample sequence obtained through a LUT in a linear transition of frequency \\
 \ref{freqExponencial} & frequency at each sample in an exponential transition of frequency (linear pitch) \\
 \ref{indiceExponencial} & indices of a LUT in an exponential transition of frequency (linear pitch) \\
 \ref{serieAmostralLog} & sample sequence obtained through a LUT in an exponential transition of frequency \\
 \ref{seqAmp} & amplitude factors at each sample in an exponential transition of amplitude ($\approx$ linear volume) \\
 \ref{transAmp} & sample sequence with an exponential transition of amplitude ($\approx$ linear volume) \\
 \ref{seqAmpLin} & amplitude factors at each sample in a linear transition of amplitude \\
 \ref{seqAmpDB} & sample sequence in an exponential transition of amplitude ($\approx$ linear volume) with difference given in decibels \\
 \ref{eq:conv} & sample sequence obtained through the convolution of two other sequences (e.g. for applying FIR filters) \\
 \ref{eq:diferencas} & difference equation (e.g. for applying IIR filters) \\
 \ref{eq:passa-baixas} & IIR coefficients for a simple, useful and well-behaved low-pass filter \\
 \ref{eq:passa-altas} & IIR coefficients for a simple, useful and well-behaved high-pass filter \\
 \ref{eq:varAux} & auxiliary variables for the following band-pass and band-reject filters \\
 \ref{eq:passa-banda} & IIR coefficients for a simple, useful and well-behaved band-pass filter \\
 \ref{eq:rejeita-banda} & IIR coefficients for a simple, useful and well-behaved band-reject filter \\
 \ref{eq:branco} & Fourier coefficients of a white noise \\
 \ref{eq:rosa} & Fourier coefficients of a pink noise \\
 \ref{eq:marrom} & Fourier coefficients of a brown noise \\
 \ref{eq:azul} & Fourier coefficients of a blue noise \\
 \ref{eq:violeta} & Fourier coefficients of a violet noise \\
 \ref{eq:preto} & Fourier coefficients of a black noise \\
 \ref{vbrGamma} & indices for a vibrato given its frequency and using a LUT \\
 \ref{vbrAux} & samples for applying a vibrato \\
 \ref{vbrF} & frequency at each sample of a sound with vibrato \\
 \ref{vbrGamma2} & indices for LUT in a sound with vibrato \\
 \ref{vbrT} & sample sequence of a sound with vibrato \\
 \ref{trA} & amplitude at each sample for a tremolo \\
 \ref{trT} & sample sequence of a sound with tremolo \\
 \ref{eq:fmEsp} & components in FM synthesis when both modulator and carrier are sines \\
 \ref{eq:Bessel} & the Bessel function \\
 \ref{eq:specAM} & components in AM synthesis when both modulator and carrier are sines \\
 \ref{fmGammaAux} & indices for LUT in modulator of an FM synthesis \\
 \ref{fmAux} & sample sequence of a modulator in an FM synthesis \\
 \ref{fmF} & frequeny at each sample of a sound derived from FM synthesis \\
 \ref{fmGamma} & indices for the final signal in FM synthesis using LUT \\
 \ref{fmT} & sample sequence of a sound generated through FM and using LUT \\
 \ref{amA} & amplitude at each sample in a sound generated though AM \\
 \ref{amT} & sample sequence of a sound generated through AM and using LUT \\
\end{tabular}
\end{table*}

\begin{table*}[htp!]
\centering
\caption{Equation numbers and their descriptions.
All these equations are implemented in file \texttt{src/sections/3.py}.}
\begin{tabular}{ c | p{12cm} }
   Number & Description \\\hline
 \ref{eq:vinculos} & an example of bonds between musical characteristics \\
 \ref{eq:fDoppler} & relation between frequencies and speed in the Doppler effect \\
 \ref{eq:aDoppler} & relation between position, speed and amplitude in the Doppler effect \\
 \ref{eq:ffDoppler} & relation between position, speed and amplitude in the Doppler effect \\
 \ref{eq:p1rev} & samples of a FIR filter for the first period of a reverberation \\
 \ref{eq:p2rev} & samples of a FIR filter for the second period of a reverberation \\
 \ref{eq:rev} & samples of the FIR filter for a reverberation (considering both first and second periods) \\
 \ref{eq:adsr} & amplitude factors for each sample in an ADSR envelope \\
 \ref{eq:adsrApl} & sample sequence of a sound with an ADSR envelope \\
\end{tabular}
\end{table*}

\begin{table*}[htp!]
\centering
\caption{Equation numbers and their descriptions.
All these equations are implemented in file \texttt{src/sections/4.py}.}
\begin{tabular}{ c | p{12cm} }
   Number & Description \\\hline
 \ref{escSim} & perfectly symmetric scales in each octave with the twelve semitones \\
 \ref{eq:escalas} & diatonic scales \\
 \ref{eq:relacaoDia} & the succession of tones and semitones of a diatonic scale \\
 \ref{eq:escalasMenores} & sequences of semitones for the three types minor scales \\
 \ref{eq:serieHarmonica} & harmonic series in terms of semitones \\
 \ref{triades} & triads (chords constituted by thirds) \\
 \ref{eq:rhythmicUnit} & a convention to specify a unit of rhythmic division or agglomeration \\
 \ref{eq:groups} & definition of algebraic groups \\
\end{tabular}
\end{table*}




\subsection{In Section 4 - Notes in music}

\subsection{In Section 3 - Variations of the basic note}
\subsection{In Section 4 - Notes in music}

\clearpage
\section{Figures}

\clearpage
\section{Tables}

\clearpage
\section{Scripts}
\subsection{For all equations in each chapter}
\clearpage
\subsection{To render musical pieces}
\clearpage
\subsection{To render the figures used in the article}

\clearpage
\section{Other documents}
Latex files. PDF files.
MSc Dissertation.

\section{Final considerations}

\end{document}
