%% ------------------------------------------------------------------------- %%
\chapter{Fundamentos teóricos e técnicos} %Nome do capítulo.
\label{cap:intro} %Rótulo para futura referência ao capítulo. Em qualquer lugar da tese, você poderá citar este capítulo através de ~\ref{cap:introducao}. Você escolhe o argumento de \label e pode ser qualquer coisa (Ex: \label{Procedimento_Experimental})

\section{Áudio}

\subsection{Som como fenômeno físico}

Ver livros em casa.

\subsection{Som digitalizado}

Ver livros em casa.

Representação digital do som.

Padrões PCM. Padrões compactados.

\subsection{Processamento de áudio}

Ver livros em casa.

Procedimentos básicos. Processamento
espectral. Outros processamentos.

Vincular a um apêndice com excertos de código.

\section{Linguagens e modularidade}

Discursar brevemente sobre as diversas linguagens,
inclusive voltadas para áudio. E sobre a reutilização
de trechos de código.

\section{Sistema Operacional e Web}

\subsection{Breve Histórico}

O hardware estava ficando mais ambundante, sendo replicado em 
vários lugares. Modelos eram vendidos com diferenças de hardware 
de mínimas a discrepantes. Neste contexto, começou-se a
pensar em uma camada de código que abstraísse o hardware, assim
como a linguagem C se propôs.

Iniciou-se assim o desenvolvimento de Sistemas Operacionais (SO).
Um SO é um software que gerencia os recursos do computador e ajuda
também em abstrair o hardware. Orientamos a esta excelênte linha do
tempo da Wikipédia em língua inglesa \cite{solinhadotempo}:



\subsection{Unix e GNU/LINUX}

Resumo de utilização. No apêndice o resumo de:
The Unix programing envinoment
e
Running Linux

\subsection{Web e Browsers}

Dados do livrinho do google.

Possibilidades de uso.

Breve Histórico.




