%% ------------------------------------------------------------------------- %%
\chapter{Fundamentos teóricos e técnicos} %Nome do capítulo.
\label{cap:intro} %Rótulo para futura referência ao capítulo. Em qualquer lugar da tese, você poderá citar este capítulo através de ~\ref{cap:introducao}. Você escolhe o argumento de \label e pode ser qualquer coisa (Ex: \label{Procedimento_Experimental})

\section{Áudio}

\subsection{O som como fenômeno físico}

O som é o fenômeno físico de propagação de ondas mecânicas. Ele pode
somente ocorrer na presença da matéria, ou seja, o vácuo é um
isolante acústico perfeito.

\subsection{A audição e a psicofísica}

A audição é definida com a percepção de vibrações mecânicas com
ou sem a contribuição direta do tato. Em diversos
seres vivos, encontramos um órgão dedicado a captar e transmitir
estas vibrações mecânicas (som) para o sistema nervoso.

O ouvido humano cumpre exatamente este papel. Um esquema básico
do ouvido humano conta com 3 divisões. No ouvido externo tem a orelha.
No médio, tem os martelinhos e o tímpano. No ouvido interno tem a cóclea
e os cíclios. Uma visualização simplificada do ouvido humano está abaixo.


FIGURA MOSTRANDO ENTRADA DE VIBRAÇÃO E TRANSMISSÃO AO CÉREBRO

Vale notar que a audição conta em grande parte com o tato e potencialmente
outros órgãos, como o estômago.

Já a psicofísica busca relacionar características físicas do fenômeno sonoro
com as percepções individuais causadas\footnote{PsycoPy para escrever experimentos psicofísicos em Python: http://www.psychopy.org/}.
Importantes exemplos destas relações
é a escala logarítimica que a percepção linear das alturas assumem no
domínio da frequência e as curvas iso-audíveis (conhecidas como curvas de Fletcher-Münsen).

Outro importante aspecto da psicofísica do som - e da música - é a resolução da percepção
e a capacidade de notar características do espectro em contextos específicos. O mascaramento
de frequências é a base na qual se faz compressores de tamanhos de arquivos de áudio. Sabe-se
que, por exemplo, no extremo grave, não conseguimos distinguir duas senóides a quase um tom de
diferença. Outros fenômenos psicoacústicos estudados com fins tecnológicos, científicos e artísticos são: espacialidade,
relação da percepção de volume com espectro agudo e reverberação, compreensibilidade de sinais de fala, etc.


\subsection{Som digitalizado}

Ver livros em casa.

Representação digital do som.

Padrões PCM. Padrões compactados.

\subsection{Processamento de áudio}

Ver livros em casa.

Procedimentos básicos. Processamento
espectral. Outros processamentos.

Vincular a um apêndice com excertos de código.

\section{Linguagens e modularidade}

\subsection{Breve histórico}

Discursar brevemente sobre as diversas linguagens,
inclusive voltadas para áudio. E sobre a reutilização
de trechos de código.

\subsection{C/C++}

\subsection{Python}


\section{Sistema Operacional e Web}

\subsection{Breve histórico}

O hardware estava ficando mais ambundante, sendo replicado em 
vários lugares. Modelos eram vendidos com diferenças de hardware 
de mínimas a discrepantes. Neste contexto, começou-se a
pensar em uma camada de código que abstraísse o hardware, assim
como a linguagem C se propôs.

Iniciou-se assim o desenvolvimento de Sistemas Operacionais (SO).
Um SO é um software que gerencia os recursos do computador e ajuda
também em abstrair o hardware. Orientamos a esta excelênte linha do
tempo da Wikipédia em língua inglesa \cite{solinhadotempo}:



\subsection{Unix e GNU/LINUX}

Resumo de utilização. No apêndice o resumo de:
The Unix programing envinoment
e
Running Linux

\subsection{Web e Browsers}

Dados do livrinho do google.

Possibilidades de uso.

Breve Histórico.

No apêndice resumo de:
http://www.html5rocks.com/en/tutorials/internals/howbrowserswork/



