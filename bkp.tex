\documentclass[
 aip,
 jmp,
 amsmath,amssymb,
 reprint,
]{revtex4-1}
\usepackage{graphicx}
\usepackage{grffile}
\usepackage{dcolumn}
\usepackage{bm}
\usepackage{multirow}
\usepackage{color}

\newcommand{\massa}{{\large \textsc{massa}}}
\newcommand{\mass}{{\large \textsc{mass}}}
\newcommand{\figgus}{{\large \textsc{figgus}}}


\begin{document}
\preprint{AIP/123-QED} 

\title{Psychophysics of musical elements in discrete-time representation of sound}

\author{Renato Fabbri}
 \homepage{http://www.estudiolivre.org/el-user.php?view\_user=gk}
 \email{renato.fabbri@gmail.com}
  \affiliation{ 
Instituto de F\'isica de S\~ao Carlos, Universidade de S\~ao Paulo (IFSC/USP)
}

\author{Luciano da Fontoura Costa}
  \homepage{http://cyvision.ifsc.usp.br/~luciano/}
  \email{ldfcosta@gmail.com}
 \altaffiliation[Also at ]{IFSC-USP}

 \author{Osvaldo N. Oliveira Jr.}
  \homepage{www.polimeros.ifsc.usp.br/professors/professor.php?id=4}
  \email{chu@ifsc.usp.br}
 \altaffiliation[Also at ]{IFSC-USP}

\date{\today}
\begin{abstract}

    The representation of the basic elements of music - such as notes, ornaments and intervalar structures - in terms of discrete audio signal is often used in software for music creation and design. Nevertheless, there is no unified approach that relates these elements to the sound discrete samples. In this article, each musical element is described by equations in sonic time samples, which are all implemented in scripts within an open source software toolbox, referred to as \massa\ (Music and Audio in Sequences and Samples). The fundamental element, the musical note with duration, volume, pitch and timbre, is related quantitatively to the characteristics of the discrete-time signal. Internal variations, such as tremolos, vibratos and spectral fluctuations, are also considered, which enables the synthesis of notes inspired by real instruments and new sonorities. With this representation of notes, resources are provided for the generation of musical structures, such as rhythmic meter, pitch intervals and cycles. The efficacy of these physical descriptions of basic musical elements was confirmed by the synthesis of small musical pieces within each frame: basic notes, incremented notes and notes in music. It is possible to synthesize whole albums through collage of the scripts and parameterization. 
    The sample-based analitical description, and the paradigm of open source implementation, enables cientific experiments in precise and trustful ways. In fact, \massa\ has already been employed by external users for diverse purposes. Among these, it is mentioned an acoustic effect recognized by diverse individuals in mailing lists but not found in literature, and uses related to art and education.

\end{abstract}
\pacs{*43.66.-x,43.66.+y,05.10.-a} % PACS, the Physics and Astronomy
\keywords{psychophysics, acoustics, statistics, signal processing, digital audio, music}
\maketitle

\section{\label{sec:level1}Introduction}

Representing musical structures and artifices by it's discrete sound characteristics is the purpuse of this work. The results include mathematical relations and it's computer program implementations. Next section exposes the theoretical description, which is implemented as scripts in a one-to-one relation to the equations.

\subsection{Sound and digital audio}

Sound is a longitudinal wave of mechanical pressure. The frequency bandwidth between $20Hz$ and $20kHz$ is appreciated by human hearing system with boudaries dependent on the person, climate conditions and sonic charachteristics itself~\cite{Roederer}. If considereded the speed of sound of $\approx 343.2 m/s$, this limits corresponds to $\frac{343.2}{20} = 17.16\,m$ and $\frac{343.2}{20000}=17.16\,mm$.

Human perception of sound envolves captivation by bones, stomach, ears, tranfer functions of head and processing dorso and nervous system. Besides that, the ear is a dedicated organ to the capture of this waves. Its mechanism decomposes sound into its sinusoidal spectrum and delivers them to the nervous system. This sinusoidal components are crucial to musical fenomena, as one can observe in the composition of sounds with musical interest and in tunnings and scales. Subsection~\ref{subsec:dicNote} exposes the presence of sinusoids in discrete-time audio and charachterizes a basic musical note.

The representation of sound is called audio (although these terms are often used without distinction), and this can be provenient from caption by microphones or from sythesis. Often enough, digital audio is specified by protocols that eases file storage and transfering, in cost of a direct representation or even some loss in quality. Standard representation of digital audio, on the other hand, consists of samples equally spaced by $\lambda_s$ durations in time, with each sample spacified by a sample number of bits. This is called the Pulse Code Modulation representation of sound (PCM). A PCM digital sound is charachterized by it's sampling frequency $f_s=\frac{1}{\lambda_s}$, also called sampling rate, and bit depth, which is the number of bits used of representing the amplitude of each sample. Figure~\ref{fig:PCM} shows $25$ samples of a PCM áudio with $4$ bits each. The $2^4=16$ possible steps for each sample, together with the regular spacing $\lambda_s$ between them, introduces a quantization error. This noise, caused by this errors, diminishes as these spacing diminishes.


\begin{figure*}
    \centering
        \includegraphics[width=\textwidth]{pcm}
        \caption{Pulse Code Modulation (PCM) audio: an analogic signal is represented by 25 samples with 4 bits each.}
        \label{fig:PCM}
\end{figure*}




By the Nyquist theorem, it is known that half the sampling frequency is the maximum frequency of the signal. Thus, it is necessary to have a sampling frequency at least twice the highest frequency heard by humans $f_s \geq 2\times 20kHz = 40kHz$. This is the basis for the use of the sampling frequencies $44.1kHz$ and $48kHz$, standards in Compact Disks (CD) and Broadcast systems (Radio and TV), respectively.

\subsection{Sonic art and musical theory}

A common definition for music is the art made by sounds and silences. For the average listener -- and a reasonable part of specialists -- the notion of music presupposes rhythmic and pitch organization such as explained in subsection~\ref{subsec:notesMusic}. Music from the twentieth century enlarged this traditional comprehension of music. This occured in concert music, specially in the concrete, electronic and electroacoustic styles. On the last decade of the century, it was evident that popular music has also incorporated sounds without defined pitch and temporal organization out of simple metrics. Even though, the note stands paradigmatic as a 'fundamental unit' of musical structures and, in practice, it can unfold in sounds that observe this recent developments. The definition and expansion of the musical note as the fundamental unit of music is approached in subsections~\ref{subsec:discNote} and~\ref{subsec:internalVar}, respectively. Subsection~\ref{subsec:notesMusic} tackles the organization of musical notes in a higher level~\cite{Wisnick,Webern,Lerdhal,Cook,Lacerda}.

Musical theory embody topics as diverse as psicho-acoustics, cultural manifestations and formalisms. The section~\ref{sec:results} point this topics as needed and designate external complements~\cite{Zamacois,Schoenberg,microsound}.

\subsection{Computational implementation}

The results presented in this article are implemented as scripts, i.e.\ small computer programs implemented using accessible technologies for better distribution and validation. This constitute the \massa\ toolbox, available in public domain in an open Git repository~\cite{gitBook}. This scripts are written in Python and make use of external libraries Numpy and Scikits/Audiolab that performs calls to Fortran routines for better computational efficiency. Part of this code has been trascribed to JavaScript and native Python with readiness, what points to uses of this contribution in Web browsers such as Firefox and Chromium~\cite{numpy, audiolab, tutpython, python}.

This are all open technologies, that is, published using licenses that allows copy, distribution and use of any part for research and derivatives. This way, the work here presented is available and eases co-authorship processes~\cite{Raymond,Lessig}. 

\subsection{Objectives}
\label{subsec:objectives}
The main goal of this article is to present a concise set of relations among musical basic elements and sequences of PCM audio samples. The next section is a minimum text in which music elements are presented side-by-side with the discrete-time samples they result. As validation and sharing, implementations on computer code of these relations and little musical pieces where gathered the \massa\ toolbox, available online.

Secondary objectives include presenting a framework of sound synthesis with control at sample level, with potential uses in psychoacoustical experiments and high-fidelity synthesis. The didatic presentation of the content favours use and apprehension on a problem which calls diverse topics to be tackled: signal processing, music and psycho-acoustics, to name just a few.

\subsection{Related work}
Due to the general interest, and number of knowledge areas involved, there is a number of books and computer implementations that are of interest or present similarities to what is presented in this work. A more detailed comparrison of them is pointed out in the bibliography~\cite{dissertacao}. There is almost no articles which could be found on the topic. In summary, there are computer implementations that use this analitycal descriptions implicitly, but there is no such a concise and mathemathical description of the processes implemented, as they aim to be libraries for sound and music. There are books on the topic that cover various aspects of effects and physical modeling, but none of them carry a concise description of musical elements and structures, but focus on aspects of musical sounds and ways to mimic traditional instruments.

\section{Characterization of the discrete-time musical note}
In diverse artistic and theoretical contexts, music is though of as being constituted by units called notes and this units taken as "atoms" that constitute music itself~\cite{Wisnick, Lovelock, Webern}.
Nowadays, this notes are undestood as a central element of certain musical paradigms. In a cognitive sense, the notes are seen as discretizations that eases and enrich the flow of information through music~\cite{Roederer, Lacerda}.
Canonically, a musical note posses at least duration, volume, pitch and tibre~\cite{Lacerda}. These are qualities which can be managed quantitatively, dictated by the evenly time spaced sound samples.

All the relations on this section are implemented at the file \emph{eqs2.1.py} of the \massa\ toolbox. Musical pieces \emph{5 sonic portraits} and \emph{reduced-fi} are available online as proof of concept.

\subsection{Duration}
The sample frequency $f_s$ is defined as the number of samples in each second of the discrete-time signal. Let $T_i=\{t_i\}$ be an ordered set of real samples separated by $\delta_s=1/f_s$ seconds. A Musical note of duration $\Delta$ seconds is presented as a sequence of $\lfloor \Delta . f_s \rfloor $ samples. That is, taken the integer part of the multiplication, it is admited an error of at most $-\lambda_a$ seconds, which, for musical porpuses, are usually fine, for $f_s=44.1kHz \;\;\Rightarrow\;\;\lambda_s=\frac{1}{44100}\approx 23$ microseconds. It is reasonable to state:


\begin{equation}\label{eq:dur}
T_{i}^{\Delta}={\{t_i\}}_{i=0}^{\lfloor \Delta . f_a \rfloor -1}
\end{equation}

Being $\Lambda=\lfloor \Delta . f_a \rfloor$ the number of samples in the sequence, so that $T_i=\{t_i\}_0^{\Lambda-1}$.

\subsection{Volume}\label{subsec:volume}
The sensation of sound volume depends on reverberation and harmonic distribution, among other characteristics worked on section~\ref{sec:varInternas}. One can get variations of volume through the potency of the wave~\cite{Chowning}:

\begin{equation}\label{eq:potencia}
pot(T_i)=\frac{\sum_{i=0}^{\Lambda -1} t_i^2}{\Lambda}
\end{equation} 

The final volume is dependent on the speakers amplification of the signal. Thus, what matters is the relative potency of a note in relation to the ones around it or the potency of a music section in relation to the rest. Differences in volume are measured in decibels, and these are calculated directly from the amplitudes through energy or potency:

\begin{equation}\label{decibels}
V_{dB}=10log_{10}\frac{pot(T^{'}_i)}{pot(T_i)}
\end{equation}

The quantity $V_{dB}$ has the decibel unit ($dB$). 
To each $10dB$ it is associated a "doubled volume".
Handy references are the $10dB$ for each step in the intensity scale: \emph{pianissimo}, \emph{piano}, \emph{mezzoforte}, \emph{forte} e \emph{fortissimo}. Other useful references are que equivalent in $dB$ of doubling amplitude or potency:

\begin{equation}\label{eq:ampVol}
\begin{split}
t_i^{'}=2 . t_i \Rightarrow pot(T^{'}_i)=4 . pot(T_i) \Rightarrow \\ \Rightarrow V^{'}_{dB}=10log_{10} 4 \approx 6 dB
\end{split}
\end{equation}

\begin{equation}\label{eq:potVol}
\begin{split}
pot(T^{'}_i)=2 pot(T_i) \Rightarrow \\ \Rightarrow V^{'}_{dB}=10log_{10} 2 \approx 3 dB
\end{split}
\end{equation}

and the amplitude gain for a sequence whose volume has been doubled ($10dB$):

\begin{equation}\label{eq:dobraVol}
\begin{split}
10log_{10}\frac{pot(T^{'}_i)}{pot(T_i)} = 10 \quad \Rightarrow \\ \Rightarrow \quad \sum_{i=0}^{\lfloor \Delta.f_a \rfloor -1}t^{'2}_i=10\sum_{i=0}^{\Lambda-1}t_i^2=\sum_{i=0}^{\Lambda-1}(\sqrt{10}.t_i)^2 \\
\therefore \quad t^{'}_i=\sqrt{10}t_i \quad \Rightarrow \quad t^{'}_i \approx 3,16t_i
\end{split}
\end{equation}

Thus, it is necessary a little bit more than to triplicate the amplitude for a doubled volume. This values are guides for the increases and decreases on the absolute values on the sample sequences with musical purposes. The direct conversion from decibels to amplitude gain (or attenuation) is:

\begin{equation}\label{ampDec}
A = 10^{\frac{V_{dB}}{20}}
\end{equation}

Where $A$ is the mutiplicative factor that relates the amplitudes before and after the amplification.

\subsection{Pitch}
Recapitulating, the musical particle (note) is a sequence $T_i$ which duration and volume corresponds to the size of sequence and the amplitude of its samples. The pitch is specified by the fundamental frequency $f_0$ whose cycle has duration $\delta_{f_0}=1/f_0$. This duration, multiplied by the sampling frequency $f_s$ results on the number of samples of the cycle: $\lambda_{f_0}=f_a . \delta_{f_0} =f_a/f_0$.

For didatic reasons, be $f_0$ such that it divides $f_s$ and $\lambda_{f_0}$ results an integer. If $T_i^f$ is a sonic sequence of fundamental frequency $f$, then:

\begin{equation}\label{periodicidade}
     T^f_i=\left\{ t_i^f \right\}=\left\{ t^f_{i+\lambda_{f}}  \right\}= \left\{ t^f_{i+\frac{f_a}{f}} \right\}
\end{equation}

In the next secion, frequencies $f$ that does not divide $f_s$ will be considered. This restriction does not imply in lost of generality of this section's content.

\subsection{Timbre}
While the period of the wave corresponds to a fundamental frequency, the trajectory of the wave inside the period - called the waveform - defines a harmonic spectrum and, thus, a timbre\footnote{The timbre is a subjective and complex characteristic. Physicaly, the timbre is multidimentionsl and given by the temporal dinamics behavior of enegy in spectral components that are harmonic or noisy. Beyond that, the word timbre is used to designate different things: one same note has different timbres, a same instrument has different timbres, two intruments of the same family possesses the same timbre that blends them in the same family and different timbres as they are different instruments. It is worth to mention that not all that is associated to timbre is manifested in spectral traces, even cultural or circunstancial aspects alter our perception of timbre}. Musically, it matters that sonic spectra with minimum differences result in timbres with crucial expressive differences and that, hence, diffetent timbres can be produced by using different spectra\cite{Roederer}.

The simplest case (and most important, as the following texts shows) in that of the spectrum that consists only of the fundamental frequency $f$ itself. This is the sinusoid, frequency in pure oscilatory movement called 'simple harmonic movement'. Be $S_i^f$ a sequence whose samples $s_i^f$ describes a sibusoid of frequency $f$:

\begin{equation}\label{senoide}
     S^f_i=\{ s^f_i \}=\Bigl\{ \sin\bigl(2\pi \frac{i}{\lambda_f} \bigr)  \Bigr\} = \Bigl\{ \sin\bigl(2\pi f \frac{i}{f_a}\bigr)  \Bigr\} 
\end{equation}

Where $\lambda_f=\frac{f_a}{f}=\frac{\delta_f}{\lambda_a}$  is the number of samples in the period.

In a similar fashion, other waveforms are used in music because of its spectral qualities and simplicity. While the sinusoid is an isolated point in the spectrum, these waves present a succession of harmonic components. These waveforms are specified in equations~\ref{sinusoid},~\ref{sawTooth},~\ref{triangular} and~\ref{square} are in figure~\ref{fig:formasDeOnda}.
These artificial waveforms are traditionally used in music for synthesis and oscilatory control of variables, they also present diverse used outside music\cite{Openheim}.

The sawtooth present all components of the harmonic series, with decreasing energy of $-6dB/octave$. The sequence of temporal samples can be described like this:

\begin{equation}\label{sawTooth}
     D^f_i=\left\{ d^f_i \right\}=\left\{ 2\frac{i\,\%\lambda_f}{\lambda_f} -1 \right\}
\end{equation}

The triangular waveform present only odd harminics falling with $-12dB/octave$:
\begin{equation}\label{triangular}
     T^f_i=\left\{ t^f_i \right\}=\left\{1- \left| 2 - 4\frac{i\,\%\lambda_f}{\lambda_f} \right| \right\}
\end{equation}

The square wave presets only odd harmonics falling at $-6dB/octave$:

\begin{equation}\label{quadrada}
     Q^f_i=\left\{ q^f_i \right\}= \left\{
         \begin{array}{l l}
              1 & \quad \text{for } \; \; (i\,\%\lambda_f)   <  \lambda_f /2  \\
              -1 & \quad \text{otherwise}\\
         \end{array} \right.
\end{equation}

The sawtooth is a common starting point for a subtractive synthesis, beacuse it has both odd and even harmonics with high energy. For musical intention, these waveforms are excessively rich in sharp harmonics and atenuant filtering on treble and midle parts of the spectrum is useful for reaching a more natural and pleasant sound. 
The relativelly attenuated harmonics of the triangle wave makes it the more functional - among the listed ones - to be used in the synthesis of musical notes without any treatment.

The square wave can be used in a subtractive synthesis the aims to mimic a clarinet. This instrument has only the odd harmonic components and the square wave is convenient with its abundant energy in high frequencies.

\begin{figure*}
    \centering
        \includegraphics[width=\textwidth]{waveForms}
    \caption{Basic musical waveforms. The synthetic  waveforms are in (a) and the real waveforms are in (b).}
        \label{fig:formasDeOnda}
\end{figure*}

Figure~\ref{fig:formasDeOnda} presents the waveforms described in equations  ~\ref{senoide}, ~\ref{denteDeSerra}, ~\ref{triangular} and ~\ref{quadrada} for $\lambda_f=100$ (period of $100$ samples). If $f_s=44.1kHz$, as PCM standard in Compact Disks, the wave has fundamental frequency $f=\frac{f_a}{\lambda_f}=\frac{44100}{100} = 441 \; Herz$, an A4, just above the central "C", whatever the waveform is.

The spectrum of each basic waveform is in figure~\ref{fig:espectroDeondas}. The isolated and exaclty harmonic components of the spectrum is a consequence of the fixed period usage. The sinusoid consists of a one and only node in spectrum, pure frequency. The sawtooh is the only with a complete harmonic series (odd and even components). Triangular and square waves has the same components (odd harmonics), decaing at $-12dB/octave$ and $-6dB/octave$ respectivelly.

\begin{figure*}
    \centering
        \includegraphics[width=\textwidth]{waveSpectrum}
    \caption{Spectrum of basic artificial musical waveforms.}
        \label{fig:espectroDeOndas}
\end{figure*}

The harmonic spectrum is formed by frequencies multiple from the fundamental frequency $f_n=(n+1)f_0$. As the human linear perception of pitch follows the a geometric progression of frequencies, the spectrum has notes different from the fundamental frequency. Additionally, the number of harmonics will be limited to the maximum frequency $f_s/2$ (by Nyquist's theorem).

Musically crucial here is to internalize that energy in an component of frequency $f_n$ means an oscilation in the constitution of the sound, purely harmonic and in that frequency $f_n$. This energy, specificaly concentrated on the $f_n$ is separated by the ear to enter a cognitive level of processing (this separation is done in many species is mechanisms similar to the human cochlea)~\cite{Roederer}.
The sinusoidal components are usually the main responsibles for the quality we call timbre. If they are not presented in harmonic proportions (small number relations), the sound is perceived as noisy or dissonant and not as a sonority with a univocally stablished fundamental frequency. Furthermore, the notion of absolute pitch in a complex sonority is based relies on the similarity of the spectrum to the harmonic series~\cite{Roederer}.

In the case of a fixed length period and waveform, the spectrum is perfectly harmonic and static. Each waveform is compound of specific proportions of harmonic components and the greater the curvature of a the part, the greater the contribution of the part to the energy on the high harmonics. That can be seen on real sounds. The wave rotulated as ``sampled real sound'' in figure~\ref{fig:formasDeOnda} is a period of $\Lambda_f=114$ samples, extracted from a relatively well behaved real sound. The oboe wave was sampled of a A4 also in $44.1kHz$. The chose period for sampling was a relatively short, with $98$ samples, corresonds to the frequency $\frac{44100}{98}=450Hz$. It can be noticed, from the curvatures, the oboe's rich spectrum in high frequencies and the lower spectrum of the real sound.

The sequence 
$ R_i=\{ r_i \}_0^{\lambda_f-1}$ of samples in the real sound of figure~\ref{fig:formasDeOnda} can be taken as a basis for a sound $T_i^f$ din the following way: 

\begin{equation}\label{sampleandoFormaDeOnda}
     T^f_i=\{ t_i^f \}=\Bigl\{ r_{(i\,\%\lambda_{f})} \Bigr\}
\end{equation}

The resulting sound has a the momentary spectrum of the original sound. Because it is repeated in an identical form, the spectrum is perfectly harmonic, withoud noise and variations tipical of the natural phenomenon.  This can be seen in figure~\ref{fig:espectroOboe}, that show the spectrum of the original oboe note and a note with same duration and whose samples consists of the repetition of cycle of figure~\ref{fig:formasDeOnda}. The natural spectrum exibit variations in the frequencies of the harmonics, its intensities and some noise. The note made from the sampled period has a perfecly harmonic spectrum

\begin{figure*}
    \centering
        \includegraphics[width=\textwidth]{oboeNaturalSampledSpectrum}
    \caption{Spectrum of the sonic waves of a natural oboe note and one made from a sampled period. The natural sound has fluctuations in the harmonics and in noise, while the sampled period note has a perfectly harmonic spectrum.}
        \label{fig:espectroOboe}
\end{figure*}

%%%%%%%%%%%%%%%%%%%%%%%%%%%%%%%%%%%%%%%%%%%%%%%%%%%%%%

\subsection{Spectrum at sampled sound}
The presence and behaviour of these sinosoidal components in the discretized sound have particularities. Having a signal $T_i$ and its Fourier decomposition $\mathcal{F}\langle T_i\rangle=C_i=\{c_i\}_0^{\Lambda-1}$, the recomposition is the sum of frequential components as time samples\footnote{It is important to note that the factor $\frac{1}{\Lambda}$ could be distributed among the Fourier transform and its reconstruction, as prefered.}:

\begin{equation}\label{recomposicaoFourier}
t_i = \frac{1}{\Lambda}\sum_{k=0}^{\Lambda-1}c_ke^{j \frac{2\pi k}{\Lambda} i } = \frac{1}{\Lambda}\sum_{k=0}^{\Lambda-1}(a_k+ j . b_k)\left[cos(w_k i) +j . sen(w_k i)\right]
\end{equation}

Where $c_k = a_k + j . b_k$ defines the amplitude and phase of each frequency: $w_k=\frac{2\pi}{\Lambda}k$ in radians or $f_k=w_k\frac{f_a}{2\pi}=\frac{f_a}{\Lambda}k$ in Hertz, paying attention to the respective limits in $\pi$ and in $\frac{f_a}{2}$ given by the Nyquist Theorem. 

For a sound signal, samples $t_i$ are real and are given by the real part of equation~\ref{recomposicaoFourier}:

\begin{equation}\label{moduloEfase}
\begin{split}
t_i& = \frac{1}{\Lambda}\sum_{k=0}^{\Lambda-1}\left[a_k cos(w_k i) -b_k sen(w_k i)\right] \\
   & = \frac{1}{\Lambda}\sum_{k=0}^{\Lambda-1}\sqrt{a_k^2 + b_k^2} \; cos\left[w_k i - tg^{-1}\left(\frac{b_k}{a_k}\right)\right]
\end{split}
\end{equation}

Equation xxx shows how the imaginary term $c_k$ adds a phase to the real sinusoid, e.g.\ the imaginary terms $b_k$ from the Fourier spectral decomposition make possible the phase sweep $\left[-\frac{\pi}{2},+\frac{\pi}{2}\right]$ given by $tg^{-1}\left(\frac{b_k}{a_k}\right)$ which has this image. The signal from $a_k$ specify the right and left side of the trigonometric circle, complementing the phase sweeping: $\left[-\frac{\pi}{2},+\frac{\pi}{2}\right] \cup \left[\frac{\pi}{2},\frac{3\pi}{2}\right]\equiv [2\pi]$.


 \begin{figure}[h!]
     \centering
         \includegraphics[width=\textwidth]{figuras/amostras2c__}
     \caption{Oscillation for 2 samples (maximum frequency for any $f_a$). The first coefficient reflects a dettachment (\emph{offset} or \emph{bias}) and the second coefficient specifies the oscillation amplitude.}
         \label{fig:amostras2}
 \end{figure}

Figure~\ref{fig:amostras2} shows two samples and its spectral components. In this case, the Fourier decomposition has one unique pair of coefficients $\{c_k=a_k-j.b_k\}_0^{\Lambda-1=1}$ relatives to frequencies $\{f_k\}_0^1=\left\{w_k\frac{f_a}{2\pi}\right\}_0^1=\left\{k\frac{f_a}{\Lambda=2}\right\}_0^1=\left\{0,\frac{f_a}{2}=f_{\text{máx}}\right\}$
with energies $e_k=\frac{(c_k)^2}{\Lambda=2}$. The role of amplitudes $a_k$ is clearly observed with $\frac{a_0}{2}$, the fixed dettachment\footnote{Also called \emph{bias} or \emph{offset}.} and $\frac{a_1}{2}$, oscillation own amplitude given by the relation $f_k=k \frac{f_a}{\Lambda=2}$.
This case have special relevance. The minimum needed to represent an oscillation are 2 samples and it yelds the Nyquist frequency $f_{\text{máx}}=\frac{f_a}{2}$. That is the maximum frequency in a sound sampled with $f_a$ samples per second\footnote{Any sampled signal has this property, not only the digitalized sound.}.

All fixed sequences $T_i$ of only $3$ samples also have just $1$ frequency because their first harmonic has $1,5$ samples and exceeds the bottom limit of 2 samples, e.g.\ the frequency of the harmonic would exceed the Nyquist frequency:  $\; \frac{2. f_a}{3} > \frac{f_a}{2} $. 
The coefficients $\{c_k\}_0^{\Lambda-1=2}$ are present in 3 frequency components. One is relative to zero frequency ($c_0$), the other two ($c_1$ and $c_2$) have the same role for the reconstruction of sinusoid with $f=f_a/3$.

 \begin{figure}[h!]
     \centering
         \includegraphics[width=\textwidth]{figuras/amostras3b}
     \caption{Three fixed samples presents only one non-null frequency. $c_1=c_2^*$ and $w_1 \equiv w_2$.}
         \label{fig:amostras3}
 \end{figure}

$\Lambda$ real samples $t_i$ result in $\Lambda$ complex coefficients $c_k=a_k+j.b_k$. The coefficients $c_k$ are equivalent two by two, corresponding to the same frequencies and with same relevance\footnote{Equal real part and imaginary with inversed order: $a_{k1}=a_{k2}$ and $b_{k1}=-b_{k2}$. As consequence the modules are equal and phases have inversed order.}. Remembering that $f_k = k\frac{f_a}{\Lambda}, \; k \in \left\{0, ..., \left \lfloor \frac{\Lambda}{2} \right \rfloor \right\} $. When $k > \frac{\Lambda}{2}$, the frequency $f_k$ is mirrored by $\frac{f_a}{2}$ in this way: $f_k=\frac{f_a}{2} - (f_k-\frac{f_a}{2})=f_a-f_k=f_a - k\frac{f_a}{\Lambda}=(\Lambda-k)\frac{f_a}{\Lambda} \;\;\;\; \Rightarrow \;\;\;\; f_k\equiv f_{\Lambda-k} \; ,\;\; \forall \;\; k<\Lambda$. 

The same could be observed in $w_k=f_k.\frac{2\pi}{f_a}$ and remembering the periodicity $2\pi$ it results in $w_k=-w_{\Lambda-k}$. Given the cosine as as a even function and the inverse tangent as odd function, the components in $w_k$ and $w_{\Lambda-k}$ sums up in the equationi that reconstructs the real samples willing equation~\ref{recomposicaoFourier}.


In other words, in a decomposition of $\Lambda$ samples, the $\Lambda$ given frequential components $\{c_i\}_0^{\Lambda-1}$ are equivalents in pairs.
Exception for $f_0$ and when having an even $\Lambda$, from $f_{\Lambda/2}=f_{\text{máx}}=\frac{f_a}{2}$ both components are isolated, e.g.\ there is any other component in frequency $f_0$ or $f_{\Lambda/2}$ (if even $\Lambda$) regarding itself. 
It is true because $f_{\Lambda/2}=f_{(\Lambda-\Lambda/2) = \Lambda/2}$ and $f_0=f_{(\Lambda-0)=\Lambda}=f_0$.
Furthermore, these two frequencies (zero and maximum frequency) are not represented having phase variation, being strictly real. In this way, it is possible to conclude the number $\tau$ of equivalent coefficient pairs are:

\begin{equation}\label{coefsPareados}
\tau = \frac{\Lambda - \Lambda \% 2}{2} +\Lambda \% 2 -1
\end{equation}

and became clearly visible the equivalences ~\ref{equivalenciasFreqs}, ~\ref{equivalenciasModulos} and ~\ref{equivalenciasFases}:

\begin{equation}\label{equivalenciasFreqs}
f_{k}\equiv f_{\Lambda-k}\;, \;\; w_{k}\equiv-w_{\Lambda-k}\;\;\;, \quad \;\; \forall \quad 1 \leq k \leq \tau  
\end{equation}

\begin{figure}[h!]
    \centering
        \includegraphics[width=\textwidth]{figuras/amostras4__}
    \caption{Frequential components for 4 samples.}
        \label{fig:amostras4}
\end{figure}

Having $a_k = a_{\Lambda -k}\;\;$ and $\;\;b_k = - b_{\Lambda -k}$:

\begin{equation}\label{equivalenciasModulos}
\sqrt{a_k^2 + b_k^2} = \sqrt{a_{\Lambda - k}^2 + b_{\Lambda -k}^2} \;\;, \quad \;\; \forall \quad 1 \leq k \leq \tau  \\
\end{equation}

\begin{equation}\label{equivalenciasFases}
tg^{-1}\left(\frac{b_k}{a_k}\right)=-tg^{-1}\left(\frac{b_{\Lambda -k}}{a_{\Lambda - k}}\right)\;\;,\quad \;\; \forall \quad 1 \leq k \leq \tau
\end{equation}

Having $k \in \mathbb{N}$.

The observation for the real signal reconstruction equation~\ref{moduloEfase} together with modules and phases equivalences~\ref{equivalenciasModulos} and~\ref{equivalenciasFases}, the number of paired coefficients~\ref{coefsPareados} and equivalence of paired frequencies~\ref{equivalenciasFreqs} expose a general case for components combination in each sample $t_i$:

\begin{equation}\label{eq:reconsCompleta}
t_i = \frac{a_0}{\Lambda} + \frac{2}{\Lambda}\sum_{k=1}^{\tau}\sqrt{a_k^2 + b_k^2} \; cos\left[w_k i - tg^{-1}\left(\frac{b_k}{a_k}\right)\right]+ \frac{ a_{\Lambda/2}}{\Lambda}.(1-\Lambda\% 2)
\end{equation}

\begin{figure}[h!]
    \centering
        \includegraphics[width=\textwidth]{figuras/amostras4formas__}
    \caption{Basic wave forms for 4 samples.}
        \label{fig:formas4}
\end{figure}

Therefore, as in Figure~\ref{fig:amostras3}, the Fourier transform of 3 samples have 2 frequential coefficients with same amount of energy in the same frequency.

With 4 samples it is possible to represent 1 or 2 frequencies with any proportions. Figure~\ref{fig:amostras4} shows wave forms for 4 samples and its two components. The individual contributions sums together yielding the original wave form and a brief inspection reveals the major curvatures resulting from the higher frequency, while a fixed dettachment of component summation results from the zero frequency component.

\begin{figure}[h!]
    \centering
        \includegraphics[width=\textwidth]{figuras/amostras6}
    \caption{Frequential components for 6 samples: 3 sinosoids sums with \emph{bias}.}
        \label{fig:amostras6}
\end{figure}

Figure~\ref{fig:formas4} shows harmonics for 4 samples in basic wave forms of equations ~\ref{senoide}, ~\ref{denteDeSerra}, ~\ref{triangular} and ~\ref{quadrada}. All together results in only 1 sinusoid, with exception to sawtooth wave which have even harmonics.

Figure~\ref{fig:amostras6} presents a sinusoidal decomposition for the 6 samples case and figure~\ref{fig:formas6} decompose the basic wave forms.
In this case all wave forms are different in spectrum: square and triangular ones have the same components but with different proportions, while the sawtooth have an extra component.

\begin{figure}[h!]
    \centering
        \includegraphics[width=\textwidth]{figuras/amostras6formas___}
    \caption{Basic wave forms for 6 samples: triangular and square wave forms have odd harmonics, with different proportions and phases; the sawtooth wave form also has even harmonics.}
        \label{fig:formas6}
\end{figure}

\subsection{The basic note}\label{notaBasica}

%% Seja $f$ tal que $f$ divida $f_a$\footnote{Como apontado anteriormente, esta limitação facilita a exposição sem perda de generalidade.
%% A limitação será superada no início da próxima seção.}.
%% Uma sequência $T_i$ de amostras sonoras separadas por $\delta_a=1/f_a$ descreve uma nota musical de frequência $f$ Hertz e duração $\Delta$ segundos se, e somente se, possuir a periodicidade $\lambda_f=f_a/f$
%%  e tamanho $\Lambda=\lfloor f_a . \Delta \rfloor $:

%% \begin{equation}\label{eq:notaBasica}
%% T_i^{f,\; \Delta}=\{t_{i \, \% \lambda_f} \}_0^{\Lambda-1}= \left \{t^f_{i \; \% \left( \frac{f_a}{f} \right) } \right \}_0^{\Lambda-1}
%% \end{equation}

%% A nota por si só não especifica um timbre. Mesmo assim, faz-se necessária a escolha de uma forma de onda para que as amostras $t_i$ tenham um valor estabelecido individualmente. Um único período dentre as ondas básicas pode ser utilizado para a especificação da nota da seguinte forma:

%% $\lambda_f=\frac{f_a}{f}$ é o número de amostras do período. Seja $L_i^{f,\, \delta_f} $
%% a sequência que descreve um período da onda $L_i^f \in \{S_i^f,Q_i^f,T_i^f,D_i^f,R_i^f \}$ de duração 
%% $\delta_f=1/f$, dadas pelas equações ~\ref{senoide}, ~\ref{denteDeSerra}, ~\ref{triangular} e ~\ref{quadrada} e onde $R_i^f$ é
%% uma onda real amostrada:

%% \begin{equation}\label{periodoUnico}
%% L_i^{f , \delta_f } = \left\{ l_i^f \right\}_0^{\delta_f . f_a -1}=\left\{ l_i^f \right\}_0^{\lambda_f-1}
%% \end{equation}

%% Então a sequência $T_i$ consistirá em uma nota de duração $\Delta$ e frequência $f$ se:

%% \begin{equation}\label{eq:notaBasicaTimbre}
%% T_i^{f,\; \Delta}=\left\{t_i^f\right\}_0^{\lfloor f_a . \Delta \rfloor -1}=\left \{ l^f_{i\,\%\left(\frac{f_a}{f}\right)} \right \}_0^{\Lambda-1}
%% \end{equation}

%% \subsection{Localização espacial e espacialização}\label{subsec:spac}
%% Embora não seja uma das quatro qualidades básicas tradicionais de uma nota musical, esta possui sempre uma localização espacial, que é a posição da fonte que a emitiu, no espaço físico tridimensional ordinário. Além disso, há um ambiente que reverbera a nota emitida, assunto ao qual a 'espacialização' é dedicada. Ambas, a espacialização e a localização espacial, são bastante valorizadas
%%  por audiófilos e pela indústria fonográfica.\cite{floEsp}

%%  \subsubsection{Localização espacial}
%% Acredita-se que a percepção da localização espacial do som se dê em nosso sistema nervoso através destas
%% três informações: o atraso de chegada do som entre um ouvido e o outro, a diferença de intensidade do som direto em cada ouvido e a 
%% filtragem realizada pelo corpo, incluindo tórax, cabeça e orelhas.\cite{Roederer, hrtf, Heeger}


%% \begin{figure}[h!]
%%     \centering
%%         \includegraphics[width=.5\textwidth]{figuras/espacializacao___}
%%     \caption{Detecção de localização espacial de fonte sonora: esquema utilizado para cálculo da diferença de tempo interaural (DTI) e da diferença de intensidade interaural (DII).}
%%         \label{fig:spac}
%% \end{figure}



%% 	Se consideradas somente as incidências diretas em cada ouvido, as equações são simples. Dada a separação $\zeta$ entre os ouvidos\footnote{Constata-se que $\zeta \approx 21,5cm$ para um humano adulto.},
%% um objeto localizado em $(x,y)$ conforme a figura~\ref{fig:spac}
%% está distante de cada ouvido:

%% \begin{equation}\label{eq:distOuvidos}
%% \begin{split}
%% d & =\sqrt{\left (x-\frac{\zeta}{2} \right )^2+y^2} \\
%% d' & =\sqrt{\left (x+\frac{\zeta}{2} \right )^2 + y^2}
%% \end{split}
%% \end{equation}


%% e cálculos imediatos resultam na Diferença de Tempo Interaural:

%% \begin{equation}\label{eq:dti}
%% DTI=\frac{d'-d}{v_{som\;no\;ar}\approx 343.2 }\quad \text{segundos}
%% \end{equation}

%% e na Diferença de Intensidade Interaural:
%% \begin{equation}\label{eq:dii}
%% DII=20\log_{10}\left (\frac{d}{d'}\right) \quad decibels
%% \end{equation}

%% Convertendo para amplitude, obtém-se $DII_a=\frac{d}{d'}$. A $DII_a$ pode
%% ser utilizada como constante multiplicativa do canal direito de um sinal sonoro estéreo: $\{t_i'\}_0^{\Lambda -1}=\{DII_a . t_i\}_0^{\Lambda -1}$. Pode-se utilizar a DII junto à DTI como adiantamento no tempo do canal direito com relação ao esquerdo, vínculo crucial para a localização em sons graves e em sonoridades percussivas.\cite{Heeger}  
%% Considerando $\Lambda_{DTI}=\lfloor DTI . f_a \rfloor$:

%% \begin{equation}\label{eq:locImpl}
%% \begin{split}
%% \Lambda_{DTI} & = \left \lfloor \frac{d'-d}{343,2}  f_a \right \rfloor \\
%% DII_a & = \frac{d}{d'} \\
%% \left\{t_{(i+\Lambda_{DTI})}'\right\}_{\Lambda_{DTI}}^{\Lambda+\Lambda_{DTI}-1} & =\left\{DII_a . t_i\right\}_0^{\Lambda-1} \\
%% \left\{t_i'\right\}_0^{\Lambda_{DTI}-1} & = 0
%% \end{split}
%% \end{equation}

%% Com $t_i$ o canal direito e $t_i'$ o canal esquerdo. Caso $\Lambda_{DTI} < 0 $, basta trocar $t_i$ por $t_i'$  e utilizar $\Lambda_{DTI}'= | \Lambda_{DTI} | $.

%% Embora consideravelmente simples até aqui, a localização espacial depende drasticamente de outras pistas. Pela
%% DTI e DII especifica-se somente o ângulo horizontal (azimutal) $\theta$ dado por:

%% \begin{equation}\label{eq:angulo}
%% \theta=\tan^{-1}\left ( \frac{y}{ x }  \right )
%% \end{equation}

%% com $x,y$ tais como representados na figura~\ref{fig:spac}. Mesmo assim, há dificuldades quando $\theta$ incide sobre o chamado "cone de confusão" em que um mesmo par de especificações DTI, DII resultam de vários dos pontos 
%% do cone. Nestes pontos, a inferência do ângulo azimutal depende especialmente da filtragem atenuante nos agudos, pois a cabeça interfere um tanto mais nas ondas mecânicas agudas do que nas graves.\cite{Heeger,hrtf}  Também pertinente à audição de fonte lateral, quando o som é grave o suficiente, há uma difração e a onda chega ao ouvido $\approx 0,7ms$ depois.\cite{floEsp}

%% A figura~\ref{fig:spac} mostra também esta sombra acústica do crânio, importante para a percepção do ângulo azimutal da fonte no cone de confusão. O cone em si não foi disposto na figura pois não é exatamente um cone e suas dimensões precisas não foram encontradas na literatura visitada e não são facilmente concebíveis, dadas as filtragens e a difração dependente do espectro do som em si. De toda forma, o cone de confusão pode ser entendido como um cone com o ápice no meio da cabeça e saindo por cada uma das orelhas.\cite{hrtf}

%% Já a localização completa, incluindo distância e elevação da fonte sonora, é dada pela função de transferência de cabeça (HRTF - do inglês \emph{Head Related Transfer Function}).\cite{hrtf} Existem bases abertas e conhecidas de HRTF como a CIPIC e pode-se aplicar estas funções de transferência em um som por convolução (veja equação~\ref{eq:conv}).\cite{CIPIC} O corpo do indivíduo altera bastante as filtragens realizadas e existem técnicas para gerar HRTFs que sejam - como proposta - utilizáveis de forma universal.\cite{lazaSPA} 

%% \subsubsection{Espacialização}
%% Já a espacialização é o resultado das reflexões e absorções do som nas superfícies do recinto/paisagem no qual a nota foi emitida. O som se propaga no ar a $\approx 343,2m/s$, e pode ser emitido da fonte com qualquer padrão de direcionalidade. Quando uma frente sonora encontra uma superfície, há uma reflexão. Nesta reflexão ocorrem tanto 1) a inversão da componente da velocidade de propagação que é perpendicular à superfície, quanto 2) a absorção de energia, especialmente nos agudos. As ondas se propagam até atingirem níveis inaudíveis. Quando alguma frente de onda atingir o ouvido, pode ser descrita com o momento de chegada ao ouvido e os filtros de absorção de cada superfície que atingiu. Pode-se simular reverberações não possíveis em sistemas reais. Para experimentações, pode-se usar reflexões assimétricas com relação ao eixo perpendicular à superfície, ou ainda ganhos em determinadas bandas de frequência (tidos como 'ressonâncias'), ambas as características não são encontradas em sistemas reais.

%% Existem algumas modelagens de reverberação menos atreladas ao cálculo de cada reflexão, exploram informações valiosas do ponto de vista auditivo. De fato, a reverberação pode ser modelada com um conjunto de 2 características temporais e no espectro:
%% \begin{itemize}
%%     \item Primeiro período: as 'primeiras reflexões' são mais intensas e esparsas.
%%     \item Segundo período: a 'reverberação tardia' é praticamente uma sucessão densa de atrasos indistintos com um decaimento exponencial e ocorrências estatísticas.
%%     \item Primeira banda: o grave possui algumas frequências de ressonância relativamente espaçadas.
%%     \item Segunda banda: o médio e agudo possuem um decaimento progressivo e suave com flutuações estatísticas.
%% \end{itemize}

%% Smith III aponta que boas salas de concerto possuem um tempo total de reverberação de aproximadamente $1,9$ segundos. Aponta também o período das primeiras reflexões de $0,1$ segundos. Estas quantidades sugerem que, nas condições contempladas, há frentes de onda perceptíveis que se propagam até $652,08$ metros ($83,79k$ amostras em $f_a=44,1kHz$) antes de atingirem o ouvido. Além disso, as reflexões do som formam, após a propagação por $34,32$ metros ($4,41k$ amostras em $f_a=44,1kHz$ ), um emaranhado cujas incidências são pouco distintas na audição. Estas primeiras reflexões são particularmente importantes para a sensação de espaço. A primeira incidência é o som direto, descrito por DTI e DII das equações~\ref{eq:dti} e~\ref{eq:dii}. Admitindo que cada uma das primeiras reflexões, antes de chegar ao ouvido, se propagará, ao menos, $3-30m$ dependentes das dimensões da sala, a separação entre as primeiras reflexões é de, ao menos, $8-90$ milissegundos ($\approx 350-4000$ amostras em $f_a=44.1kHz$). Verifica-se experimentalmente que o número de reflexões aumenta em proporção quadrática  $ \approx k.n^2$. Apontamentos do uso de convoluções e filtragens para facilitar estas implementações estão na subseção~\ref{subsec:mus2}, especialmente nos parágrafos sobre reverberação.

%% \subsection{Usos musicais}\label{subsec:basMus}


%% A partir da nota básica, cabe realizar estruturas musicais com
%% sequências destas partículas. A soma dos elementos de mesmo índice de $N$ sequências $T_{k,i}=\{t_{k,i}\}_{k=0}^{N-1}$ de mesmo tamanho $\Lambda$ resulta em seus conteúdos espectrais sobrepostos em um processo de mixagem sonora:

%% \begin{equation}\label{eq:mixagem}
%% \{t_i\}_0^{\Lambda-1}=\left \{ \sum_{k=0}^{N-1}t_{k,i} \right \}_0^{\Lambda-1}
%% \end{equation}

%% \begin{figure}[h!]
%%     {\centering
%%         \includegraphics[width=\textwidth]{figuras/mixagem}}
%%     \caption{Mixagem de três sequências sonoras. As amplitudes são sobrepostas diretamente.}

%%         \label{fig:mixagem}
%% \end{figure}


%% A figura~\ref{fig:mixagem} ilustra este processo de superposição de ondas sonoras discretizadas. A figura dispõe 100 amostras, de onde pode-se concluir que, se $f_a=44.1kHz$, as frequências da dente de serra, da onda quadrada e da senoide são,
%% respectivamente, $\frac{f_a}{100/2}=882Hz$, $\frac{f_a}{100/4}=1764Hz$ e $\frac{f_a}{100/5}=2205Hz$. A duração do trecho é bastante curto $\frac{f_a=44.1kHz}{100} \approx 2 \text{ milissegundos}$. Basta completar com zeros para somar sequências de tamanhos diferentes. 

%% As notas mixadas são em grande parte separadas pelo ouvido por leis físicas de ressonância e pelo sistema nervoso.\cite{Roederer} O resultado da mixagem de notas musicais é a harmonia musical, cujos intervalos entre as frequências e os acordes de notas simultâneas regem aspectos subjetivos e abstratos da música e sua apreciação.\cite{Harmonia} 

%% As sequências podem também ser concatenadas no tempo. Caso as sequências $\{t_{k,i}\}_0^{\Lambda_k-1}$ de tamanhos $\Lambda_k$  representem $k$ notas musicais, sua concatenação em uma única sequência $T_i$ é em uma sequência musical simples ou melodia:

%% \begin{equation}\label{eq:concatenacao}
%% \{t_i\}_0^{\sum\Delta_k-1}=\{t_{l,i}\}_0^{\sum\Delta_k-1}, \;\; l\text{ menor inteiro } : \quad \Lambda_l > i -\sum_{j=0}^{l-1}\Lambda_j
%% \end{equation}

%% Este mecanismo é demonstrado de forma ilustrativa na figura~\ref{fig:concatenacao} com as mesmas sequências da figura ~\ref{fig:mixagem}.
%%  As sequências são curtas para as taxas de amostragem usuais, mas pode-se observar a concatenação de sequências sonoras. Além disso, cada nota tem a duração maior que $100ms$ se $f_a<1kHz$.

%% \begin{figure}[h!]
%% {    \centering
%%         \includegraphics[width=\textwidth]{figuras/concatenacao}}
%%     \caption{Concatenação de três sequências sonoras através da justaposição temporal de suas amostras.}
%%         \label{fig:concatenacao}
%% \end{figure}

%% A montagem musical \emph{reduced-fi} explora de forma isolada este uso de justaposição temporal das notas, resultando em uma peça homofônica. O princípio vertical está demonstrado nos \emph{quadros sonoros}, sons estáticos com espectros peculiares. Ambas as peças estão em código Python nos Apêndices~\ref{ap:quadros} e~\ref{ap:reduced} e estão disponíveis como parte da \emph{toolbox} \massa.\cite{MASSA}

%% Está descrita a nota musical digital básica e a seção seguinte desenvolve a evolução temporal de seus conteúdos, como nos \emph{glissandi} e nas envoltórias de volume. A filtragem de componentes espectrais e a geração dos ruídos completam a constituição da nota musical como unidade isolada e se desdobra na seção~\ref{notasMusica}, dedicada à estruturação destas notas em música através de métricas e trajetórias.




%% \afterpage{\blankpage}
%% \clearpage

%% \section{Variações na nota musical básica}\label{varInternas}

%% A nota musical digital básica foi definida na seção~\ref{sec:notaDisc} com os parâmetros:
%% duração, altura, intensidade (volume) e timbre. Esta é uma modelagem
%% útil e paradigmática, mas não esgota o que se entende por
%% uma nota musical.

%% Em primeiro lugar, as características da nota se modificam no decorrer
%% da própria nota.\cite{Chowning} Por exemplo, uma nota de piano
%% de 3 segundos tem a intensidade com início abrupto e decaimento progressivo,
%% além de variações do espectro, com harmônicos que
%% decaem antes dos outros e alguns que aparecem com o tempo.
%% Estas variações não são obrigatórias e sim orientações da
%% síntese sonora para usos musicais, pois é como os sons
%% se apresentam na natureza\footnote{A regra de ouro
%% aqui é: para que um som isolado desperte interesse
%% por si só, faça com que tenha variações internas.\cite{Roederer}}. 
%% Explorar todas as formas pelas quais estas variações ocorrem está fora
%% do escopo de qualquer trabalho dada a considerável sensibilidade do ouvido humano
%% e a complexidade da nossa cognição sonora. A seguir, serão apontados
%% recursos primários para estas variações das características na nota
%% básica.
%% Todas as relações descritas nesta seção estão implementadas em Python no Apêndice~\ref{sec:cod2}. As montagens musicais \emph{Transita para metro}, \emph{Vibra e treme}, \emph{Tremolos, vibratos e a frequência}, \emph{Trenzinho de caipiras impulsivos}, \emph{Ruidosa faixa}, \emph{Bela rugosi}, \emph{Chorus infantil}, \emph{ADa e SaRa} estão nos Apêndices~\ref{ap:transita}, \ref{ap:vibra}, \ref{ap:tremolos}, \ref{ap:trenzinho}, \ref{ap:ruidosa}, \ref{ap:bela}, \ref{ap:chorus} e \ref{ap:ada}. Estes códigos são parte da caixa de ferramentas \massa, disponível online.\cite{MASSA}



%% \subsection{Tabela de busca}\label{subsec:lookup}

%% Mais conhecida pelo termo em inglês, a \emph{Lookup Table} (ou simplesmente
%% LUT), é uma estrutura de dados para
%% consultas indexadas usada
%% frequentemente para reduzir a complexidade computacional
%% e por
%% permitir o uso de funções sem possibilidade de cálculo direto, como
%% amostras recolhidas da natureza. 
%% Na música seu uso transcende estes
%% primeiros, facilitando as operações e permitindo que um único
%% período de onda possa ser usado para sintetizar sons em toda a banda
%% de frequências audíveis, qualquer que seja a forma de onda amostrada.


%% \begin{figure}[h!]
%%     \centering
%%         \includegraphics[width=\textwidth]{figuras/lut}
%%     \caption{Procedimento de busca em tabela (conhecido como \emph{Lookup Table}) para síntese de sons em frequências diferentes a partir de uma única forma de onda em alta resolução.}
%%         \label{fig:lut}
%% \end{figure}

%% Seja $\widetilde{\Lambda}$ o tamanho 
%% do período e $\widetilde{L_i} = \left\{\, \widetilde{l}_i \,\right\}_0^{\widetilde{\Lambda} -1}$ os elementos $\widetilde{l_i}$ de um
%% período de onda qualquer (veja equação ~\ref{periodoUnico}). Uma sequência
%% $T_i^{f,\,\Delta}$ com amostras de um som de frequência $f$ e duração $\Delta$
%% pode ser obtida a partir de $\widetilde{L_i}$ da seguinte forma:

%% \begin{equation}\label{eq:lut}
%% T_i^{f,\,\Delta}=\left\{t_i^f\right\}_0^{\lfloor \, f_a . \Delta \, \rfloor -1} = \left\{ \, \widetilde{l}_{\gamma_i \% \widetilde{\Lambda} }\, \right\}_{0}^{\Lambda-1}\; , \quad \text{onde} \;\; \gamma_i = \left \lfloor i . f \frac{ \widetilde{\Lambda}}{f_a} \right \rfloor  
%% \end{equation}

%% Ou seja, com os índices corretos ($\gamma_i\%\widetilde{\Lambda}$)
%% da LUT, pode-se sintetizar o som em qualquer frequência. 
%% A figura~\ref{fig:lut} ilustra
%% o cálculo de uma amostra de $\{t_i\}$ 
%% a partir de $\left\{\,\widetilde{l}_i\,\right\}$ para uma frequência
%% de $f=200Hz$, $\widetilde{\Lambda}=128$ e considerada a taxa de amostragem em $f_a=44.1kHz$.
%% Esta não é uma configuração praticável, como assinalado abaixo, mas possibilita uma
%% disposição gráfica do procedimento.


%% O cálculo do inteiro $\gamma_i$ introduz um ruído, 
%% e este diminui com o aumento de $\widetilde{\Lambda}$.
%% Para fins de síntese, em $f_a=44.1 kHz$
%%  o padrão é usar $\widetilde{\Lambda} = 1024$ amostras, pois já não gera ruído
%%  relevante no espectro audível. O método de arredondamento ou interpolação não é decisivo.\cite{Geiger}

%%  A expressão que define a variável $\gamma_i$ pode ser compreendida da
%%  seguinte forma: $i$ é acrescida de $f_a$ a cada $1$ segundo.
%%  Caso seja dividida pela frequência de amostragem, resulta $\frac{i}{f_a}$,
%% que é acrescida de $1$ a cada $1$ segundo. Multiplicada pelo comprimento do período, resulta $i \frac{\widetilde{\Lambda}}{f_a}$
%%  que varre o 
%%   período em $1$ segundo. Por fim,
%%  com a frequência $f$, resulta $i . f \frac{\widetilde{\Lambda}}{f_a}$
%%  que completa $f$ varreduras do período  $\widetilde{\Lambda}$ em $1$ segundo, i.e. a sequência
%%  resultante apresenta a frequência fundamental $f$.

%% Importantes considerações: $f$ é qualquer, só há limitantes nas frequências
%% graves quando o tamanho da tabela $\widetilde{\Lambda}$ não é suficientemente grande para a taxa de amostragem
%% $f_a$. O procedimento de busca em tabela
%% é computacionalmente bastante barato, substituindo cálculos por buscas simples (por isso geralmente
%% é entendido como um processo de otimização). Salvo quando assinalado,
%% no texto que usará este procedimento para todos os casos cabíveis pois
%% simplifica as rotinas e é computacionalmente coerente.

%% O uso de LUTs é bastante difundido nas implementações computacionais
%% voltadas para música e um uso clássico que explora com ênfase
%% as LUTs na síntese sonora musical, é a  chamada 
%% \emph{Wavetable Synthesis} que consiste em várias LUTs utilizadas em 
%% conjunto através da mixagem para gerar uma nota musical quasi-periódica.~\cite{Cook,Wavetable}.


%% \subsection{Variações incrementais de frequência e intensidade}\label{subsec:vars}

%% Segundo a lei de Weber e Fechner, a percepção humana tem uma relação logarítmica com
%% o estímulo que a causa.\cite{Weber-Fechner} Em outras palavras, um estímulo em progressão exponencial
%% é percebido como linear.
%% Por razões didáticas e dado o uso nas AM e FM (veja subseção~\ref{subsec:tvaf}), a variação linear será abordada primeiro.

%% Em uma nota de duração $\Delta = \frac{\Lambda}{f_a}$, a frequência $f=f_i$ varia de $f_0$ até $f_{\Lambda -1}$
%% linearmente. Pode-se escrever:

%% \begin{equation}\label{freqLinear}
%% F_i=\{f_i\}_0^{\Lambda-1}=\left\{f_0 + (f_{\Lambda-1}-f_0)\frac{i}{\Lambda-1} \right\}_0^{\Lambda-1}
%% \end{equation}

%% \begin{equation}\label{indiceLinear}
%% \Delta_{\gamma_i}=f_i\frac{\widetilde{\Lambda}}{f_a} \quad \Rightarrow \quad \gamma_i=\left \lfloor \sum_{j=0}^{i} f_j\frac{\widetilde{\Lambda}}{f_a} \right \rfloor   =\left \lfloor \sum_{j=0}^{i} \frac{\widetilde{\Lambda}}{f_a} \left [f_0 + (f_{\Lambda-1}-f_0)\frac{j}{\Lambda-1} \right ] \right \rfloor 
%% \end{equation}

%% \begin{equation}\label{serieAmostralLin}
%% \left\{t_i^{\;\overline{f_0,\, f_{\Lambda-1}}}\right\}_0^{\Lambda-1}=\left\{\,\widetilde{l}_{\gamma_i \% \widetilde{\Lambda}}\,\right\}_0^{\Lambda-1}
%% \end{equation}

%% Onde $\Delta_{\gamma_i}=f_i\frac{\widetilde{\Lambda}}{f_a}$ é o incremento da LUT entre duas amostras dada a frequência do som na primeira amostra.

%% Desta forma, pode-se calcular os elementos $t_i^{\;\overline{f_0,f_{\Lambda-1}}}$
%% com base no período $\left\{\widetilde{l}_i\right\}_0^{\Lambda-1}$.

%% As equações \ref{freqLinear}, \ref{indiceLinear} e \ref{serieAmostralLin} são relativas à progressão linear
%% da frequência. Como assinalado para o caso geral, também aqui
%% uma progressão de frequência
%% \emph{percebida} como linear segue uma progressão exponencial\footnote{Ou,
%% dito ainda de outra forma, uma progressão geométrica da frequência
%% é percebida como uma progressão aritmética de alturas.}.
%% Pode-se escrever que: $f_i=f_0 . 2^{\frac{i}{\Lambda-1} n_8}$ onde 
%% $n_8=\log_2\frac{f_{\Lambda-1}}{f_0}$ é o número de oitavas entre $f_0$ e $f_{\Lambda-1}$.
%% De forma que $f_i=f_0 . 2^{\frac{i}{\Lambda-1}\log_2\frac{f_{\Lambda-1}}{f_0}}=
%% f_0 . 2^{\log_2\left ( \frac{f_{\Lambda-1}}{f_0} \right )^{\frac{i}{\Lambda-1}}}=
%% f_0 \left ( \frac{f_{\Lambda-1}}{f_0} \right ) ^{\frac{i}{\Lambda -1}}$. Portanto,
%%  as equações de transições de frequência
%% lineares para o ouvido são:

%% \begin{equation}\label{freqExponencial}
%% F_i=\{f_i\}_0^{\Lambda-1}=\left\{f_0 \left ( \frac{f_{\Lambda-1}}{f_0} \right ) ^{\frac{i}{\Lambda -1}} \right\}_0^{\Lambda-1}
%% \end{equation}

%% \begin{equation}\label{indiceExponencial}
%% \Delta_{\gamma_i}=f_i\frac{\widetilde{\Lambda}}{f_a} \quad \Rightarrow \quad \gamma_i=\left \lfloor \sum_{j=0}^{i} f_j\frac{\widetilde{\Lambda}}{f_a} \right \rfloor   =\left \lfloor \sum_{j=0}^{i} f_0 \frac{\widetilde{\Lambda}}{f_a} \left ( \frac{f_{\Lambda-1}}{f_0} \right ) ^{\frac{j}{\Lambda -1}} \right \rfloor
%% \end{equation}

%% \begin{equation}\label{serieAmostralLog}
%% \left\{t_i^{\;\overline{f_0,\,f_{\Lambda-1}}}\right\}_0^{\Lambda-1}=\left\{\,\widetilde{l}_{\gamma_i \% \widetilde{\Lambda}}\,\right\}_0^{\Lambda-1}
%% \end{equation}

%% \begin{figure}[h!]
%%     \centering
%%         \includegraphics[width=\textwidth]{figuras/transicao}
%%     \caption{Transições de intensidade para diferentes valores de $\alpha$ (veja equações~\ref{seqAmp} e ~\ref{transAmp}).}
%%         \label{fig:transicao}
%% \end{figure}


%% O termo $\frac{i}{\Lambda-1}$ varre o intervalo $[0,1]$ e pode-se elevá-lo a uma potência
%% para que o início da transição seja mais suave ou abrupto.
%% Este procedimento é útil para variações de energia
%% da onda vibratória para alteração do volume\footnote{A mudança do volume (qualidade psicofísica) ocorre através de diferentes características 
%% do som, como a reverberação e a concentração de harmônicos agudos, dentre as quais está a energia da onda.
%% A manipulada com mais facilidade é a energia da onda (veja equação~\ref{eq:potencia}) e esta também pode variar de diferentes formas.
%% Uma forma mais simples é variar a amplitude através da multiplicação da sequência toda
%% por um número real. O aumento de energia sem variação de
%% amplitude é a \emph{compressão sonora}, útil na
%% produção musical atual.\cite{guillaume}}. Basta multiplicar a sequência original
%% (seja ela gerada ou pré-estabelecida) pela sequência $a_{\Lambda-1}^{\left( \frac{i}{\Lambda-1} \right )^\alpha}$
%% onde $\alpha$ é o coeficiente citado e $a_{\Lambda-1}$ é fração da amplitude original que se visa atingir ao final da transição.

%% Assim, para variações de amplitude:

%% \begin{equation}\label{seqAmp}
%% \{a_i\}_0^{\Lambda-1}=\left \{ a_0 \left ( \frac{a_{\Lambda-1}}{a_0} \right )^{\left ( \frac{i}{\Lambda-1} \right )^\alpha} \right \}_0^{\Lambda-1}=\left \{ \left ( {a_{\Lambda-1}} \right )^{\left ( \frac{i}{\Lambda-1} \right )^\alpha} \right \}_0^{\Lambda-1} \text{ com } a_0=1
%% \end{equation}

%% \begin{equation}\label{transAmp}
%% T_i^{'}=T_i \odot A_i = \{t_i . a_i\}_0^{\Lambda-1}=\left \{ t_i . (a_{\Lambda-1} )^{\left ( \frac{i}{\Lambda-1} \right )^\alpha} \right \}_0^{\Lambda-1}
%% \end{equation}

%% Pode-se tomar $a_0=1$ para iniciar a nova sequência com a amplitude original e então ir modificando com o decorrer das amostras.
%% Esta restrição faz com que o termo $a_{\Lambda-1}$ seja a variação da amplitude.
%% Caso $\alpha=1$, a variação de amplitude segue exatamente a progressão geométrica que caracteriza
%% a percepção linear. A figura~\ref{fig:transicao} exibe as transições para diferentes valores de $\alpha$ e para a transição entre os valores $1$ e $2$, um ganho de $\approx 6dB$ segundo a equação~\ref{eq:ampVol}.


%% Algum cuidado é necessário para lidar com $a=0$.
%% Na equação~\ref{seqAmp}, se $a_0=0$ há divisão por zero e
%% se $a_{\Lambda-1}=0$, há uma multiplicação por zero. Ambos os casos
%% tornam o procedimento inútil pois nenhum número diferente de zero pode ser representado como uma proporção com relação ao zero. Pode-se resolver isso escolhendo um número suficientemente pequeno como $-80dB\;\Rightarrow a=10^{\frac{-80}{20}}=10^{-4}$ como o volume mínimo no caso de um
%% \emph{fade in} ($a_0=10^{-4}$) ou de um \emph{fade out} ($a_{\Lambda-1}=10^{-4}$).


%% Para uma amplificação linear, mas não linear para a percepção, basta usar uma sequência $\{a_i\}$ adequada:

%% \begin{equation}\label{seqAmpLin}
%% a_i=a_0 + (a_{\Lambda-1}-a_0)\frac{i}{\Lambda-1}
%% \end{equation}

%% Aqui convém a conversão de decibels para amplitude. Assim, as equações ~\ref{ampDec} e \ref{transAmp}
%% especificam a transição de $V_{dB}$ decibels:

%% \begin{equation}\label{seqAmpDB}
%% T_i^{'}=\left\{ t_i 10^{\frac{V_{dB}}{20}\left( \frac{i}{\Lambda-1} \right)^\alpha} \right\}_0^{\Lambda-1}
%% \end{equation}

%% para o caso geral de variações de amplitude segundo a progressão geométrica. Quanto maior o valor de $\alpha$, mais suave é a introdução do som e mais intenso o final da transição. $\alpha>1$ resulta em transições de volume muitas vezes chamadas de \emph{slow fade} enquanto $\alpha<1$ resulta em \emph{fast fade}.\cite{guillaume}

%% As transições lineares serão usadas para
%% as sínteses AM e FM e a aplicação das transições
%% logarítmicas para os tremolos e vibratos.
%% Uma exploração não oscilatória destas variações
%% está na montagem musical \emph{Transita para metro},
%% cujo código está no Apêndice~\ref{ap:transita} e online
%% na \massa.\cite{MASSA}

%% \subsection{Aplicação de filtros digitais}\label{subsec:filtros}
%% Esta subseção limita-se a uma descrição
%% do processamento das sequências, por convolução
%% e equação a diferenças, e em aplicações
%% imediatas, pois a complexidade facilmente
%% foge ao escopo\footnote{A elaboração de filtros
%% constitui uma área reconhecidamente complexa, com literatura
%% e pacotes de software dedicados. 
%% Recomendamos ao leitor
%% interessado uma visita à nossa bibliografia.\cite{Openheim,smith}}. A aplicação de filtros pode
%% ser parte constituinte da síntese ou feita posteriormente
%% como parte dos processos tipicamente chamados de tratamento sonoro.

%% \begin{itemize}

%% \item  Convolução e filtros de resposta ao impulso finita (FIR)

%% \begin{figure}[h!]
%%     \centering
%%         \includegraphics[width=\textwidth]{figuras/convolucao______}
%%     \caption{Interpretação gráfica da convolução. Cada amostra resultante é a soma das amostras anteriores de um sinal uma a uma multiplicadas pelas amostras retrógradas do outro sinal.}
%%         \label{fig:conv}
%% \end{figure}

%% Os filtros aplicados por convolução são conhecidos
%% pela sua sigla FIR (do inglês Finite Impulse Response)
%% e são caracterizados por possuírem uma representação amostral
%% finita no tempo. Esta representação amostral é chamada
%% de 'resposta ao impulso' $\{h_i\}$. Os filtros FIR são aplicados 
%% no domínio temporal ao som
%% digitalizado pela convolução do som com a 
%% resposta ao impulso do filtro\footnote{Pode-se aplicar o filtro do domínio espectral através da multiplicação das transformadas de Fourier de ambos o som e a resposta ao impulso, e então realizada a transformada inversa de Fourier do espectro resultante.\cite{Openheim}}. Para os fins deste trabalho, a
%% convolução fica definida como:

%% \begin{equation}\label{eq:conv}
%% \begin{split}
%% \left\{t_i'\right\}_0^{\Lambda_t+\Lambda_h-2\; = \;\Lambda_{t\, '}-1} =\{(T_j*H_j)_i\}_0^{\Lambda_{t \, '}-1} & =\left \{ \sum_{j=0}^{min(\Lambda_h-1,i)}h_{j} . t_{i-j} \right \}_0^{\Lambda_{t\, '}-1} \\
%%     & =\left \{ \sum_{j=max(i+1-\Lambda_h,0)}^{i}t_j . h_{i-j} \right \}_0^{\Lambda_{t\, '}-1}
%% \end{split}
%% \end{equation}

%% Onde $t_i=0$ para as
%% amostras não definidas de antemão.
%% Ou seja, o som $\{t_i'\}$ resultante da convolução de $\{t_i\}$ com a resposta ao impulso $\{h_i\}$
%% tem cada i-ésima amostra $t_i$ substituída pela soma de suas últimas $\Lambda_h$ amostras $\{t_{i-j}\}_{j=0}^{\Lambda_h-1}$
%% multiplicadas uma a uma pelas amostras da resposta ao impulso $\{h_i\}_0^{\Lambda_h-1}$. Este
%% procedimento está ilustrado na figura~\ref{fig:conv}, onde a resposta ao impulso $\{h_i\}$
%% é percorrida na forma retrógrada e
%% $t_{12}'$ e $t_{32}'$ são duas amostras calculadas
%% pela convolução $(T_j*H_j)_i=t_i'$. O sinal resultante possui
%% sempre o tamanho $\Lambda_t+\Lambda_h -1=\Lambda_{t'}$.


%% Com este procedimento pode-se aplicar reverberadores, equalizadores, \emph{delays}
%% e vários outros tipos de filtros para fins de tratamento sonoro ou
%% efeitos musicais/artísticos.
 
%% A resposta ao impulso pode provir de medições
%% físicas ou da síntese. Uma resposta 
%% ao impulso para a aplicação
%% de reverberação pode resultar da gravação sonora em um ambiente ao disparar
%% um estalo que se assemelhe a um impulso ou 
%% de uma varredura em senoide, que transformada se aproxima
%% da resposta em frequência.
%% Ambas são respostas ao impulso
%% que, convoluidas com a sequência sonora, resultam na própria sequência
%% com uma reverberação que se assemelha àquela do ambiente 
%% em que ocorreu a medição.\cite{Cook}

%% A transformada inversa
%% de Fourier de uma envoltória par e real é uma
%% resposta ao impulso de um FIR. Este realiza
%% uma filtragem em frequência com a envoltória.
%% Quanto maior o número de amostras maior
%% a resolução da envoltória e também 
%% o processamento computacional, pois a convolução é cara.

%% Uma propriedade importante é o deslocamento temporal causado pela convolução com o impulso deslocado. Embora caro computacionalmente, 
%% pode-se criar linhas de \emph{delays} através da convolução do som com uma resposta ao impulso que possui um impulso
%% para cada reincidência do som.
%% Na figura~\ref{fig:delays}
%% pode-se observar o deslocamento causado pela convolução
%% com o impulso. Dependendo da densidade dos impulsos, o resultado
%% é de caráter rítmico (20 impulsos por segundo ou menos) ou de amálgama
%% sonoro (20-40 impulsos por segundo ou mais). Neste último caso,
%% ocorrem processos tipicamente vinculados à síntese granular, delays, reverbs e equalizações.

%% \begin{figure}[h!]
%%     \centering
%%         \includegraphics[width=\textwidth]{figuras/delays__}
%%     \caption{Convolução com o impulso: deslocamento (a), linhas de delays (b) e síntese granular~(c). Dispostos em ordem crescente de densidade de pulsos.}
%%         \label{fig:delays}
%% \end{figure}


%% \item Filtros de resposta ao impulso infinita (IIR)

%% Esta classe de filtros é
%% conhecida pela sigla IIR (do inglês Infinite Impulse Response)
%% e é caracterizada por possuir uma representação temporal
%% infinita, i.e. a resposta ao impulso não converge para zero.
%% Sua aplicação é usualmente feita pela equação:

%% \begin{equation}\label{eq:diferencas}
%% t_i' = \frac{1}{b_0}\left ( \sum_{j=0}^Ja_j . t_{i-j} + \sum_{k=1}^Kb_k . t_{i-k}' \right )
%% \end{equation}

%% com $b_0=1$ na grande maioria dos casos pois pode-se normalizar as variáveis:
%% $a_j'=\frac{a_j}{b_0}$ e $b_k'=\frac{b_k}{b_0} \Rightarrow b_0' = 1$.
%% A equação~\ref{eq:diferencas} é chamada 'equação a diferenças' por exibir as amostras resultantes $\left\{t_i'\right\}$
%% através das diferenças entre as amostras originais $\{t_i\}$ e as amostras resultantes anteriores $\left\{t_{i-k}'\right\}$.

%% Existem
%% diversos métodos e ferramentas para a elaboração de filtros IIR
%% e segue abaixo uma seleção com fins didáticos e para consulta futura por
%% utilidade.
%% São filtros bem comportados e cujas
%% filtragens estão na figura~\ref{fig:iir}.

%% No caso dos filtros de ordem simples, a frequência de corte $f_c$ é onde 
%% o filtro realiza uma atenuação de $-3dB \approx 0.707 $ da amplitude original.
%% No caso dos filtros passa e rejeita banda, esta mesma atenuação é
%% resultado de duas especificações: $f_c$ (neste caso mais bem compreendida como 'frequência central') e a largura de banda $bw$,
%% em ambas as frequências $f_c \pm bw$ há uma atenuação de $\approx 0.707$ da amplitude original.
%% Existe amplificação do som no caso dos filtros passa e rejeita banda quando a frequência
%% de corte é baixa e a largura de banda é grande o suficiente. Nos agudos, estes filtros apresentam
%% somente um desvio do perfil esperado, expandindo a envoltória para o lado grave da banda em
%% evidência.

%% Para filtros cujas respostas em frequência possuem outras envoltórias (para o módulo),
%% pode-se realizar cascatas destes filtros aplicando-os sucessivamente.
%% Outra possibilidade é utilizar alguma receita de filtro
%% biquad\footnote{Abreviação
%% de 'biquadrado' pois sua função de transferência possui dois polos e dois zeros, i.e. sua
%% forma normal consiste em dois polinômios quadráticos formando uma fração:
%% $\mathbb{H}(z)=\frac{a_0+a_1.z^{-1}+a_2.x^{-2}}{1- b_1.z^{-1} -b_2 . z^{-2}}$.}
%% ou rotinas para cálculo de coeficientes
%% de filtros Chebichev\footnote{Filtros Butterworth e Elípticos podem
%% ser considerados como casos específicos dos Filtros do tipo Chebichev.\cite{Openheim,smith}}.
%% Ambas as possibilidades são exploradas
%% por títulos em nossas referências, em especial~\cite{JOSFM,smith} e a coleção de filtros da comunidade \emph{Music-DSP}, da Universidade de Columbia.\cite{music-dsp,Openheim}

%% \end{itemize}

%% \begin{enumerate}
%% \item Passa-baixas de polo simples com módulo da resposta em frequência no canto superior esquerdo da figura~\ref{fig:iir}. A fórmula geral tem
%% por referência da frequência de corte $f_c \in (0,\frac{1}{2})$,
%% fração da frequência de amostragem $f_a$
%% em que há aproximadamente uma atenuação de $3dB$.
%% Os coeficientes do filtro IIR
%% $a_0$ e $b_1$ 
%% são dados através da variável intermediária $x \in [e^{-\pi},1]$:

%% \begin{figure}[h!]
%%     \centering
%%         \includegraphics[width=\textwidth]{figuras/iir___}
%%     \caption{Módulos da resposta em frequência (a), (b), (c) e (d) respectivamente dos filtros IIR das equações \ref{eq:passa-baixas}, \ref{eq:passa-altas}, \ref{eq:passa-banda} e \ref{eq:rejeita-banda} para diferentes frequências de corte, frequências centrais e larguras de banda.}
%%         \label{fig:iir}
%% \end{figure}


%% \begin{equation}\label{eq:passa-baixas}
%% \begin{split}
%% x & =e^{-2\pi f_c} \\
%% a_0 & =  1-x \\
%% b_1 & =  x
%% \end{split}
%% \end{equation}

%% \item Passa-altas de polo simples com o módulo da resposta em frequência no canto superior direito da figura~\ref{fig:iir}. A fórmula geral,
%% com frequência de corte $f_c \in (0,\frac{1}{2})$, é calculada através da variável
%% intermediária $x \in [e^{-\pi},1]$:


%% \begin{equation}\label{eq:passa-altas}
%% \begin{split}
%% x & =e^{-2\pi f_c} \\
%% a_0 & =  \frac{x+1}{2} \\
%% a_1 & =  -\frac{x+1}{2} \\
%% b_1 & =  x
%% \end{split}
%% \end{equation}


%% %\item Passa-banda
%% %\item Rejeita-banda
%% \item Nó (\emph{notch filter}). Este filtro é parametrizado
%% pela frequência central\footnote{ Atenção com a frequência de corte também $f_c$ nos filtros passa baixas e passa altas.} $f_c$
%% e a largura de banda $bw$
%% - $f_c \pm bw$, que resultam em $0.707$ da amplitude, i.e. atenuação de $3dB$ -
%% ambos dados como frações de $f_a$, portanto $f,\; bw \in (0,0.5)$.

%% Por facilidade, sejam as variáveis auxiliares $K$ e $R$:

%% \begin{equation}\label{eq:varAux}
%% \begin{split}
%% R & = 1 - 3bw \\
%% K & = \frac{1-2R\cos(2\pi f_c) + R^2}{2 - 2 \cos (2 \pi f_c)}
%% \end{split}
%% \end{equation}

%% O filtro passa banda do canto inferior esquerdo da figura~\ref{fig:iir}
%% possui os seguintes coeficientes para a equação~\ref{eq:diferencas}:

%% \begin{equation}\label{eq:passa-banda}
%% \begin{split}
%% a_0 & =  1 - K \\
%% a_1 & =  2(K-R)\cos (2\pi f_c) \\
%% a_2 & =  R^2-K \\
%% b_1 & =  2R \cos (2\pi f_c) \\
%% b_2 & =  -R^2
%% \end{split}
%% \end{equation}

%% Os coeficientes do filtro rejeita banda são:

%% \begin{equation}\label{eq:rejeita-banda}
%% \begin{split}
%% a_0 & =  K \\
%% a_1 & =  -2K\cos (2\pi f_c) \\
%% a_2 & =  K \\
%% b_1 & =  2R \cos (2\pi f_c) \\
%% b_2 & =  -R^2
%% \end{split}
%% \end{equation}

%% com o módulo de sua resposta em frequência 
%% disposto na parte inferior esquerda da figura~\ref{fig:iir}.

%% %\item Biquad: pela especificação de uma frequência central, da qualidade
%% %e da intensidade do filtro, este filtro é simples e usual para áudio,
%% %permitindo ajustes mais finos. Diversas receitas podem ser encontradas
%% %na literatura, recomendamos especialmente as diferentes especificações
%% %em ~\ref{musicDSP} e ~\ref{dspguide}.

%% \end{enumerate}

%% \subsection{Ruídos}\label{subsec:ruidos}
%% De forma geral, os sons sem altura definida 
%% são chamados ruídos.\cite{Lacerda}
%% Estes são constituintes importantes dos sons musicais de altura definida,
%% como os ruídos presentes nas notas do piano, do violino, etc. Além disso, os instrumentos
%% de percussão, em grande parte, não possuem altura definida e seus sons
%% são em geral compreendidos como ruídos.\cite{Roederer} Na música eletrônica, incluindo
%% a eletroacústica e gêneros de pista de dança, os ruídos possuem usos diversificados e comumente
%% característicos do estilo musical.\cite{Cook}

%% A ausência de uma altura definida é fruto da ausência de uma organização harmônica perceptível nas componentes senoidais que formam o som. Assim,
%% são incontáveis as possibilidades de gerar ruídos.
%% A utilização
%% de valores aleatórios para a geração da sequência sonora $T_i$
%% é um método atraente,
%% mas os resultados não são tão úteis, tendendo geralmente ao ruído branco.\cite{Cook}

%% Outra possibilidade é a geração de ruído através do espectro desejado, a partir
%% do qual executamos a transformada inversa de Fourier.
%% A distribuição espectral deve ser feita com cuidado 
%% pois caso se utilize a mesma fase,
%% ou fases com forte correlação, 
%% o som sintetizado possuirá energia bastante concentrada
%% em alguns trechos de sua duração.


%% \begin{figure}[htpq!]
%%     \centering
%%         \includegraphics[width=\textwidth]{figuras/ruidos___}
%%     \caption{Ruídos coloridos realizados através das equações~\ref{eq:branco}, \ref{eq:rosa}, \ref{eq:marrom}, \ref{eq:azul}, \ref{eq:violeta}: espectros e ondas sonoras resultantes.}
%%         \label{fig:ruidos}
%% \end{figure}


%% Abaixo elencamos alguns ruídos de espectro estático. São chamados \emph{coloridos} por terem sido associados a cores. 
%% A figura~\ref{fig:ruidos} mostra lado a lado o perfil espectral e
%% a sequência sonora. Os ruídos foram gerados com a mesma fase, então
%% pode-se observar o resultado das contribuições em diferentes regiões do espectro.

%% \begin{itemize}

%% \item O ruído branco deve seu nome por possuir energia distribuída
%% igualmente por todas as frequências. Pode-se realizar
%% o ruído branco com a transformada inversa dos seguintes coeficientes:


%% \begin{equation}\label{eq:branco}
%% \begin{split}
%% c_0 & =0 \quad \text{pois evita-se bias} \\
%% c_i & =e^{j.x}\;,\;\; j^2=-1 \;, \;\; x \; \text{randômico} \; \in \; [0,2\pi]\;,\;\; i \; \in \; \left[1, \, \frac{\Lambda}{2}-1\right] \\
%% c_{\Lambda/2} & = 1 \quad\quad \text{(se $\Lambda$ par)}\\ 
%% c_i & = c_{\Lambda - i}^*\;,\;\; \text{para}\;  i \; > \;  \frac{\Lambda}{2}
%% \end{split}
%% \end{equation}

%% O valor de $c_i$ calculado pela exponencial é apenas um artifício para resultar em módulo unitário e fase aleatória.
%% Já $c_{\Lambda/2}$ é sempre puramente real (como vimos na seção anterior).

%% \item O ruído rosa possui uma queda de $3dB$ por oitava. Este ruído é muito usual no teste de equipamentos e montagens de aparelhos além de presença destacada na natureza~\cite{Roederer}. 

%% \begin{equation}\label{eq:rosa}
%% \begin{split}
%% f_{\text{min}} & \approx 15 Hz \\
%% f_i & = i \frac{f_a}{\Lambda} \;, \;\; \quad i \;\leq\; \frac{\Lambda}{2},\;\; i\;\in\;\mathbb{N}  \\
%% \alpha_i & = \left(10^{-\frac{3}{20}}\right)^{\log _2 \left ( \frac{f_i}{f_{\text{min}}} \right )}  \\
%% c_i & =0\;,\;\; \forall \; i \; : f_i<f_{\text{min}} \\
%% c_i & =e^{j.x} . \alpha_i\;,\;\; j^2=-1 \;, \;\;\  x \;\; \text{randômico} \; \in \; [0,2\pi]\;,\;\; \forall \; i \; : f_{\text{min}} \le f_i < f_{\lceil \Lambda/2-1 \rceil}  \\
%% c_{\Lambda/2} & = \alpha_{\Lambda/2} \quad\quad \text{(se $\Lambda$ par)}\\ 
%% c_i & = c_{\Lambda - i}^*\;,\;\; \text{para}\;  i \; > \;  \Lambda/2
%% \end{split}
%% \end{equation}

%% A frequência mínima $f_{\text{min}}$ pode ser escolhida com base no limite da audição, pois não se escuta como altura uma componente sonora cuja frequência esteja abaixo de $\approx\; 20Hz$.

%% Os ruídos restantes podem ser feitos com base no procedimento descrito para 
%% o ruído rosa, bastando que modificar detalhes, em especial a equação que define $\alpha_i$.

%% \item O ruído marrom deve seu nome a Robert Brown, que descreveu o movimento browniano.
%% Embora esta origem seja um tanto díspar do que pode-se considerar motivo para uma associação com a cor marrom, o ruído sonoro ficou consagrado com este nome. De qualquer forma, é bastante comum declarar satisfatória a associação do ruído com a cor marrom, uma vez que os ruídos branco e rosa são mais estridentes e relacionados a cores mais intensas~\cite{Cook,guillaume}.

%% O que caracteriza este ruído é a queda de $6dB$ por oitava. Desta forma, $\alpha_i$ 
%% no conjunto \ref{eq:rosa} fica:

%% \begin{equation}\label{eq:marrom}
%% \alpha_i=(10^{-\frac{6}{20}})^{\log _2 \left( \frac{f_i}{f_{\text{min}}} \right )}
%% \end{equation}

%% \item No ruído azul há ganho de $3dB$ por oitava em uma banda limitada
%% pela frequência mínima $f_{\text{min}}$ e a frequência
%% máxima $f_{\text{máx}}$. Assim, também com base no conjunto de equações \ref{eq:rosa}:

%% \begin{equation}\label{eq:azul}
%% \begin{split}
%% \alpha_i & = (10^{\frac{3}{20}})^{\log _2 \left ( \frac{f_i}{f_{\text{min}}} \right )} \\
%% c_i & =0\;,\;\; \forall \; i \; : f_i<f_{\text{min}} \;\; \text{ou} \;\; f_i>f_{\text{máx}} \\
%% \end{split}
%% \end{equation}

%% \item O ruído violeta é similar ao ruído azul, mas o ganho é de $6dB$ por oitava:

%% \begin{equation}\label{eq:violeta}
%% \alpha_i = (10^{\frac{6}{20}})^{\log _2 \left ( \frac{f_i}{f_{\text{min}}} \right )} \;\;, \quad f_{\text{min}} \approx 15 Hz \\
%% \end{equation}

%% \item O ruído preto possui perdas maiores que $6dB$ por oitava, assim:

%% \begin{equation}\label{eq:preto}
%% \alpha_i=(10^{-\frac{\beta}{20}})^{\log _2 \left( \frac{f_i}{f_{\text{min}}} \right )}\;\;, \quad \beta > 6
%% \end{equation}



%% \item O ruído cinza é definido como
%% um ruído branco sujeito a uma das curvas iso-audíveis. Estas curvas são resultados
%% experimentais e necessárias para a obtenção de $\alpha_i$. Uma implementação da ISO 226, a última revisão destas curvas, está na toolbox \massa.\cite{MASSA}

%% \end{itemize}

%% Foram expostos somente ruídos com espectro estático. Existem classificações
%% de ruídos com variações do espectro no decorrer do tempo. Existem também ruídos que
%% são fundamentalmente transientes, como os clicks e os chirps. O primeiro é modelado
%% facilmente por um impulso relativamente isolado, enquanto o segundo não é um ruído, mas uma varredura rápida de 
%% alguma banda de frequência.\cite{Cook}

%% Os ruídos das equações \ref{eq:branco}, \ref{eq:rosa}, \ref{eq:marrom},
%% \ref{eq:azul}, \ref{eq:violeta} estão na figura \ref{fig:ruidos}. Os espectros foram feitos com a mesma fase em cada coeficiente de mesma frequência, de forma que
%% se pode observar a contribuição dos harmônicos agudos e
%% das frequências graves.


%% \subsection{Tremolo e vibrato, AM e FM}\label{subsec:tvaf}

%% O vibrato é uma variação periódica de altura (frequência) e 
%% o tremolo é uma variação periódica de volume (intensidade).\footnote{Alguns instrumentos e contextos musicais usam nomenclaturas diferentes. Por exemplo, no piano, o chamado tremolo é um vibrato e um tremolo segundo a classificação aqui utilizada. As definições presentes neste trabalho tem por base uma literatura mais abrangente do que a utilizada para um único instrumento, prática ou tradição musical e comum em contextos de teoria musical e musica eletrônica.\cite{Lacerda,Harmonia}}
%% Para o caso geral, o vibrato é descrito da seguinte forma:


%% \begin{equation}\label{vbrGamma}
%% \gamma_i'=\left \lfloor i f' \frac{\widetilde{\Lambda}_M}{f_a} \right \rfloor
%% \end{equation}

%% \begin{equation}\label{vbrAux}
%% t_i'=\widetilde{m}_{\gamma_i' \;\% \widetilde{\Lambda}_M}
%% \end{equation}

%% \begin{equation}\label{vbrF}
%% f_i=f \left ( \frac{f + \mu }{f} \right )^{t_i'}=f . 2^{t_i'\frac{\nu}{12}}
%% \end{equation}

%% \begin{equation}\label{vbrGamma2}
%% \Delta_{\gamma_i}=f_i\frac{\widetilde{\Lambda}}{f_a} \quad \Rightarrow \quad \gamma_i = \left \lfloor \sum_{j=0}^{i} f_j \frac{\widetilde{\Lambda}}{f_a} \right \rfloor = \left \lfloor \sum_{j=0}^{i} \frac{\widetilde{\Lambda}}{f_a}f \left ( \frac{f + \mu }{f} \right )^{t_j'}  \right \rfloor= \left \lfloor \sum_{j=0}^{i} \frac{\widetilde{\Lambda}}{f_a}f . 2^{t_j'\frac{\nu}{12}}  \right \rfloor
%% \end{equation}

%% \begin{equation}\label{vbrT}
%% T_i^{f, vbr(f',\,\nu)}=\left\{ t_i^{f,vbr(f',\,\nu)} \right\}_0^{\Lambda-1}=\left\{ \widetilde{l}_{\gamma_i \%\; \widetilde{\Lambda} } \right\}_0^{\Lambda-1}
%% \end{equation}


%% \begin{figure}[h!]
%%     \centering
%%         \includegraphics[width=\textwidth]{figuras/vibrato___}
%%     \caption{Espectrograma de um som com vibrato senoidal de $3Hz$ e profundidade de uma oitava em uma dente de serra de $1000Hz$ (considerada $f_a=44.1kHz$).}
%%         \label{fig:vibrato}
%% \end{figure}

%% Para a correta realização do vibrato, é importante atenção para as duas tabelas e sequências.
%% A tabela $\widetilde{M}_i$ de tamanho $\widetilde{\Lambda}_M$ e a sequência de índices $\gamma_i'$ formam a sequência $t_i'$
%%  que é o padrão da oscilação da frequência enquanto
%% a tabela $\widetilde{L}_i$ de tamanho $\widetilde{\Lambda}$ e a sequência de índices $\gamma_i$ formam $t_i$ que é o som em si.
%% As variáveis $\mu$ e $\nu$ quantificam a intensidade do vibrato: $\mu$ é uma medida direta da quantidade
%% de Hertz envolvidos no limite superior da oscilação e $\nu$ é a medida direta de semitons envolvidos na oscilação ($2\nu$ é o número de semitons entre os picos superiores e inferiores de oscilação da frequência do som $\{t_i\}$ causada pelo vibrato).
%% É conveniente $\nu=\log_{2}\frac{f+\mu}{f} $ neste caso pois o aumento máximo de frequência
%% não equivale à diminuição máxima, mas a variação de semitons se mantém.

%% A Figura \ref{fig:vibrato} é o espectrograma de um vibrato artificial de uma nota em
%% $1000Hz$ (entre um si e um dó) e cujo desvio da frequência atinge uma oitava
%% para cima e para baixo. Qualquer forma de onda pode
%% ser utilizada para gerar o som e o padrão de oscilação do vibrato
%% em quaisquer frequência de oscilação e desvio de altura envolvidos\footnote{O desvio de altura
%% é chamado profundidade do vibrato e é geralmente dado por conveniência em semitons ou cents.}. Estas oscilações com formas precisas e amplitudes arbitrárias não são praticáveis em instrumentos musicais tradicionais, introduzindo novidade nas possibilidades artísticas.

%% O caso do tremolo é semelhante: $f'$, $\gamma_i'$ e $t_i'$ permanecem os mesmos.
%% A sequência de amplitudes a serem multiplicadas pela sequência original $t_i$ fica:

%% \begin{equation}\label{trA}
%% a_i=10^{\frac{V_{dB}}{20}t_i' } = a_{\text{máx}}^{t_i'}
%% \end{equation}

%% \begin{equation}\label{trT}
%% T_i^{tr(f')}=\left \{ t_i^{tr(f')} \right \}_0^{\Lambda-1}=\{ t_i . a_i \}_0^{\Lambda-1}=\left \{t_i .10^{t_i' \frac{V_{dB}}{20}}    \right \}_0^{\Lambda-1}=\left\{t_i . a_{\text{máx}}^{t_i'} \right\}_0^{\Lambda-1}
%% \end{equation}

%% Onde $V_{dB}$ é a profundidade da oscilação em decibels do tremolo e $a_{\text{máx}}=10^{\frac{V_{dB}}{20}}$
%%  é o ganho máximo de amplitude envolvido.
%% A medição em decibels é pertinente pois o aumento máximo de amplitude
%% não equivale à diminuição máxima relacionada, enquanto a diferença em decibels se mantém.

%% A figura~\ref{fig:tremolo} mostra a amplitude das sequências $\{a_i\}_0^{\Lambda-1}$ e $\{t_i'\}_0^{\Lambda-1}$
%% para três oscilações de um tremolo com forma da dente de serra. A curvatura é devida à progressão logarítmica de
%% intensidade. A frequência do tremolo é de $1,5Hz$ pois $f_a=44,1kHz \; \Rightarrow \; \text{duração} = \frac{i_{\text{máx}}=82000}{f_a}= 2s \; \Rightarrow \; \frac{3\text{oscilações}}{2s}=1,5$ oscilações por segundo ($Hz$). 

%% A montagem musical \emph{Vibra e treme} explora estes recursos dos tremolos e vibratos em associação e isoladamente
%% com frequências $f'$
%% e profundidades ($\nu$ e $V_{dB}$) diferentes, variações progressivas dos parâmetros\footnote{Os tremolos e vibratos ocorrem muitas vezes juntos em instrumentos tradicionais e na voz.}. A peça desenvolve também uma comparação entre os vibratos e tremolos em escala logarítmica e em escala linear para uma apreciação qualitativa. Seu código está no Apêndice~\ref{ap:vibra} e disponível online como parte da \emph{toolbox} \massa.


%% \begin{figure}[h!]
%%     \centering
%%         \includegraphics[width=\textwidth]{figuras/tremolo}
%%     \caption{Tremolo de profundidade $V_{dB}=12dB$ com padrão oscilatório de uma dente de serra em $f'=1.5Hz$ em uma senoide de $f=40Hz$ (considerada taxa de amostragem $f_a=44,1kHz$).}
%%         \label{fig:tremolo}
%% \end{figure}

%% No aumento progressivo de $f'$,
%% a aproximação do limiar de frequência
%% para audição do fenômeno sonoro como
%% altura ($\approx 20Hz$) gera rugosidades para
%% ambos tremolos e vibratos. Estas rugosidades são muito apreciadas
%% tanto na tradição erudita quanto na música eletrônica atual, especialmente no \emph{Dubstep}.
%% A rugosidade também é gerada através de conteúdos espectrais que geram batimentos.\cite{Porres,porres2009} A sequência \emph{Bela Rugosi}
%% explora este limiar com concomitâncias de tremolos e vibratos na mesma
%% voz, com intensidades e formas de onda diferentes. Seu código está o Apêndice~\ref{ap:bela} e disponível online como parte da \massa.

%% Aumentando ainda mais a frequência, estas oscilações
%% deixam de ser eventos identificáveis. 
%% Neste caso, as oscilações são
%% audíveis como altura. Assim, $f'$, $\mu$ e a forma de onda realizam alterações espectrais no som
%% original $T_i$ de formas diferentes para os tremolos e para os vibratos. São as 
%% chamadas sínteses AM (\emph{Amplitude Modulation}) e FM (\emph{Frequency Modulation}), respectivamente.
%% Estas são técnicas conhecidas, com aplicações em sintetizadores
%% como o \emph{Yamaha DX7}, e com aplicações fora da música, como em telecomunicações para transmissão de informação via ondas eletromagnéticas (ex. rádios AM e FM).

%% Para fins musicais e em resumo, pode-se entender a síntese FM através do caso entre senoides
%% e decompor os sinais em seus espectros de Fourier (i.e. senoidais) para casos mais complexos.
%% Assim, a síntese FM realizada com um 'vibrato' senoidal de frequência $f'$ e profundidade $\mu$ em um som também senoidal $T_i$ de frequência $f$
%% gera bandas centradas em $f$ e distantes $f'$ entre si:

%% \begin{equation}\label{eq:fmEsp}
%% \begin{split}
%% \{t_i'\} & = \left \{ \cos \left [f . 2 \pi \frac{i}{f_a-1} + \mu . sen \left ( f' . 2 \pi \frac{i}{ f_a -1 } \right ) \right ] \right \} = \\
%%  & = \left \{ \sum_{k=-\infty}^{+\infty} J_k(\mu) \cos \left [ f . 2 \pi \frac{i}{f_a-1} + k . f' . 2 \pi \frac{i}{f_a-1} \right ]  \right \} = \\
%%  & = \left \{ \sum_{k=-\infty}^{+\infty} J_k(\mu) \cos \left [ (f+k.f') . 2 \pi \frac{i}{f_a-1} \right ]  \right \}
%% \end{split}
%% \end{equation}

%% onde 
%% \begin{equation}\label{eq:Bessel}
%% J_k(\mu) = \frac{2}{\pi} \int_0^{\frac{\pi}{2}}\left [ cos \left (\overline{k}\;\frac{\pi}{2} + \mu . \sin w \right ) . cos \left ( \overline{k}\;\frac{\pi}{2} + k . w \right ) \right ] dw \quad , \quad \overline{k} = k \% 2 \;\;,\;\; k \in \mathbb{N}
%% \end{equation}
%% é a função de Bessel~\cite{BesselCCRMA,JOSFM} que na FM especifica a amplitude de cada componente. 

%% Nestas equações, a variação de frequência introduzida por $\{t_i'\}$ não respeita a progressão geométrica de frequência que acompanha a percepção de altura, mas sim a equação~\ref{freqLinear}. A utilização das equações~\ref{vbrF} para a FM está no Apêndice~\ref{cap:fmam}, onde é calculado o conteúdo espectral da síntese FM obtida com oscilações na escala logarítmica. De fato, o comportamento simples que torna a FM atraente é obtido somente com as variações lineares em~\ref{eq:fmEsp}.

%% O caso da modulação de amplitude (AM) é mais simples:

%% \begin{equation}\label{eq:amEsp}
%% \begin{split}
%% \{t_i'\}_0^{\Lambda-1} & =\{(1+a_i) . t_i\}_0^{\Lambda-1}= \left \{ \left [ 1+M.\sin \left ( f'.2\pi\frac{i}{f_a -1} \right ) \right] . P .\sin \left ( f.2\pi\frac{i}{f_a -1} \right ) \right \}_0^{\Lambda-1} = \\
%%                        & =  \left\{P.\sin \left( f.2\pi\frac{i}{f_a -1}  \right ) + \frac{P.M}{2} \left [ \sin \left( (f-f').2\pi\frac{i}{f_a -1}  \right ) + \sin \left( (f+f').2\pi\frac{i}{f_a -1}  \right ) \right ] \right \}_0^{\Lambda-1}
%% \end{split}
%% \end{equation}

%% Ou seja, o som resultante é o original
%% e a reprodução de seu conteúdo espectral acima e abaixo da frequência
%% original, distantes $f'$ de $f$. Novamente, isso é obtido com a variação na escala linear de amplitude. No Apêndice~\ref{cap:fmam} está uma exposição do espectro da AM realizada com a oscilação na escala logarítmica de amplitude. Esta também perde o comportamento simples.

%% A sequência $T_i$ de frequência $f$, chamada portadora, é modulada pela $f'$, chamada moduladora. No jargão de FM e AM, $\mu$ e $\alpha=10^{\frac{V_{dB}}{20}}$ são chamados índices de modulação. Assim, 
%% para o padrão de vibração da moduladora $\{t_i'\}$ as equações:

%% \begin{equation}\label{fmGammaAux}
%% \gamma_i'=\left \lfloor i f' \frac{\widetilde{\Lambda}_M}{f_a} \right \rfloor
%% \end{equation}

%% \begin{equation}\label{fmAux}
%% t_i'=\widetilde{m}_{\gamma_i' \;\% \widetilde{\Lambda}_M}
%% \end{equation}

%% Para aplicação da moduladora $\{t_i'\}$ na portadora $\{t_i\}$
%% por FM:

%% \begin{equation}\label{fmF}
%% f_i=f + \mu . t_i'
%% \end{equation}

%% \begin{equation}\label{fmGamma}
%% \Delta_{\gamma_i}=f_i\frac{\widetilde{\Lambda}}{f_a} \quad \Rightarrow \quad \gamma_i = \left \lfloor \sum_{j=0}^{i} f_j \frac{\widetilde{\Lambda}}{f_a} \right \rfloor = \left \lfloor \sum_{j=0}^{i} \frac{\widetilde{\Lambda}}{f_a}(f+\mu . t_j') \right\rfloor
%% \end{equation}

%% \begin{equation}\label{fmT}
%% T_i^{f,\, FM(f',\,\mu)}=\left\{ t_i^{f,\,FM(f',\,\mu)} \right\}_0^{\Lambda-1}=\left\{\,\widetilde{l}_{\gamma_i \%\; \widetilde{\Lambda} } \,\right\}_0^{\Lambda-1}
%% \end{equation}

%% Onde $\widetilde{l}$ é um período da forma de onda de comprimento $\widetilde{\Lambda}$ da portadora.

%% Para realizar a AM, basta modular $\{t_i\}$ com $\{t_i'\}$ através das equações:

%% \begin{equation}\label{amA}
%% a_i=1 + \alpha . t_i'
%% \end{equation}

%% \begin{equation}\label{amT}
%% T_i^{f,\,AM(f',\,\alpha)}=\left\{ t_i^{f,\,AM(f',\,\alpha)} \right\}_0^{\Lambda-1}=\{ t_i . a_i \}_0^{\Lambda-1}=\{t_i . (1 + \alpha . t_i')    \}_0^{\Lambda-1}
%% \end{equation}



%% \subsection{Usos musicais}\label{subsec:mus2}
%% A este ponto as possibilidades musicais explodiram. As
%% características de altura (dada pela frequência),
%% timbre (dado pela forma de onda e filtragens),
%% volume (dado pela intensidade) e duração (dada pelo número de amostras)
%% podem ser consideradas
%% de forma absoluta ou tratadas ao longo de sua duração,
%% com a única exceção da duração em si.

%% Desta forma, os usos musicais aqui dispostos são uma coleção de possibilidades
%% com o objetivo de exemplificar manipulações sonoras que resultem algo
%% musical, por razões variadas e aprofundadas na próxima seção.

%% \subsubsection{Vínculos entre características}

%% Uma outra possibilidade interessante é usar vínculos
%% entre os parâmetros do tremolo e do vibrato e algum parâmetro da nota básica,
%% como a frequência. Assim, pode-se estabelecer que
%% a frequência do vibrato é diretamente proporcional à altura, e a profundidade do tremolo é inversamente proporcional
%% à altura.
%% Desta forma, com as equações \ref{vbrGamma}, \ref{vbrF} e \ref{trA}
%% pode-se escrever:

%% \begin{equation}\label{eq:vinculos}
%% \begin{split}
%% f^{vbr} = f^{tr} & = func_a(f) \\
%% \nu & = func_b(f) \\
%% V_{dB} & = func_c(f)
%% \end{split}
%% \end{equation}

%% Com $f^{vbr}$ e $f^{tr}$ como $f'$ nas equações de referência, ou seja, a frequência
%% de oscilação do vibrato e do tremolo da equação~\ref{vbrGamma}. Já $\nu$ e $V_{dB}$ são as profundidades
%% do vibrato e do tremolo, respectivamente. As funções $func_a$,
%% $func_b$ e $func_c$ são arbitrárias e dependentes das intenções musicais. A montagem 
%% \emph{Tremolos, vibratos e a frequência} explora
%% recursos como este e variações da forma de onda da oscilação com vínculos, de modo a formar um \emph{idioma musical}\footnote{Veja na próxima seção.}. Seu código está no Apêndice~\ref{ap:tremolos} e também disponível online como parte da caixa de ferramentas \massa.


%% Com relação à convolução, pode-se estabelecer uma duração como pulso musical - a exemplo de um pulso BPM - 
%% e distribuir impulsos no decorrer deste pulso, de forma a estabelecer métricas e ritmos\footnote{Lembrando
%% que a convolução com o impulso resulta no som deslocado ao instante de ocorrência do impulso.}.
%% Por exemplo, 2 impulsos igualmente espaçados fazem uma
%% divisão binária básica do pulso. Dois sinais, um com 2 pulsos e outro com 3 pulsos,
%% ambos com os impulsos igualmente espaçados na duração do pulso, resultam na manutenção
%% do pulso, com uma marcação rítmica usada tanto em divisões binárias quanto ternárias em diversos
%% estilos de música étnica e tradicional.\cite{Gramani} 
%% Os próprios valores absolutos destes impulsos resultam em proporções entre as amplitudes dos sinais
%% convoluidos.
%% Este recurso da métrica
%% estabelecida pela convolução com impulsos é explorado na montagem \emph{Trenzinho de caipiras impulsivos}. Os recursos explorados incluem a criação de amálgamas sonoros provenientes da síntese granular e esta montagem é já uma ponte para a próxima seção. Veja especialmente a figura~\ref{fig:pulsoSubAgl}. O código da peça está na seção~\ref{ap:trenzinho} e disponível como parte da toolbox \massa.

%% \subsubsection{Fonte em movimento, receptor em movimento, efeito Doppler}

%% Retomada a exposição da subseção~\ref{subsec:spac}, quando uma fonte ou um receptor está em movimento, suas características idealmente são atualizadas a cada amostra do sinal digital. As velocidades são decompostas com relação à direção de cada ouvido. Assim, dada a velocidade $v_s$ da fonte ($s$ de \emph{source}), positiva se a fonte se move no sentido de se afastar do receptor e a velocidade $v_r$ do receptor, positiva no sentido de se aproximar da fonte, a frequência é dada pela conhecida fórmula do efeito Doppler:

%% \begin{equation}\label{eq:fDoppler}
%%     f=\left(\frac{v_{som}+v_r}{v_{som}+v_s}\right)f_0
%% \end{equation}

%% Com esta frequência e com as relações dadas pela nova DII da nova localização da fonte, pode-se realizar o efeito Doppler. Há um acréscimo para maior fidelidade ao fenômeno físico: um aumento da potência recebida. Pode-se entender que este aumento em potência é proporcional à velocidade relativa, que, a cada segundo, acrescenta aquele trecho percorrido, pela forma de onda, de potência: $\Delta P=P_0\left(\frac{v_r-v_s}{343,2}\right)$, com $P_0$ a potência do sinal.

%% Assim, pode-se obter a amplitude e frequência de uma fonte em movimento.
%% Esteja esta fonte à frente do receptor
%% a uma distância horizontal de $y_0$ metros e altura $z_0$ metros. A distância é dada por
%% $D_i=\left\{ d_i=\sqrt{ y_{i}^{2}+z_{0}^{2} } \right\}_0^{\Lambda-1}$
%% onde $y_i=y_0+v_s-v_r$ se consideradas $v_s$ e $v_r$ horizontais. A amplitude muda com a distância e com o fator de potência citado acima (veja a subseção~\ref{subsec:volume} para a conversão de potência para amplitude):

%% \begin{equation}\label{eq:aDoppler}
%%     A_i=\left\{ \frac{z_0}{d_i}A_{\Delta P}\right\}_0^{\Lambda-1} = \left\{ \frac{z_0}{\sqrt{y_i^2+z_0^2}} \sqrt{\frac{v_r-v_s}{343,2}+1}  \,\right\}_0^{\Lambda-1}
%% \end{equation}

%% Observe que o fator alterador de amplitude devido à distância é par enquanto a o fator devido à variação de potência é antissimétrico com relação ao cruzamento da fonte e do receptor. Já a frequência tem uma progressão simétrica com relação à altura, ou seja, os mesmos semitons (ou frações) acrescentados na aproximação são decrescidos no afastamento. Além disso, a transição é abrupta caso a fonte e o receptor se cruzem com exatidão, caso contrário, há uma progressão monotônica. No caso considerado, em que há uma altura fixa $z_0$, é necessário observar a componente de velocidade na direção entre o observador e a fonte:

%% \begin{equation}\label{eq:ffDoppler}
%%     F_i=\{f_i\}_0^{\Lambda-1}=\left\{\frac{v_{som} + v_r\frac{y_i}{\sqrt{z_0^2+y_i^2}}}{v_{som}+v_s\frac{y_i}{\sqrt{z_0^2+y_i^2}}}f_0\right\}_0^{\Lambda-1}
%% \end{equation}

%% No Apêndice~\ref{sec:cod2} está uma implementação em Python do efeito Doppler como descrito acima, com as alterações de frequência e amplitude, e com o cruzamento da fonte com o receptor.

%% \subsubsection{Filtragens e ruídos (subseções~\ref{subsec:ruidos} e~\ref{subsec:filtros})}
%% Com os filtros as possibilidades explodem ainda mais vertiginosamente. Pode-se convoluir um sinal para reverberá-lo, para
%% remover algum ruído, para gerar distorções ou para tratamento com intuito estético mesmo. Por exemplo,
%% a aplicação de um filtro passa banda em que deixa passar somente entre $1kHz$ e $3kHz$, resulta em sons
%% que parecem de telefone ou de televisão antiga. Ao remover com alguma precisão somente
%% a frequência de oscilação da rede elétrica (usualmente $50Hz$ ou $60Hz$) e harmônicas, pode-se remover
%% ruídos introduzidos por equipamentos de áudio.
%% Um uso mais incrementado
%% e propriamente musical é realizar filtragens em bandas específicas e usar estas bandas
%% pré-estabelecidas como um parâmetro adicional das notas.

%% Inspirado nos instrumentos tradicionais, pode-se aplicar uma a filtragem dependente do tempo~\cite{Roederer}.
%% Cascatas
%% destes filtros podem realizar filtragens complexas e mais precisas. A montagem \emph{Ruidosa faixa} explora
%% este recursos, realizando filtragens em ruídos diversos e síntese de ruídos diversos. O código da peça está no Apêndice~\ref{ap:ruidosa} e disponível online como parte da \massa.

%% Estes recursos utilizados em conjunto podem incidir na realização de um efeito chamado \emph{chorus}. A
%% exemplo do que ocorre com um coro de cantores, neste efeito o som é realizado com diversas pequenas modificações,
%% potencialmente aleatórias, em parâmetros como frequência central, presença (ou ausência) de vibrato
%% e tremolo e suas características, equalizações, volume etc. Para o resultado final, estas versões do som
%% inicial são então mixadas (ver equação \ref{eq:mixagem}). A peça \emph{Chorus infantil} realiza chorus de formas
%% diferentes em sons diferentes e seu código está no Apêndice~\ref{ap:chorus}. A \emph{toolbox} \massa\ disponibiliza online este \emph{script}.

%% \subsubsection{Reverberação}\label{subsubsec:reverb}
%% Com a mesma nomenclatura da subseção~\ref{subsec:spac} sobre espacialização, a reverberação tardia pode ser modelada como a convolução com um trecho de ruído colorido rosa, marrom ou preto, em decaimento exponencial de amplitude com relação também ao tempo. Assim, a atenuação nos agudos e irregularidade suave na resposta em frequência são contempladas de forma bastante satisfatória.\cite{JOSPhy,guillaume} Linhas de atrasos podem ser adicionadas como prefixo ao ruído com o decaimento, assim contemplado as duas partes temporais da reverberação: as primeiras reflexões e a reverberação tardia. Pode-se, para melhora de qualidade, calcular geometricamente a localização da última superfície em que cada frente de onda refletiu antes de chegar ao ouvido nos primeiros $100-200$ milissegundos e aplicar um filtro passa-baixas como em~\ref{subsec:filtros}. O ruído colorido pode ser introduzido gradualmente, desde o instante inicial dado pela incidência direta (i.e. sem reflexão alguma e dada pela DTI e DII), em \emph{fade-in}, atingindo o máximo no começo da 'reverberação tardia', quando as incidências geométricas perdem sua importância para as estatísticas do ruído em decaimento.

%% Como um exemplo de implementação, seja $\Delta_1$ a duração do primeiro período e $\Delta_R$ a duração total da reverberação ($\Lambda_1=\Delta_1 f_a$, $\Lambda_R=\Delta_R f_a$). Pode-se
%% associar uma probabilidade $p_i$ de uma reincidência do som na $i$-ésima amostra com amplitude em decaimento exponencial. Seguindo
%% as indicações na subseção~\ref{subsec:spac}, a reverberação $R_i^1$ do primeiro período pode ser descrita desta forma:

%% \begin{equation}\label{eq:p1rev}
%%     R_i^1=\left\{r_i^1\right\}_0^{\Lambda_1-1}\;:\;r_i^1=\left\{
%%         \begin{array}{l l}
%%             10^{\frac{V_{dB}}{20}\frac{i}{\Lambda_R-1}}\;  & \text{com probabilidade}\quad p_i=\left(\frac{i}{\Lambda_1}\right)^2 \\
%%                                      0 \; & \text{com probabilidade}\quad 1-p_i \\
%%         \end{array} \right.
%% \end{equation}

%% Onde $V_{dB}$ é o decaimento total considerado em decibels, tipicamente $-80dB$ ou $-120dB$.   Já a reverberação $R_i^2$, do segundo período, pode ser emulada por um ruído marrom $R_i^m$ (ou rosa $R_i^r$) em decaimento exponencial: 

%% \begin{equation}\label{eq:p2rev}
%%     R_i^2=\left\{r_i^2\right\}_{\Lambda_1}^{\Lambda_R-1}=\left\{10^{\frac{V_{dB}}{20}\frac{i}{\Lambda_R-1}}\,.\,r_i^m\right\}_{\Lambda_1}^{\Lambda_R-1}
%% \end{equation}

%% Como:
%% \begin{equation}\label{eq:rev}
%%     R_i=\left\{r_i\right\}_0^{\Lambda_R-1}\;:\;r_i=\left\{
%%         \begin{array}{l l}
%%             1\; & \text{se }\quad i=0 \\
%%             r_i^1\;  & \text{se }\quad 1\leq i<\Lambda_1-1 \\
%%                                      r_i^2 \; & \text{se}\quad \Lambda_1 \leq i < \Lambda_R-1 \\
%%         \end{array} \right.
%% \end{equation}

%% a aplicação da reverberação representada por $R_i$ é a simples convolução de $R_i$ (chamada 'resposta ao impulso' da reverberação) com a sequência sonora $T_i$, tal como descrito na subseção~\ref{subsec:filtros}.

%% A reverberação é conhecida por causar bastante interesse no ouvinte e tornar as sonoridades mais agradáveis. Além disso, a modificação do espaço em que são reverberadas a música, ou as sonoridades individualmente, constitui um macete (quase um clichê) para causar surpresa e interesse no ouvinte. 



%% \subsubsection{Envoltória ADSR}
%% A variação de volume no decorrer de um som é crucial para nossa percepção de timbre. A envoltória de volume chamada ADSR (sigla de \emph{Atack-Decay-Sustain-Release}) possui numerosas implementações em sintetizadores em hardware e software. Uma implementação pioneira pode ser encontrada no Hammond Novachord de 1938 e algumas variantes são citadas logo abaixo.\cite{ADSR}

%% A envoltória ADSR escolástica é caracterizada por 4 parâmetros: duração do ataque (tempo que o som demora para atingir seu volume máximo), duração do decaimento (segue ao ataque imediatamente), nível de volume de sustentação (em que o volume fica estável após o decaimento) e duração de soltura (após a sustentação, nesta duração o volume decai a zero).
%% Note que o tempo de sustentação não é especificado, pois é resultante da duração em si menos os tempos de ataque, decaimento e soltura.

%% A aplicação da envoltória ADSR com durações $\Delta_A$, $\Delta_D$ e $\Delta_R$, duração total $\Delta$ e nível de sustentação $a_S$, dado
%% como fração da amplitude máxima, em uma sequência sonora $t_i$ qualquer, pode ser feita da seguinte forma:

%% \begin{equation}\label{eq:adsr}
%% \begin{split}
%% \{a_i\}_0^{\Lambda_A-1} & = \left\{\xi\left(\frac{1}{\xi}\right)^{\frac{i}{\Lambda_A-1}}\right\}_0^{\Lambda_A-1} \;\;\quad\quad\quad \text{ou} \;\;\quad \left\{\frac{i}{\Lambda_A-1}\right\}_0^{\Lambda_A}\\
%% \{a_i\}_{\Lambda_A}^{\Lambda_A+\Lambda_D-1} & =\left\{a_S^{\frac{i-\Lambda_A}{\Lambda_D-1}}  \right\}_{\Lambda_A}^{\Lambda_A+\Lambda_D-1} \;\;\quad \quad\quad \text{ou} \quad\;\; \left\{1-(1-a_S)\frac{i-\Lambda_A}{\Lambda_D-1}\right\}_{\Lambda_A}^{\Lambda_A+\Lambda_D-1}\\
%% \{ a_i \}_{\Lambda_A+\Lambda_D}^{\Lambda-\Lambda_R-1} & =\left\{ a_S \right\}_{\Lambda_A+\Lambda_D}^{\Lambda-\Lambda_R-1} \\
%% \{ a_i \}_{\Lambda-\Lambda_R}^{\Lambda-1} & =\left\{ a_S\left(\frac{\xi}{a_S} \right)^{\frac{i-(\Lambda-\Lambda_R)}{\Lambda_R-1}} \right\}_{\Lambda-\Lambda_R}^{\Lambda-1} \quad\;\; \text{ou} \quad\;\; \left\{ a_S - a_S\frac{i+\Lambda_R-\Lambda}{\Lambda_R-1}\right\}_{\Lambda-\Lambda_R}^{\Lambda-1} 
%% \end{split}
%% \end{equation}

%% Com $\Lambda_X=\lfloor \Delta . f_a \rfloor\;\;\forall\;\; X \; \in (A,D,R,\;)$ e $\xi$ um valor pequeno que torne o \emph{fade in} e o \emph{fade out} satisfatórios, e.g. $\xi=10^{\frac{-80}{20}}=10^{-4}\;$ ou $\;\xi=10^{\frac{-40}{20}}=10^{-2}$. Quanto menor for $\xi$ mais lento é o \emph{fade}, a exemplo de $\alpha$ da figura~\ref{fig:transicao}. Já os termos do lado direito de~\ref{eq:adsr} podem realizar as entrada e saída do som a partir da intensidade zero, por serem lineares. 
%% Esquematicamente, a figura~\ref{fig:adsr} mostra esta envoltória ADSR, 
%% uma implementação clássica que comporta variações diversas. Por exemplo,
%% entre o ataque e o decaimento, pode-se adicionar uma partição adicional em que a amplitude
%% máxima perdura. Outro exemplo comum é o uso de traçados mais elaborados para o
%% ataque ou para o decaimento. A montagem musical \emph{ADa e SaRa}, disponível no Apêndice~\ref{ap:ada}
%% e na \massa, explora diversas destas configurações da envoltória ADSR


%% \begin{equation}\label{eq:adsrApl}
%% \left\{t_i^{ADSR}\right\}_0^{\Lambda-1} =\{t_i . a_i\}_0^{\Lambda-1}
%% \end{equation}

%% \begin{figure}[htpq!]
%%     \centering
%%         \includegraphics[width=\textwidth]{figuras/adsr}
%%     \caption{Envoltória ADSR (\emph{Attack, Decay, Sustain, Release}) e uma sequência sonora arbitrária submetida à envoltória. A variação linear de amplitude está acima. Abaixo a variação de amplitude é exponencial.}
%%         \label{fig:adsr}
%% \end{figure}


%% \afterpage{\blankpage}

%% \clearpage
%% \section{Organização de notas em música}\label{notasMusica}
%% Seja $ S_j=\left\{  s_j=T_i^j=\{t_i^j\}_{i=0}^{\Lambda_j-1} \right\}_{j=0}^{H-1} $ uma sequência $S_j$ de $H$ eventos
%% musicais $s_j$. Seja $S_j$ chamada uma 'estrutura musical' composta de eventos $s_j$ que são também estruturas musicais,
%% p.ex. notas.
%% Esta seção é dedicada às técnicas que tornam $S_j$ interessante e agradável na audição.

%% Os elementos de $S_j$ podem ser sobrepostos por mixagem, como na equação~\ref{eq:mixagem} e figura~\ref{fig:mixagem}, formando intervalos e acordes. Este é o 'pensamento vertical' da música. A concatenação de $S_j$, como na equação~\ref{eq:concatenacao} e na figura~\ref{fig:concatenacao}, forma sequências melódicas e ritmos, associados ao 'pensamento horizontal' na música. A frequência fundamental $f$ e o momento de início (ataque) são, em geral, as características mais importantes dos elementos de $S_j$. Estas viabilizam músicas de alturas (harmonia e melodia) e a presença da métrica temporal e ritmos, respectivamente.



%% \subsection{Afinação, intervalos, escalas e acordes}\label{subsec:afinacao}
%% \subsubsection{Afinação}
%% O dobro da frequência é uma oitava ascendente ($f=2f_0$).
%% A divisão da oitava em doze notas é o cânone da música ocidental clássica,
%% além de usos cerimoniais/religiosos e étnicos observados fora da tradição ocidental.\cite{Wisnick}
%% Doze semitons equidistantes para o ouvido formam uma oitava,
%% portanto, se $f=2^{\frac{1}{12}}f_0$, há um semitom entre $f_0$
%% e $f$.
%% O fator $\varepsilon=2^{\frac{1}{12}}$, um semitom, forma uma grade de notas
%% no espectro audível. Fixada uma frequência $f$, as frequências fundamentais possíveis
%% estão separadas por intervalos múltiplos de $\varepsilon$.
%% Esta precisão absoluta é característica de implementações
%% computacionais, os instrumentos reais possuem desvios destas frequências para melhor compatibilizar os harmônicos
%% de suas notas. Além disso, a referência fixa $\varepsilon=2^{\frac{1}{12}}$ caracteriza
%% a afinação de temperamento por igual. Há afinações com intervalos propostos como razões de inteiros de baixa ordem, com
%% fruto da observação de comportamentos físicos.  As primeiras destas afinações foram formalizadas por volta de 2 mil anos antes do advento do temperamento por igual.\cite{Roederer}

%% Duas afinações emblemáticas são:
%% \begin{itemize}
%%     \item A {\bf intonação justa} consiste em notas da escala diatônica associadas a razões de inteiros de pequena ordem como apontados pela série harmônica. As razões básicas estão contidas no modo jônico (dó a dó nas teclas brancas do piano, veja abaixo na subseção~\ref{subsec:escalas}): 1, 9/8, 5/4, 4/3, 3/2, 5/3, 15/8, 2/1. Os intervalos são considerados com relação às notas das escala e são usados o semitom 16/15, o 'tom menor' 10/9 e o 'tom maior' 9/8. Há diferentes formas de realizar uma divisão de 12 notas.
%%     \item A {\bf afinação pitagórica} baseia-se no intervalo 3/2 (quinta justa). O modo jônico fica: 1, 9/8, 81/64, 4/3, 3/2, 27/16, 243/128, 2/1. Os intervalos são também considerados com relação às notas da escala. Além dos intervalos do modo, são usados a segunda menor 256/243, a terça menor 32/27, a quarta aumentada 729/512, a quinta diminuta 1024/729, a sexta menor 128/81 e a sétima menor 16/9. 
%% \end{itemize}

%% Para a realização de microtonalidade\footnote{O uso de
%% intervalos menores que o semitom é chamado microtonalidade e tem usos
%% ornamentais e estruturantes da música. A
%% divisão da oitava em $12$ notas possui fundamentos físicos mas
%% não deixa de ser uma \emph{convenção}
%% adotada inclusive pela música erudita clássica de origem europeia. Outras afinações
%% são incidentes. Para citar somente um exemplo, a
%% música tradicional tailandesa utiliza uma divisão da oitava em sete notas igualmente
%% espaçadas ($\varepsilon=2^{\frac{1}{7}}$),
%% resultando em intervalos que pouco se assemelham aos intervalos na
%% divisão de doze notas.\cite{Wisnick}},
%% pode-se usar reais não inteiros
%% para a sequência de alturas, ou modificar o fator $\varepsilon=2^{\frac{1}{12}}$
%% e continuar usando inteiros. Por exemplo, uma afinação
%% bastante próxima da série harmônica em si
%% é proposta na forma da divisão da oitava em $53$ notas:
%% $\varepsilon_2=2^{\frac{1}{53}}$.\cite{microtonalidade}
%% As notas nesta divisão da oitava em $53$ notas se relacionam por inteiros
%% com $\varepsilon_2$.
%% Note que se $S_i$ é uma sequência de alturas relacionadas por $\varepsilon_1$,
%% um mapeamento para notas relacionadas por $\varepsilon_2$
%% constitui uma nova sequência $S_i'=\left\{s_i'\right\}=\left\{ s_i \frac{\varepsilon_1}{\varepsilon_2}\right\}$. A montagem musical \emph{Micro tom} explora recursos microtonais e seu código está no Apêndice~\ref{ap:micro}, assim como na \massa\ online.



%% \subsubsection{Intervalos}\label{subsec:intervalos}
%% Em proporções de $\varepsilon=2^{\frac{1}{12}}$ entre as frequências das notas (i.e. um semitom), os intervalos do sistema de 12 notas são representados por inteiros. A tabela~\ref{eq:intervalos} resume as características de cada intervalo: sua nomenclatura tradicional, características de consonância e dissonância e número de semitons de cada um.

%% \begin{table}[htpq!]
%% \centering
%% \caption{Intervalos musicais, suas notações tradicionais, classificações básicas de dissonância e número de semitons.
%% As consonâncias perfeitas são os uníssonos, as quintas e as oitavas justas (J). As consonâncias imperfeitas são
%% as terças e as sextas maiores (M) e menores (m). As dissonâncias fortes são as segundas menores e sétimas maiores. As dissonâncias
%% brandas são as segundas maiores e as sétimas menores. O primeiro caso especial consiste na quarta justa, que é consonante perfeita
%% se considerada uma inversão da quinta justa, caso contrário pode ser considerada uma dissonância ou uma consonância imperfeita. O segundo caso especial é o trítono (4aum, 5dim, tri). Este é consonante em algumas culturas. Já para a música tonal, o trítono indica dominante e busca sua resolução em uma terça ou sexta e, por esta instabilidade, é considerado intervalo dissonante.}
%% \begin{tabular}{| c | c | c | }\hline
%%     \multicolumn{3}{|c|}{\bf consonâncias}  \\\hline
%%    & notação tradicional & número de semitons \\
%%    perfeitas: & 1J, 5J, 8J & 0, 7, 12 \\
%%  imperfeitas: & 3m, 3M, 6m, 6M & 3, 4, 8, 9 \\\hline\hline
%%     \multicolumn{3}{|c|}{\bf dissonâncias} \\\hline
%%  & notação tradicional & número de semitons \\
%%  fortes: & 2m, 7M & 1, 11 \\
%%  brandas: & 2M, 7m & 2, 10 \\\hline\hline
%%     \multicolumn{3}{|c|}{\bf casos especiais} \\\hline
%%  & notação tradicional & número de semitons \\
%%  consonante ou dissonante: & 4J & 5 \\
%%  dissonante na tradição ocidental: & trítono, 4aum, 5dim & 6 \\\hline
%% \end{tabular}\label{eq:intervalos}
%% \end{table}


%% A nomenclatura, com base em imposições e conveniências do sistema tonal, e de aspectos práticos da manipulação de notas, pode ser especificada assim:\cite{Roederer,Wisnick}
%% \begin{itemize}
%%         \item Intervalos por número de grados entre as notas: primeira (uníssono), segunda, terça, quarta, quinta, sexta, sétima, oitava. Nona, décima, décima primeira, etc, são os intervalos compostos, de uma ou mais oitavas + um intervalo dentro da oitava, que caracteriza o intervalo composto. Os intervalos são representados pelos dígitos numéricos, e.g. 1, 3, 5 é um uníssono, uma terça e uma quinta.
%%         \item Qualidades de cada intervalo: as consonâncias perfeitas - i.e. uníssono, quarta, quinta e oitava - são 'justas'. As consonâncias imperfeitas - i.e. terças e sextas - e as dissonâncias - i.e. segundas e sétimas - podem ser maiores ou menores. Exceção para o trítono.
%%         \item A quarta justa é tida como consonante perfeita ou dissonante de acordo com o contexto e arcabouço teórico. Como regra geral, pode ser considerada consonante salvo casos em que é usada de passagem para uma quinta ou terça como resolução.
%%         \item O trítono é dissonante na música ocidental por caracterizar a "dominante" no sistema tonal (veja abaixo na subseção~\ref{subsec:harmonia}) e representar instabilidade. Algumas culturas entoam o intervalo como consonante.
%%         \item Um intervalo maior, decrescido de um semitom, resulta em um intervalo menor. Um intervalo menor, acrescido de um semitom, resulta em um intervalo maior.
%%         \item Um intervalo justo (uníssono, quarta justa, quinta justa, oitava justa), ou um intervalo maior (segunda maior 2M, terça maior 3M, sexta maior 6M ou sétima maior 7M), acrescido de um semitom, resulta em um intervalo aumentado (p.ex. terça aumentada 3aum com cinco semitons). A quarta aumentada é também chamada de trítono (4aum ~ tri).
%%         \item Um intervalo justo, ou um intervalo menor (segunda menor 2m, terça menor 3m, sexta menor 6m ou sétima menor 7m), decrescido de um semitom, resulta em um intervalo diminuto. A quinta diminuta é também chamada de trítono (5dim ~ tri).
%%         \item Um intervalo aumentado, acrescido de um semitom, resulta em um intervalo 'mais que aumentado' e um intervalo diminuto, decrescido de um semitom, resulta em um intervalo 'mais que diminuto'.
%%         \item Caso as notas soem simultaneamente, o intervalo é harmônico.
%%         \item Caso as notas soem em sequência no tempo, o intervalo é melódico. A ordem das notas, primeiro a nota mais grave ou a mais aguda, resulta em um intervalo ascendente ou descendente, respectivamente. 
%%         \item Passada a nota mais grave para a oitava acima, ou a nota mais aguda uma oitava para baixo, o intervalo é invertido. Um intervalo, somado à sua inversão, resulta 9 (7m inverte para 2M: $7m+2M=9-$). Um intervalo maior invertido resulta em um intervalo menor e vice-versa. Um intervalo aumentado invertido resulta diminuto e vice-versa, assim como o mais que aumentado resulta em mais que diminuto e vice-versa. Um intervalo justo invertido resulta igualmente justo.
%%         \item Um intervalo maior que a oitava é dito 'intervalo composto' e é classificado como o intervalo entre as mesmas notas, mas na mesma oitava. São também especificados por 7 acrescido deste intervalo: 11J é uma oitava mais uma quarta (7+4J==11J), 9M é uma oitava mais uma segunda maior (7+2M=9M).
%% \end{itemize}

%% Os intervalos aumentados/diminutos e mais que aumentados/diminutos são consequências do sistema tonal. Os graus da escala (veja abaixo na subseção~\ref{subsec:escalas}) são realmente notas diferentes, com funções e usos específicos. Assim, em uma escala de dó bemol maior, a tônica - primeiro grau - é dó bemol, não si, e a sensível - sétimo grau - é si bemol, não lá sustenido ou dó duplo bemol. De forma semelhante, o segundo grau de uma escala pode estar a um semitom do primeiro grau, assim como a sensível (sétimo grau a um semitom ascendente do primeiro grau), momento no qual há uma terça diminuta entre os dois semitons do sétimo e segundo grau da escala pois o primeiro grau está entre os dois graus próximos a ele: segundo e sensível.\cite{Lacerda}

%% Esta descrição resume a teoria tradicional dos intervalos musicais.\cite{Lacerda} A montagem \emph{Intervalos entre alturas} explora estes intervalos de formas isoladas e diversas. O código está no Apêndice~\ref{ap:intervalos} e disponível online com a \emph{toolbox} \massa.\cite{MASSA}


%% \subsubsection{Escalas}\label{subsec:escalas}
%% Uma escala é um conjunto ordenado de alturas. Usualmente, as escalas se repetem
%% a cada oitava. A sequência ascendente com todas as notas da divisão da oitava em 12 intervalos iguais,
%% separados pela razão $\varepsilon=2^{\frac{1}{12}}$, é a escala cromática de temperamento por igual. Há 5 divisões perfeitamente simétricas da oitava dentro da escala cromática. Estas divisões são consideradas
%% escalas pelos usos fáceis e peculiares que disso provém. Como inteiros aos quais $\varepsilon=2^{\frac{1}{12}}$ é elevado
%% para multiplicar $f_0$, as escalas são:

%% \begin{equation}\label{escSim}
%% \begin{split}
%% \text{cromática} & = E_i^c = \{e_i^c\}_0^{11} =  \{0,1,2,3,4,5,6,7,8,9,10,11\} = \{i\}_0^{11}\\
%% \text{tons inteiros} & = E_i^t = \{e_i^t\}_0^{5} = \{0,2,4,6,8,10\} = \{2.i\}_0^{5} \\
%% \text{terças menores} & = E_i^{tm} = \{e_i^{tm}\}_0^{3} = \{0,3,6,9\} = \{3.i\}_0^3 \\
%% \text{terças maiores} & = E_i^{tM} = \{e_i^{tM}\}_0^{2} = \{0,4,8\} = \{4.i\}_0^2\\
%% \text{trítonos} & = E_i^{tt} = \{e_i^{tt}\}_0^{1} = \{ 0, 6 \} = \{6.i\}_0^1
%% \end{split}
%% \end{equation}

%% Assim, a terceira nota da escala em tons inteiros com $f_0=200Hz$
%% é $f_3=\varepsilon^{e_3^t} . f_0 = 2^{\frac{6}{12}} . 200 \approxeq 282.843 Hz$. Estas
%% 'escalas' ou padrões geram estruturas estáveis pelas simetrias internas, e podem ser
%% repetidas de forma eficiente e sustentada. Abaixo, na seção~\ref{estCic}, será retomado o assunto das simetrias. A montagem \emph{Cristais} expõe cada uma destas escalas tanto melódica quanto harmonicamente e seu código está no Apêndice~\ref{ap:cristais}. Como parte da \massa, este código é disponibilizado online.

%% As escalas diatônicas:

%% \begin{equation}\label{eq:escalas}
%% \begin{split}
%% \text{menor natural} = \text{modo eólico} & = E_i^m = \{e_i^m\}_0^6 = \{0,2,3,5,7,8,10\} \\
%% \text{modo lócrio} & = E_i^{mlo} = \{e_i^{mlo}\}_0^6 = \{0,1,3,5,6,8,10\} \\ 
%% \text{maior}  = \text{modo jônico} & = E_i^M = \{e_i^M\}_0^6 = \{0,2,4,5,7,9,11\} \\
%% \text{modo dórico} & = E_i^{md} = \{e_i^{md}\}_0^6 = \{0,2,3,5,7,9,10\} \\
%% \text{modo frígio} & = E_i^{mf} = \{e_i^{mf}\}_0^6 = \{0,1,3,5,7,8,10\} \\
%% \text{modo lídio} & = E_i^{ml}=\{e_i^{ml}\}_0^6 = \{0,2,4,6,7,9,11\} \\
%% \text{modo mixolídio} & = E_i^{mmi} = \{e_i^{mmi}\}_0^6 = \{0,2,4,5,7,9,10\}
%% \end{split}
%% \end{equation}

%% possuem apenas intervalos maiores, menores e justos. Única exceção para o trítono, que se apresenta como quarta aumenta ou quinta diminuta.

%% Todas as escalas diatônicas
%% seguem o padrão de intervalos sucessivos
%% tom, tom, semitom, tom, tom, tom, semitom, e
%% pode-se escrever:
%% \begin{equation}\label{eq:relacaoDia}
%% \begin{split}
%% \{d_i\} & =\{2,2,1,2,2,2,1\} \\
%% e_0 & =0 \\
%% e_i & =d_{(i+\kappa)\%7}+e_{i-1} \quad para \;\;  i > 0
%% \end{split}
%% \end{equation}

%% Com $\kappa \in \mathbb{N}$. Para cada
%% modo, há um único valor de $\kappa \in [0,6]$.
%% Por exemplo, uma breve
%% inspeção revela que $e_i^{ml}=d_{(i+2)\%7}+e_{i-1}^{ml}$. Portanto, $\kappa=2$
%% para o modo lídio. 

%% A escala menor possui duas formas adicionais, melódica e harmônica:

%% \begin{equation}\label{eq:escalasMenores}
%% \begin{split}
%% \text{menor natural (igual acima)} & = E_i^m = \{e_i^m\}_0^6 = \{0,2,3,5,7,8,10\} \\
%% \text{menor harmônica} & = E_i^{mh} = \{e_i^{mh}\}_0^6 = \{0,2,3,5,7,8,11\} \\
%% \text{menor melódica} & = E_i^{mm} = \{e_i^{mm}\}_0^{14} = \{0,2,3,5,7,9,11,12,10,8,7,5,3,2,0\} \\
%% \end{split}
%% \end{equation}

%% O traçado ascendente e descendente da escala menor melódica é necessário para a presença da sensível (sétimo e último grau, separado por um semitom da oitava, realça a polarização na tônica) no trajeto ascendente, o que não é necessário quando descende, retomando a forma natural. Já a escala harmônica apresenta a sensível, mas não evita o intervalo de segunda aumentada entre o sexto e sétimo graus, pois não precisa contemplar trajetória melódica, apenas apresentar a sensível tão cara ao sistema tonal (a sensível tende à tônica, afirmando-a).\cite{Harmonia}
%% Outras escalas podem ser representadas da mesma forma, como as pentatônicas e os modos de transposição limitados de Messiaen.\cite{Messiaen}

%% \subsubsection{Acordes}\label{subsec:acordes}
%% A ocorrência simultânea de notas é observada através dos acordes. Destes, a base na música tonal são as tríades. Estas constituem-se de duas terças sucessivas, em 3 notas: fundamental, terça e quinta. Um acorde invertido é aquele que apresenta no grave outra nota que não a fundamental. A posição fechada é aquela em que não cabe nota alguma do acorde entre quaisquer duas notas consecutivas.\cite{Lacerda}
%% As tríades, sem inversão, na forma fechada e com a fundamental em $0$, são:

%% \begin{equation}\label{triades}
%% \begin{split}
%% \text{tríade maior} = A_i^M= \{a_i^M\}_0^2=\{0,4,7\} \\ 
%% \text{tríade menor} = A_i^m = \{a_i^m\}_0^2=\{0,3,7\} \\
%% \text{tríade diminuta} = A_i^d = \{a_i^d\}_0^2=\{0,3,6\} \\
%% \text{tríade aumentada} = A_i^a = \{a_i^a\}_0^2=\{0,4,8\}
%% \end{split}
%% \end{equation}

%% Para considerar outra terça sobreposta à quinta, basta acrescentar $10$ ao final para a tétrade
%% com sétima menor ou $11$ para a tétrade com sétima maior. As inversões e posições abertas
%% podem ser obtidas com a adição seletiva de $12$ às componentes.

%% Acordes triádicos incompletos, com notas adicionais (acordes 'sujos'), e não triádicos são comuns.
%% Orientações gerais são:
%% \begin{itemize}
%%     \item Uma quinta constitui uma fundamental confirmada pelo intervalo.
%%     \item A terça maior ou menor aponta a qualidade maior ou menor do acorde.
%%     \item Todo trítono, especialmente se formado entre uma terça maior e uma sétima menor, tende a resolver em uma terça ou uma sexta.
%%     \item Evita-se a duplicação de nota. Na necessidade de duplicação, a ordem de preferência é: a fundamental, a quinta, a terça e a sétima.
%%     \item Pode-se formar acordes com notas diferentes das triádicas, especialmente se dentro de alguma lógica recorrente ou encadeamento musical que justifique as notas diferentes.
%%     \item Acordes formados por sucessões de intervalos diferentes da terças, como as quartas ou as segundas, são recorrentes dentre composições de tonalismo avançado ou experimentais.
%%     \item A repetição de sucessões de acordes (ou suas características) fixa uma trajetória pela recorrência e permite a introdução de formações exóticas sem que haja incoerência.
%% \end{itemize}


%% \subsection{Harmonias atonal, tonal, expansão e modulação}\label{subsec:harmonia}
%% A omissão dos encadeamentos básicos do sistema tonal é chave para a obtenção de 
%% harmonias modal e atonal. No caso desta ausência de estruturas tonais mínimas,
%% se as notas coincidirem com alguma escala diatônica (veja as equações ~\ref{eq:escalas})
%% ou forem em número pequeno, pode-se dizer que a
%% harmonia é modal. Caso encadeamentos tonais básicos estejam ausentes, as notas não coincidam
%% com alguma das escalas diatônicas e forem diversas e dissonantes (com relação às outras notas) o suficiente para evitar redução por
%% polarizações, a harmonia é atonal.
%% Nesta classificação, a harmonia modal não é tonal e não é atonal.
%% A harmonia modal está reduzida à incidência de notas dentro de escalas diatônicas (ou simplificações) e à ausência de estruturas tonais.
%% Pode-se perceber, pelo conceito, que a harmonia atonal é difícil de ser realizada.\cite{harmEXT}

%% \subsubsection{Harmonia atonal}
%% De fato, as técnicas de música atonal
%% visam estruturas que evitam o vínculo da audição a modos e relações tonais. A
%% dificuldade é tamanha que o dodecafonismo surgiu. A proposta do dodecafonismo é
%% usar um conjunto de notas, idealmente usa-se as 12 notas, e executar uma a uma destas notas
%% na mesma ordem. Neste contexto, a tônica fica difícil de se estabelecer. Mesmo assim, a audição ocidental
%% procura traços tonais nas música e facilmente os encontra por caminhos inesperados e por vezes tortuosos.
%% A utilização de
%% intervalos dissonantes, especialmente trítonos sem resoluções, nonas, segundas e sétimas, reforça
%% a ausência de tonalidade. Neste contexto, para criação da peça, pode-se:
%% \begin{itemize}
%%     \item Repetir notas. Ao considerar a repetição imediata um prolongamento da incidência anterior, notas iguais, sem outras notas entre elas, não adicionam informação relevante.
%%     \item Soar notas adjacentes ao mesmo tempo, formando intervalos harmônicos e acordes.
%%     \item Apresentar durações livremente, desde que respeitada a ordem de aparição das notas.
%%     \item Para variação, além dos recursos de ampliação, transposição e translação, são usados o inverso, o retrógrado e o retrógrado do inverso. Veja nas subseções~\ref{subsec:motivos} e~\ref{subsec:usosmusicais3} para maiores detalhes.
%%     \item Variações de instrumentação, articulação, espacialização e outras possibilidades de apresentação da estrutura de notas.
%% \end{itemize}

%% A harmonia atonal pode ser observada, de forma paradigmática, dentro destas condições.
%% Vale apontar que boa parte do que escreveram os compositores
%% emblemáticos para o dodecafonismo, como Alban Berg e o próprio Schoenberg, consiste de
%% aplicações parciais e livres destas técnicas. Várias peças fazem até mesmo pontes
%% propositais entre técnicas tonais e atonais.

%% \subsubsection{Harmonia tonal}
%% No século XX, músicas rítmicas
%% e com ênfase em sonoridades/timbres ampliaram 
%% as concepções de tonalidade e harmonia. Ainda assim, a harmonia tonal tem forte presença
%% nas vertentes artísticas e comerciais. Considera-se o próprio dodecafonismo de natureza
%% tonal pois consiste na negação das características tonais de polarização.

%% Na música tonal ou modal, acordes, como os listados nas equações~\ref{triades}, constituídos com a fundamental em cada
%% grau de uma escala, como das equações~\ref{eq:escalas}, formam os pilares do campo harmônico.
%% A observação de progressões incidentes de acordes e regras para encadeamentos é o objeto de estudo da harmonia musical.
%% Mesmo uma melodia monofônica gera campos harmônicos e pode-se observar acordes sugeridos por cada passagem.



%% Na 'harmonia tonal tradicional',
%% a escala pode ter como tônica (primeiro grau da escala) qualquer nota e pode ser maior (com as mesmas notas do modo jônico) ou menor (as notas do eólico constituem a escala "menor natural" e esta possui versões harmônica e melódica, veja a equação~\ref{eq:escalasMenores}). A escala escolhida é
%% base para tríades, cada uma com a fundamental em um grau 
%% da escala: $\hat{1},\hat{2},\hat{3},\hat{4},\hat{5},\hat{6},\hat{7}$. Para a formação do acorde, são consideradas a terceira nota e a quinta nota acima da fundamental, considerando a própria nota como a primeira e utilizando as notas da escala.
%% Pode-se anotar $\hat{1},\hat{3},\hat{5}$ para o acorde do primeiro grau, formado pelo primeiro grau da escala e central para uma música tonal. Secundários são os acordes do quinto grau $\hat{5},\hat{7},\hat{2}$ ($\hat{7}$ sustenido no caso da escala menor) e do quarto grau $\hat{4},\hat{6},\hat{1}$. Depois são considerados os outros graus. A harmonia tradicional consiste em convenções e técnicas estilísticas de encadeamento destes acordes, formados em cada grau da escala.\cite{Harmonia}

%% A 'harmonia funcional' atribui funções a estes três acordes centrais e busca compreender seus usos através destas funções. O acorde formado sobre o primeiro grau é o acorde de tônica (T ou t se tônica maior ou menor) e tem a função de manter um centro, um "chão" na música. O acorde formado sobre o quinto grau é a dominante (D, a dominante é sempre maior) e tem a função de tender à tônica, direcionar a música para ela. A tríade formada sobre o quarto grau é a subdominante (S ou s se subdominante maior ou menor) e tem a função de distanciar a música da tônica. O sistema se baseia em afirmar a tônica através de encadeamentos tônica-dominante-tônica expandidos com outros acordes das formas mais diversas.

%% A estes três acordes, são associadas as outras tríades. Na escala maior, a associada relativa (tônica relativa Tr, subdominante relativa Sr e dominante relativa Dr) é a tríade formada uma terça abaixo e a associada anti-relativa (tônica anti-relativa Ta, subdominante anti-relativa Sa e a dominante anti-relativa Da) é a tríade formada na terça acima. Na escala menor ocorre o mesmo, mas a tríade a uma terça abaixo é chamada anti-relativa (tA, sA) e a tríade a uma terça acima é chamada de relativa (tR, sR). As exatas funções e efeitos musicais destes acordes é motivo de bastante controvérsia. A tabela~\ref{tab:harmonia} mostra a relação entre as tríades formadas em cada grau da escala maior.

%% \begin{table}[htpq!]
%% \centering
%% \caption{Resumo das funções harmônicas tonais para a escala maior. A tônica é o centro da música, a dominante tende à tônica e a subdominante se distancia da tônica. Os três acordes podem, a princípio, serem substituídos livremente pelas respectivas relativas ou anti-relativas.}
%% \begin{tabular}{l | c | r}
%% relativa & acorde principal da função & anti-relativa \\\hline\hline
%% $\hat{6},\hat{1},\hat{3}$ & tônica:       $\hat{1},\hat{3},\hat{5}$ & $\hat{3}, \hat{5},      \hat{7}$ \\
%% $\hat{3},\hat{5},\hat{7}$ & dominante:    $\hat{5},\hat{7},\hat{2}$ & [ $\hat{7},\hat{2},\hat{4}\#$ ] \\
%% $\hat{2},\hat{4},\hat{6}$ & subdominante: $\hat{4},\hat{6},\hat{1}$ & $\hat{6},\hat{1},       \hat{3}$
%% \end{tabular}
%% \label{tab:harmonia}
%% \end{table}

%% É necessário que a dominante anti-relativa forme um acorde menor, por isso a alteração do quarto grau um semitom para cima $\hat{7}\#$. O acorde diminuto $\hat{7},\hat{2},\hat{4}$, geralmente é considerado uma 'dominante com sétima e sem fundamental'.\cite{Koellheuteur}
%% Em modo
%% menor, há a alteração do $\hat{7}$ por um semitom ascendente para que haja somente um semitom de separação com a tônica,
%% permitindo a dominante (que deve ser maior e tender à tônica). Assim, a dominante é sempre maior, tanto em escalas maior ou menor e, por este motivo, mesmo em tom menor, a dominante relativa permanece a uma terça abaixo e a antirrelativa a uma terça acima.

%% \subsubsection{Expansão tonal: funções individuais e medianas cromáticas}
%% Cada um destes acordes pode ser confirmado e se desenvolver com uma execução de sua dominante ou subdominante individual, que é o acorde com base na tríade formada a uma quinta acima ou uma quinta abaixo respectivamente. Estas dominantes/subdominantes individuais, por sua vez, possuem também subdominantes e dominantes individuais passíveis de uso. Assim, em uma dada tonalidade, pode ocorrer qualquer acorde, por mais distante que seja do campo harmônico e das notas da escala, desde que a ocorrência apresente um percurso coerente de dominantes e subdominantes até a tonalidade de origem.

%% As medianas, ou 'medianas cromáticas', são duas para cada acorde: a mediana cromática superior, formada com a fundamental na terça do acorde original, e a inferior, formada com a quinta na terça do acorde original. São acordes formados também a uma terça, mas com uma alteração cromática com relação ao acorde de origem. Caso haja duas alterações cromáticas, i.e. duas notas alteradas por um semitom cada com relação ao acorde original, a mediana é chamada 'duplamente cromática'. Também são duas para cada acorde: a superior, com terça na quinta do acorde original, e a inferior, com terça na fundamental da tríade original. Observe que um acorde maior possui medianas maiores e medianas duplamente cromáticas maiores. Um acorde menor possui medianas menores e medianas duplamente cromáticas menores. Esta relação entre acordes é considerada de tonalismo avançado, por vezes até de expansão e dissolução do tonalismo, e tem efeitos fortes e marcantes embora perfeitamente consonantes. As medianas foram utilizadas a partir do final do romantismo por Wagner, Lizt, Richard Strauss dentre outros e são bastante simples de serem realizadas.\cite{Harmonia,Salzer}

%% \subsubsection{Modulação}
%% A modulação é a mudança da tonalidade em que se encontra a musica. Caracteriza-se a modulação através da observação
%% das tonalidades de partida e chegada e da forma de transição. As tonalidades são sempre tidas como relacionadas por quintas e suas relativas e antirrelativas. São formas de efetuar a modulação:
%% \begin{itemize}
%%     \item A transposição do discurso para a nova tonalidade, sem preparação alguma. É procedimento típico do barroco embora seja incidente em outros períodos. Por vezes chamada de modulação frasal ou modulação sem preparação.
%%     \item O uso cuidadoso de uma dominante individual, e possivelmente também a subdominante individual, para afirmar a mudança da tônica e campo harmônico.
%%     \item Uso de alterações cromáticas para atingir um acorde da nova tonalidade a partir de algum acorde da tonalidade anterior. Chamada de modulação cromática.
%%     \item O destaque para uma única nota, possivelmente repetida ou suspensa sem acompanhamento, comum às tonalidades de saída e chegada, constitui uma forma peculiar de introduzir o novo campo harmônico.
%%     \item Mudança da função, sem a modificação das notas em si, de um acorde para contemplar nova tonalidade. Procedimento chamado de enarmonia.
%%     \item A manutenção do centro tonal e mudança da qualidade maior para menor (ou vice-versa) da tonalidade é a modulação paralela. A tonalidade de mesma tônica e outra qualidade é chamada homônima.
%% \end{itemize}

%% A importância da dominante a torna pivô natural das modulações, o que desemboca no círculo das quintas.\cite{Harmonia,Salzer,Koellheuteur,Harmony} A montagem musical \emph{Acorde cedo} explora estas relações entre acordes. Seu código está disponível do Apêndice~\ref{ap:acorde} e online como parte da \massa.\cite{MASSA}


%% \subsection{Contraponto}\label{subsec:contraponto}

%% A condução de linhas melódicas simultâneas, chamadas vozes,
%% é o contraponto. A bibliografia
%% percorre formas sistemáticas de condução de vozes e desemboca em gêneros escolásticos como cânones, invenções e fugas. É possível resumir regras principais do contraponto e é dito que o próprio Beethoven - dentre outros - esboçou uma síntese deste tipo.

%% \begin{figure}[h!]
%%     \centering
%%         \includegraphics[width=\textwidth]{figuras/movContraponto}
%%     \caption{Movimentos diferenciados pelo contraponto com vistas a preservar a independência entre as vozes. 3 tipos de movimentos: direto, contrário e oblíquo, categorizam as possibilidades. O movimento paralelo é um tipo de movimento direto.}
%%         \label{fig:movContraponto}
%% \end{figure}

%% O propósito do contraponto é conduzir as vozes 
%% de forma que soem independentes. Cruciais para isso
%% são as movimentações relativas, das vozes duas a duas,
%% categorizadas em: movimento direto, oblíquo e contrário
%% conforme a figura~\ref{fig:movContraponto}. O movimento paralelo
%% é um movimento oblíquo.
%% A regra de ouro é cuidar que os movimentos diretos
%% não terminem em consonância perfeita. O movimento paralelo
%% deve só ocorrer entre consonâncias imperfeitas e não mais
%% do que três vezes consecutivas. As dissonâncias podem
%% ser não admitidas ou usadas seguidas 
%% e precedidas de consonâncias em graus conjuntos, i.e. notas vizinhas na escala.
%% Os movimentos que levam a nota a uma vizinha soam mais coerentes.
%% Na presença de 3 ou mais vozes,
%% a importância melódica recai sobre as vozes mais aguda e mais grave, nesta ordem.\cite{Fux,Tragtenberg,SchoenbergContra}

%% Estas regras foram usadas na montagem \emph{Conta ponto}, o código está no Apêndice~\ref{ap:conta} e disponível online junto à \massa.


%% \subsection{Ritmo}\label{subsec:ritmo}
%% A noção rítmica é dependente de eventos separados por durações.\cite{Lacerda} Estes eventos podem ser ouvidos individualmente se espaçados
%% por ao menos $50-63ms$. Para que a separação temporal entre eles possa ser apreciada como duração, ela deve ser maior,
%% por volta de $100ms$.\cite{microsound} Pode-se sumarizar 
%% a transição
%% de durações ouvidas como alturas para a apreciação em ritmo da seguinte forma:\cite{Alfaix, microsound}

%% \begin{table}[htpq!]
%% \tiny
%% \centering
%% \caption{Transição das durações ouvidas individualmente para alturas.}
%% \begin{tabular}{  l | r r r r   r r r    r r r || r r || r r r r r r }
%% \hline
%%            & \multicolumn{10}{c}{$\underleftarrow{\text{\bf zona de percepção de durações em ritmo}}$} & \multicolumn{2}{c}{transição} & \multicolumn{3}{c}{-} \\
%% duração (s) & {\bf ...}     & {\bf 32,}     & {\bf 16,}   & {\bf 8,}  & {\bf 4,}   & {\bf 2,}   & {\bf 1,}   & {\bf 1/2,} & {\bf 1/4,} & {\bf 1/8,} & $\frac{1}{16}=62,5ms$ , & $\frac{1}{20}=50ms$ & {\color{Gray} 1/40} & {\color{Gray} 1/80  } & {\color{Gray} 1/160 } & {\color{Gray} 1/320 } & {\color{Gray} 1/640 } & {\color{Gray} ... } \\
%% frequência (Hz) & {\color{Gray} ...} & {\color{Gray} 1/32,}   & {\color{Gray} 1/16,} & {\color{Gray} 1/8,} & {\color{Gray} 1/4,} & {\color{Gray} 1/2,} &  {\color{Gray} 1,}  & {\color{Gray} 2,}   & {\color{Gray} 4,}   & {\color{Gray} 8,}    & 16,  & 20   & {\bf 40}   & {\bf 80}   & {\bf 160}   & {\bf 320}   & {\bf 640}   & {\bf ...} \\
%%            & \multicolumn{10}{c}{ - } & \multicolumn{2}{c}{transição} & \multicolumn{6}{c}{$\overrightarrow{\text{\bf zona de percepção de durações em altura}}$} \\
%% \hline
%% \end{tabular}
%% \label{tab:duracoes}
%% \end{table}

%% A banda de durações marcada como transição está minimizada pois os limites não são bem definidos: a duração em que se começa a perceber uma frequência fundamental ou uma separação entre as ocorrências é dependente da pessoa e de características do som.\cite{microsound,Roederer}

%% A métrica rítmica costuma se basear em uma duração básica chamada pulso. O pulso tipicamente
%% compreende durações entre $0.25-1.5s$ (respectivamente $240$ e $40BPM$). Na educação musical e estudos cognitivistas,
%% costuma-se associar esta gama de frequências de pulsação às durações entre batidas 
%% do coração, da inspiração/expiração ou entre os passos ao caminhar.\cite{Lacerda,Roederer}

%% O pulso é subdividido em partes iguais e também é 
%% repetido sequencialmente. Estas relações (de divisão e de concatenação) costumam
%% seguir relações de números inteiros de baixa ordem
%% \footnote{Em ordem crescente de ocorrência na música
%% escrita e étnica,
%% as divisões do pulso musical e seus agrupamentos
%% sequenciais no tempo são: 2, 4 e 8, depois 3, 6 (dois grupos de 3 ou 3 grupos de 2) e 9 e 12 (3 e 4 grupos de 3). Por 
%% último os primos 5 e 7, completando 1-9 e 12.
%% Outras métricas são menos usuais, como divisões ou agrupamentos em 13, 17, etc, e são incidentes principalmente em contextos de música experimental e erudita do século XX e XXI. Por mais complexas que pareçam, as métricas costumam ser composições e decomposições de 1-9 partes iguais.\cite{Gramani,Roederer}.}.
%% Uma exposição esquemática está na figura~\ref{fig:pulsoSubAgl}.

%% \begin{figure}[h!]
%%     \centering
%%         \includegraphics[width=\textwidth]{figuras/metricaMusical}
%%     \caption{Divisões e aglomerações do pulso musical para estabelecimento de métrica. Ao lado esquerdo estão as divisões da semínima estabelecida como pulso. Ao lado direito, fórmulas de compasso que especificam as mesmas métricas, mas na escala das aglomerações do pulso musical.}
%%         \label{fig:pulsoSubAgl}
%% \end{figure}

%% As relações duais (compassos simples e divisões binárias) costumam ocorrer em ritmos de dança
%% e ocasiões festivas, e são chamadas imperfeitas. As relações ternárias
%% incidem mais na música ritualística e relacionada ao sagrado
%% e são ditas perfeitas.

%% As unidades mais fortes (acentuadas) são as que consistem nas 'cabeças das divisões'. A cabeça de uma unidade é a primeira parte da subdivisão. Nas divisões binárias (2, 4 e 8 dos casos considerados),
%% as unidades consideradas fortes se revezam com as fracas
%% (e.g. a divisão em 4 é forte, fraco, meio-forte, fraco).
%% Nas divisões ternárias (3, 6 e 9)
%% à unidade forte (primeira) se sucedem 2 unidades fracas (e.g. a divisão em 3 é forte, fraco fraco)\footnote{A divisão em 6 é considerada composta
%%  mas pode ocorrer também como uma divisão binária.
%%  Uma divisão binária que sofre então uma divisão ternária
%%  resulta em duas unidades divididas em três unidades cada: forte (subdividido em forte fraco, fraco) e fraco (subdividido em forte, fraco, fraco).
%% A outra forma de ocorrer a divisão em 6 é através de 
%% uma divisão ternária que sofre então uma divisão binária, resultando em:
%% uma unidade forte (subdividido em forte e fraco) e duas unidades fracas (subdivididas em forte e fraco cada).}.

%% A acentuação em tempo fraco é o contratempo, uma nota iniciada em tempo fraco e cuja duração se prolonga sobre um tempo forte é uma síncopa.

%% As notas podem ocorrer dentro e fora destas divisões da \emph{'métrica musical'}. Nos casos mais comportados, as notas ocorrem exatamente nestas divisões, com maior incidência em ataques nos tempos fortes.
%% Em casos extremos, não se pode perceber a métrica.\cite{Roederer} Variações pequenas na grade ajudam a compor a interpretação musical e diferenças entre estilos.\cite{Cook}

%% Seja o pulso o nível $j=0$ de agrupamento, o nível $j=-1$ 
%% a primeira subdivisão do pulso, o nível $j=1$ a primeira aglomeração dos pulsos e assim por diante. 
%% Desta forma, $P_i^j$ é a $i$-ésima unidade de 
%% pulsos no nível $j$ de agrupamento:
%% $P^0_{10}$ é o décimo pulso, $P^{1}_3$ é a terceira unidade de agrupamento de pulsos (é possível que seja o terceiro compasso),
%% $P^{-1}_2$ é a segunda unidade da subdivisão do pulso.

%% Especial atenção para
%% os limites de $j$: as divisões do pulso são durações apreciáveis
%% como ritmo; além disso, as junções do pulso somam, no máximo
%% do escopo, uma música ou um conjunto coeso de músicas. Dito de outra forma: a duração de $P^{min(j)}_i$, $\forall \; i$,
%% é maior que $50ms$ e as durações somadas $\sum_{\forall i}P^{\text{máx}(j)}_i$
%% é menor do que alguns minutos ou, no máximo, poucas horas.


%% Cada nível $j$ possui alguns índices $i$. Sempre que o índice $i$ possuir três valores diferentes
%% (ou múltiplo de três), há uma relação perfeita. 
%% Quando $i$ possuir somente múltiplo dois, quatro ou oito valores diferentes, há uma relação imperfeita, como na figura~\ref{fig:pulsoSubAgl}.


%% Qualquer unidade (nota), de uma
%% dada sequência musical que tenha métrica pode ser
%% univocamente assim especificada:

%% \begin{equation}
%% P^{ \{ j_k \} }_{ \{ i_{k} \}}
%% \end{equation}

%% em que $j_k$ é o nível de aglomeração e $i_k$ é a ordem
%% da unidade em si.

%% Como um exemplo, $P^{-1,0,1}_{3,2,2}$  é a terceira subdivisão $P^{-1}_3$ do segundo
%% pulso $P^0_2$ do segundo aglomerado de pulsos $P^1_2$.
%% Cada unidade ou conjunto de unidades $P_i^j$ pode ser associada a uma sequência de amostras temporais $T_i$ que forma uma nota musical. 

%% A montagem \emph{Poli Hit Mia} utiliza as diferentes métricas e está no Apêndice~\ref{ap:poli}, disponível online como parte da \massa.

%% \subsection{Repetição e variação: motivos e unidades maiores}\label{subsec:motivos}
%% Dadas as estruturas musicais básicas tanto frequenciais (acordes e escalas) quanto 
%% rítmicas (divisões e aglomerações simples, compostas e complexas), é 
%% natural apresentar estas estruturas de forma que tenham coesão e sentido.\cite{Boulez}
%% Para tal, é fundamental o conceito de arcos: partindo de algum lugar e voltando, forma-se um arco. 
%% A audição de linhas melódicas e harmônicas é permeada de arcos musicais pela natureza cognitiva
%% da escuta musical. Pode-se considerar a nota o menor arco, cada motivo e melodia também um arco.
%% Cada tempo e cada subdivisão, cada compasso e secção da música, constitui um arco próprio.
%% Uma música, cujos arcos não apresentam consistência entre si, pode ser compreendida como uma música sem coesão.
%% A sensação de coerência provém, em grande parte, do tratamento hábil dos arcos de uma peça.

%% Os arcos musicais são estruturas abstratas e passíveis de operações básicas. Um arco espectral, como um
%% acorde, pode ser invertido, ampliado e permutado, por exemplo. Os arcos temporais, como uma melodia, um motivo, um compasso ou
%% uma nota, são igualmente passíveis de variações. Lembrando que $S_j=\left\{s_j=T_i^j=\{t_i^{j}\}_0^{\Lambda_j-1}\right\}_0^{H-1}$ é uma
%% sequência de $H$ eventos musicais $s_j$, cada evento com suas $\Lambda_j$ amostras $t_i^j$ (veja no início desta seção~\ref{notasMusica}), as técnicas básicas podem ser descritas assim:

%% \begin{itemize}
%%     \item A translação temporal é o deslocamento $\delta$ do material para um outro instante $\Gamma'=\Gamma + \delta$ da música. É uma variação do material com modificação na localização no decorrer da música: $\left\{s_j'\right\}=\left\{s_j^{\Gamma'}\right\}=\left\{s_j^{\Gamma+\delta}\right\}$ onde $\Gamma$ é a duração entre o começo da peça (ou trecho considerado) e o primeiro evento $s_0$ da estrutura $S_j$ original. Observe que $\delta$ é o deslocamento no tempo.
%%     \item A dilatação ou contração temporal é a alteração da duração de cada arco por um fator $\mu\,:\; s_j'^{\Delta}=s_j^{\mu_j . \Delta}$. Possivelmente, $\mu_j=\mu$ constante.
%%     \item Reversão temporal consiste em gerar uma sequência com os elementos em ordem invertida da sequência original $S_j$, assim: $S_j'=\left\{s_j'\right\}_0^{H-1}=\left\{s_{(H-j-1)}\right\}_0^{H-1}$.
%%     \item A translação em altura é o deslocamento $\tau$ do material para uma altura diferente $\Xi'=\Xi + \tau$ da música. É uma variação com modificação na localização em altura do material: $\left\{s_j'\right\}=\left\{s_j^{\Xi'}\right\}=\left\{s_j^{\Xi+\tau}\right\}$ onde $\Xi$ é a altura do trecho $S_j$ ou do primeiro evento $s_0$ da estrutura $S_j$ original. Observe que $\tau$ é o deslocamento em altura. Caso $\tau$ seja dado em semitons, o deslocamento em frequência é $\tau_f=f_0.2^{\frac{\tau}{12}}$ onde $f_0$ é resultado da adoção de alguma referência: $f_0=\Xi_{f_0}Hz\;\sim \Xi_0$ valor absoluto de altura. Para a frequência para qualquer valor de altura: $\Xi_f=\Xi_{f_0}.2^{\frac{\Xi-\Xi_0}{12}}$. 
%%         Para altura de qualquer valor de frequência: $\Xi=\Xi_0 +12 . \log_2\left(\frac{\Xi_f}{\Xi_{f_0}}\right)$.  No protocolo MIDI, $\Xi_{f_0}=55Hz$ enquanto $\Xi_0=33$ de altura marca o lá 1, outro ponto de referência é o $\Xi_{f_0}=440Hz$ que corresponde a $\Xi_0=69$. A diferença da unidade equivale ao semitom. Não se pode dizer que a unidade é o semitom pois $\Xi=1$ não é um semitom, é uma nota de frequência audível como ritmo, com menos de 9 ocorrências por segundo (veja a tabela~\ref{tab:duracoes}).
%%     \item A inversão intervalar é a inversão do sentido dos intervalos percorridos pelo material. A inversão é estrita se o número de semitons esteja sendo usado como referência para a operação: $S_j'=\{s_j'\}_0^{H-1}=\left\{s_j^{-\varepsilon_j . f_0}\right\}$, onde $\varepsilon_j$ é o fator entre a frequência do evento $s_j$ e a frequência de $s_0$. A inversão é tonal caso as distâncias sejam consideradas em termos de números de graus da escala $E_k$: $S_j'=\{s_j'\}_0^{H-1}=\left\{s_j^{-\varepsilon^{\left(e_{\left(j_e\right)}\right)} . f_0}\right\}_0^{H-1}$ onde $j_e=e_{j}^{inv}$ é o índice $k=j_e$ em $E_k$ da nota do evento $s_j$.
%%     \item Rotação de elementos musicais é a transferência do último elemento para a posição do primeiro e o deslocamento deste ao penúltimo uma posição para frente. Pode-se definir a rotação de $\tilde{n}$ posições por $S'_n=S_{(n+\tilde{n})\%H}$. Caso $\tilde{n}<0$, basta usar $\tilde{n}'=H-\tilde{n}$. É razoável associar $\tilde{n}>0$ com a rotação horária e $\tilde{n}<0$ com a rotação anti-horária. Mais sobre rotações na subseção~\ref{estCic}.
%%     \item A inserção e remoção de materiais na estrutura $S_j$ pode ser ornamental ou estruturante: $S_j'=\{s_j'\}=\{s_j \text{ se condição A, caso contrário } r_j\}$, para qualquer material musical $r_j$, incluindo o instante vazio. Elementos de $R_j$ podem ser inseridos no começo, como um prefixo de $S_j$, ao final, como um sufixo, ou no meio, dividindo $S_j$ ou fazendo dele o prefixo e o sufixo. Os dois materiais podem se misturar das formas mais diversas.
%%     \item Modificações de articulação, instrumentação e espacialização $s_j'=s_j^{*_j}$, com $*_j$ a nova característica incorporada pelo elemento $s_j'$.
%%     \item Acompanhamento. Tanto a instrumentação quanto as linhas melódicas presentes na ocorrência de $S_j$ podem sofrer modificações e ser considerada uma variação de $S_j$ em si.
%% \end{itemize}

%% Com estes procedimentos, outros são derivados: o retrógrado do inverso, uma contração temporal com um sufixo externo, etc.
%% As estruturas musicais ressoam no sistema cognitivo devido à própria natureza do pensamento. Em suas várias facetas,
%% uma ideia lida com o mesmo número de elementos e aspectos conectivos entre eles.
%% A música, através de sintonia com estas estruturas mentais, suscita impressões. É, assim, desencadeado
%% todo um processo de ressonância mental e neurológica responsável pelos sentimentos, lembranças e imaginações
%% típicas de uma audição musical atenta. Esta atividade cortical contribui para a terapia
%% musical, conhecida pela utilidade em casos de depressão e dano neurológico. Considera-se que as regiões do cérebro responsáveis pelo processamento auditivo são também usadas para outras atividades, incluindo linguísticas e matemáticas.\cite{Sacks,Roederer}

%% As estruturas paradigmáticas orientam a criação de novos materiais musicais. Uma delas, central, é o dipolo tensão/relaxamento. Relaciona-se com este dipolo, cada outro dipolo tradicional: tônica/dominante, repetição/variação, consonância/dissonância, coerência/rompimento, simetria/assimetria, igualdade/diferença, chegada/saída, perto/longe, parado/em movimento, etc. Já as relações ternárias tendem a se relacionar com o circular e a unificação. A lúcida comunhão ternária, 'modus perfectus', se opõe ao dicotômico passional, 'modus imperfectus'.
%% A seguir está uma exposição  dedicada aos arcos direcionais 
%% e cíclicos.

%% \subsection{Estruturas direcionais}\label{subsec:dir}


%% Os arcos podem ser decompostos em duas sequências convergentes: 
%% uma que atinge o ápice, e 
%% outra que, de forma paradigmática, volta do ápice à região de partida. Este ápice é chamado de clímax pela teoria musical tradicional. Distingue-se entre arcos cujos clímax estão: no começo, no meio, no final, na primeira metade e na segunda metade da duração considerada. Estas estruturas estão na figura~\ref{fig:climax}. O parâmetro que varia pode não existir, no caso o arco consiste somente em uma estrutura de referência.\cite{Schoenberg}

%% \begin{figure}[h!]
%%     \centering
%%         \includegraphics[width=\textwidth]{figuras/climax}
%%     \caption{Distinções canônicas do clímax musical em uma melodia e outros domínios. As possibilidades diferenciadas são: clímax no começo, clímax na primeira metade, clímax no meio, clímax na segunda metade, clímax no fim. Não está especificado o eixo das ordenadas pois pode não haver variação paramétrica real, neste caso a estrutura é uma referência.}
%%         \label{fig:climax}
%% \end{figure}


%% Seja $S_i=\{s_i\}_0^{H-1}$ uma sequência crescente. A sequência 
%% $R_i=\{r_i\}_0^{2H -2}=\left\{s_{(H-1-|H-1-i|)}\right\}_0^{2H-2}$ 
%% é uma sequência que apresenta simetria especular perfeita, i.e. a segunda metade é uma versão espelhada da primeira. Segundo conceitos musicais, o clímax está exatamente no meio da sequência. Pode-se modificar isso com o uso de sequências de tamanhos diferentes. Toda a teoria matemática de sequências, já estabelecida e ensinada corriqueiramente em cursos de cálculo III, pode ser utilizada para geração destes arcos.\cite{Guidorizzo,Schoenberg} 
%% Teoricamente, estas sequências, aplicadas desta forma
%% a qualquer característica dos eventos musicais, produzem arcos,
%% pois implicam no afastamento e retorno de uma parametrização inicial. 
%% Assim, é possível para uma mesma sequência de eventos possuir um número de arcos distintos, com tamanhos e clímax diferentes. Este é um recurso
%% interessante e útil e a correlação dos arcos resulta na coerência da escuta.\cite{Salzer}

%% Na prática, e historicamente, tem especial importância a razão áurea e, por isso, a sequência de Fibonacci. A sequência de Lucas permite uma generalização da sequência de Fibonacci que pode ser compreendida facilmente. Dados dois números quaisquer $x_0$ e $x_1$, obtém-se a sequência de Lucas: $x_n=x_{n-1}+x_{n-2}$. Quanto maior for $n$, mais a razão $\frac{x_{n}}{x_{n_1}}$ se aproxima da razão áurea: $1.61803398875...$. A sequência converge rápido mesmo para valores iniciais bem discrepantes. Seja $x_0=1$ e $x_1=100$, e $y_n=\frac{x_n}{x_{n+1}}$ uma sequência auxiliar. O erro em percentagem dos primeiros valores desta sequência com relação à proporção áurea é, aproximadamente, $\{ e_n \} =\left\{100\frac{y_n}{1.61803398875}-100 \right\}_1^{10}=\{6080.33, -37.57, 23, -7.14, 2.937, -1.09, 0.42, -0.1601, 0.06125, -0.02338\}$. A sequência de Fibonacci apresenta aproximadamente a mesma progressão de erros, mas começa já no segundo passo pois $\frac{1}{1}\approx\frac{101}{100}$.

%% A montagem sonora \emph{Dirracional} expõe estes arcos em estruturas direcionais. Seu código está no Apêndice~\ref{ap:dirracional} e disponível online como parte da \massa.\cite{MASSA}

%% \subsection{Estruturas cíclicas}\label{estCic}

%% O entendimento filosófico de que o pensamento humano é fundamentado
%% na percepção de semelhanças e diferenças, dentre os estímulos
%% e objetos, coloca
%% as simetrias no cerne do processo cognitivo.\cite{Deleuze}
%% Matematicamente,
%% as simetrias são grupos algébricos e um grupo finito
%% é sempre isomorfo a um grupo de permutações. 
%% Pode-se dizer que
%% as permutações representam quaisquer simetrias em um sistema finito.
%% Na música, as permutações são ubíquas
%% e estão presentes em técnicas, o que confirma seu papel central.
%% A aplicação sucessiva das permutações gera arcos cíclicos.\cite{change,Zamacois,permMusic}
%% A esta abordagem foram dedicados dois trabalhos acadêmicos para geração de estruturas musicais.\cite{figgusOriginal, figgusEspacializacao}

%% Qualquer conjunto de permutações pode ser utilizado como gerador de grupos algébricos.\cite{permMusic} As propriedades que definem um grupo $G$ são:

%% \begin{equation}\label{eq:groups}
%% \begin{split}
%% \forall \;\; p_1,p_2 \in G \Rightarrow\quad\quad\quad\;\; p_1 \bullet p_2 & = p_3 \in G  \quad\quad\quad\;\;\;\text{(propriedade de fechamento)} \\
%% \forall \;\; p_1,p_2,p_3 \in G \Rightarrow\quad (p_1\bullet p_2)\bullet p_3 & = p_1\bullet (p_2\bullet p_3)\quad\;  \text{(propriedade da associatividade)} \\
%% \exists \;\; e \in G :\quad\quad\quad\quad\; p \bullet e & = e \bullet p \;\;\;\; \forall p \in G  \quad \text{(existência do elemento neutro)} \\
%% \forall \;\; p \in G, \;\exists\; p^{-1} :\quad\quad\quad\;  p\bullet p^{-1} & =p^{-1}\bullet p = e  \quad\quad\;\text{(existência do inverso)}
%% \end{split}
%% \end{equation}


%% Da primeira propriedade conclui-se que toda permutação pode ser operada com outra permutação. De fato, pode-se aplicar uma permutação $p_1$, depois outra $p_2$ e, se comparadas as ordenações inicial e final, há uma permutação $p_3$.

%% Todo elemento $p$ operado consigo mesmo um número suficiente de vezes $n$ atinge o elemento neutro $p^n=e$, caso contrário o grupo seria infinito (gerado por $p$). O menor $n\;:\;p^n=e$ é chamado de ordem do elemento. Assim, uma permutação finita $p$ aplicada sucessivamente atinge a ordenação inicial dos elementos, formando um ciclo. Este ciclo, se utilizado para parâmetros de notas musicais, implica em um arco musical cíclico.

%% Estes arcos podem ser efetuados pelo uso conjunto de diferentes permutações. Como exemplo
%% histórico, a tradição chamada \emph{change ringing} concebe música através de sinos tocados um após o outro e então tocados novamente, mas em uma ordem diferente. Este processo se repete até que se atinja a ordenação inicial. O conjunto de ordenações diferentes percorridas é um \emph{peal}. A tabela~\ref{tab:change} representa um \emph{peal} tradicional de 3 sinos (1, 2 e 3) que explora todas as suas ordenações. Cada linha apresenta uma ordenação dos sinos a ser tocada. As permutações estão entre as linhas.
%% Neste caso, a estrutura musical consiste em permutações propriamente ditas e algumas permutações diferentes operam para o comportamento cíclico. 

%% \begin{table}[htpq!]
%% \centering
%% \caption{Change Ringing: \emph{Peal} (padrão) com 3 sinos. As permutações estão entre as ordenações. Cada linha é uma ordenação dos sinos, cada ordenação é tocada, uma linha por vez.}
%% \begin{tabular}{l c r}
%% \textcolor{red}{1} & \textcolor{blue}{2} & \textcolor{green}{3} \\
%% \textcolor{blue}{2} & \textcolor{red}{1} & \textcolor{green}{3} \\
%% \textcolor{blue}{2} & \textcolor{green}{3} & \textcolor{red}{1} \\
%% \textcolor{green}{3} & \textcolor{blue}{2} & \textcolor{red}{1} \\
%% \textcolor{green}{3} & \textcolor{red}{1} & \textcolor{blue}{2} \\
%% \textcolor{red}{1} & \textcolor{green}{3} & \textcolor{blue}{2} \\
%% \textcolor{red}{1} & \textcolor{blue}{2} & \textcolor{green}{3}
%% \end{tabular}
%% \label{tab:change}
%% \end{table}


%% A utilização de permutações na música pode ser resumida da seguinte forma:
%% seja $S_i=\{s_i\}$ uma sequência de eventos musicais $s_i$ (e.g. notas) e $p$ uma permutação.
%% $S_i'=p(S_i)$ consiste nos mesmos elementos de $S_i$ mas em ordem diferente.
%% As permutações podem ser escritas em duas notações: cíclica ou natural. 
%% A notação natural consiste na ordem dos índices 
%% resultante da permutação. Assim,
%% convencionada a ordenação original dada pela sequência de seus índices $[0\;1\;2\;3\;4\;5\;...]$ a permutação é notada pela sequência que produz (ex. $[1\;3\;7\;0\;...]$). Na notação cíclica, a permutação é a troca de um elemento pelo
%% da frente, e o último pelo primeiro.

%% Não é necessário permutar os elementos de $S_i$, mas somente
%% alguma ou algumas de suas características. Assim, seja $p^f$ uma permutação 
%% nas frequências e $S_i$ uma sequência de notas básicas como expostas
%% ao final de~\ref{notaBasica}. A nova sequência $S_i'=p^f(S_i)=\left\{s_i^{p(f)}\right\}$ consiste nas mesmas
%% notas musicais, na mesma ordem e com as mesmas características, com as frequências fundamentais permutadas segundo o padrão que $p^f$ apresenta.

%% Duas sutilezas deste procedimento.
%% 1) A permutação $p$ não precisa envolver todos os elementos de $S_i$, i.e. ela
%% pode operar em subconjuntos de $S_i$. 2) Nem todos os elementos $s_i$ precisam ser executados a cada consulta de estado realizada.
%% Para exemplificar, seja $S_i$ 
%% uma sequência de notas musicais $s_i$. 
%% Se $i$ vai de $0$ a $n$, e $n>4$, a cada compasso
%% de $4$ notas pode-se executar as primeiras $4$ notas. As outras notas de
%% $S_i$ podem incidir nos compassos em que as permutações aloquem
%% estas notas para as primeiras quatro notas de $S_i$.

%% A cada uma destas permutações
%% $p_i$, segundo a exposição acima, relaciona-se: dimensões das notas em que opera (frequência, duração, \emph{fades}, intensidade, etc) e período de incidência (a cada quantas consultas é aplicada a permutação). Na realização das notas de $S_i$, uma forma fácil e coerente é executar as primeiras $n$ notas\footnote{A execução de notas disjuntas de $S_i$ equivale a modificar a permutação e executar as primeiras notas.}.

%% No Apêndice~\ref{cap:FIGGUScode} está a implementação computacional disponibilizada em.\cite{MASSA,figgusOriginal,figgusEspacializacao}

%% \subsection{Idioma musical?}

%% Existem diversas empreitadas que se propõem a modelar e explorar entendimentos
%% sobre a 'linguagem musical', a 'linguística aplicada à música' ou ainda
%% para discernimento entre 
%% o que seriam diferentes
%% `idiomas musicais'.\cite{Lerdahl, Harmonia, Salzer,Alfaix}
%% De forma simples, um idioma musical é fruto da escolha de materiais básicos e
%% repetição de elementos e da repetição de relações entre elementos presentes no decorrer da música. Nestas questões, as dicotomias são salientes,
%% como explicadas na subseção~\ref{subsec:motivos}: repetição e variação, relaxamento e tensão, equilíbrio e desequilíbrio, consonância e dissonância, etc. 

%% \subsection{Usos musicais}\label{subsec:usosmusicais3}

%% Primeiro, a nota básica foi definida e caracterizada em termos
%% claros e quantitativos (seção~\ref{sec:notaDisc}). Em seguida, a composição interna da nota foi abordada, e compreendidas as transições internas e tratamentos imediatos (seção~\ref{varInternas}). Por fim, esta presente seção dedica-se a organizar estas notas em música.
%% A gama de recursos e consequente infinidade de possibilidades de resultados
%% é situação típica e cara às artes.\cite{Harmonia,Webern}

%% Existem estudos para cada recurso apresentado. Por exemplo, pode-se obter as harmonias triádicas 'sujas' (com notas não pertencentes à tríade) através de sobreposições de quartas justas. Outro exemplo interessante é a presença simultânea de ritmos em diferentes métricas, constituindo o que chama-se de \emph{polirritmia}. A montagem musical \emph{Poli-hit mia} explora estas métricas simultâneas através de trem de impulsos convoluidos com as notas que compõem cada linha. Seu código está no Apêndice~\ref{ap:poli} e disponível online como parte da \massa.

%% As escalas microtonais são importantes na música do século XX~\cite{microtonalidade} e possuem resultados muito marcantes, como os quartos de tom na música indiana. A sequência musical \emph{MicroTom} explora estes recursos, incluindo melodias microtonais e harmonias microtonais com várias notas em um âmbito de alturas bastante reduzido. Seu código está no Apêndice~\ref{ap:micro} e disponível online como parte da \massa.

%% Como também apontado na subseção~\ref{subsec:mus2}, os vínculos entre parâmetros são formas poderosas de se obter peças e montagens musicais. O número de notas permutadas pode variar no decorrer da música, revelando vínculo com a duração da peça. As harmonias podem constituir-se triádicas (eqs.~\ref{triades}) com notas replicadas em várias oitavas e mais numerosas quanto menor a profundidade e frequência de vibratos (eqs.~\ref{vbrGamma},~\ref{vbrAux},~\ref{vbrF},~\ref{vbrGamma2},~\ref{vbrT}), dentre outras incontáveis possibilidades.

%% As simetrias apresentadas nas divisões da oitava (eqs.~\ref{escSim}) e as simetrias apresentadas através das permutações (tabela~\ref{tab:change} e eqs.~\ref{eq:groups}) podem ser usadas em conjunto. Nas peças \emph{3 trios} esta associação é feita de forma sistemática para possibilitar uma audição a ela dedicada. Esta é uma peça instrumental e não consta dentre os códigos dos Apêndices e da \massa.\cite{3Trios}

%% O \emph{PPEPPS} (Pure Python EP: Projeto Solvente) é um EP sintetizado com os recursos apresentados neste trabalho. Com pouca parametrização, o programa gera músicas inteiras, permitindo a composição de músicas e conjuntos de músicas com facilidade. Através de poucas linhas de código e, pela execução, pode-se obter uma pasta com as músicas. Esta facilidade e entrega tecnológica abre possibilidades estéticas, de compartilhamento e educacionais.

%% \chapter{Conclusões e trabalhos futuros} %Nome do capítulo.
%% \label{cap:conclusao}

%% No capítulo anterior está um sistema conciso
%% que relaciona elementos musicais ao som digital. \emph{Scripts} 
%% implementam estas relações, e em conjunto foram nomeados \massa\ (Música
%% e Áudio em Sequências e Séries Amostrais). 
%% A exposição didática destes desenvolvimentos no
%% capítulo anterior destina-se a facilitar a utilização
%% do arcabouço.

%% As possibilidades abertas por estes resultados envolvem a criação de interfaces de geração de ruídos e outros sons em alta fidelidade (\emph{hi-fi}), experimentos psicoacústicos e a utilização destes resultados para fins artísticos e didáticos. A incorporação de conhecimentos
%% em programação é bastante facilitada através de recursos audiovisuais, o que já realizamos por práticas de \emph{livecoding} e cursos focados em ferramentas especializadas, como o Puredata e o ChucK.
%% Está prevista a utilização destes resultados com
%% métodos de inteligência artificial para geração de materiais artísticos.

%% A disposição online destes conteúdos na forma de hipertexto junto aos códigos e exemplos sonoros, todos em licenças livres, facilita colaborações e geração de subprodutos em co-autoria, e com isso a expansão da \massa\ com novas implementações e desenvolvimentos das montagens musicais.
%% Explorações sistemáticas de parametrizações (dos tremolos, da ADSR, etc) em alta fidelidade tem utilidade artística e é possibilitada por este trabalho com controle amostral. Tal descrição analítica precisa, junto às implementações computacionais, não foi atingida anteriormente, como mostra o Apêndice~\ref{cap:trabalhosRelacionados} com uma visita aos trabalhos relacionados.

%% Este trabalho também teve resultados não previstos, como a formação de grupos
%% de interesse em torno da questão criativa aliada à computação.
%% Neste contexto, destaca-se o grupo
%% labMacambira.sf.net, que reúne colaboradores de todo o Brasil e alguns fora do país.
%% Este grupo
%% já apresentou contribuições relevantes em diferentes áreas
%% como Democracia Direta Digital, ferramentas de georreferenciamento e
%% atividades artísticas e educacionais, como cursos, workshops e apresentações artísticas. Vários destes resultados estão no Apêndice~\ref{cap:musicaExtra} e no acervo online, que ultrapassa 700 vídeos, documentações escritas, diversos software originais e contribuições em software externos utilizados no mundo todo, como o Firefox, Scilab, LibreOffice, GEM/Puredata, para citar somente alguns exemplos.\cite{siteLM,wikiLM,vimeoLM}

%% Há um aumento no número de pesquisas relacionadas à música em andamento no campus de São Carlos da USP, o que sugere facilidade para estabelecer parcerias.
%% As publicações acadêmicas efetivadas durante este mestrado também apontam para uma multidisciplinaridade,
%% tratando diretamente de questões humanas como artes, filosofia, humor e linguagem falada e escrita, através de artifícios lineares e estatísticos.\cite{FabbriSTAT,FabbriACL,FabbriComplenetVoz,FabbriComplenetTexto} Os desdobramentos estão alcançando redes sociais e teorias epistemológicas com base em pesquisas prévias, dos orientadores deste trabalho, com forte presença de redes complexas e processamento de linguagem natural. 


\nocite{*}
\bibliography{phi+mus}
\end{document}
