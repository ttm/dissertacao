%% ------------------------------------------------------------------------- %%
\chapter{Introdução} %Nome do capítulo.
\label{cap:intro} %Rótulo para futura referência ao capítulo. Em qualquer lugar da tese, você poderá citar este capítulo através de ~\ref{cap:introducao}. Você escolhe o argumento de \label e pode ser qualquer coisa (Ex: \label{Procedimento_Experimental})
Partindo do som como um fenomeno fíisico, esta dissertação se propõe a ser um guia relativamente 
completo para uma atuação eficiênte no desenvolvimento
de software voltado ao áudio e tecnologias livres. Optamos
realizar este trabalho através da síntese e da simplicidade. 

Abordamos o som, o áudio, o áudio digital e práticas de programação voltadas para
áudio. Não menos importantes são as plataformas nas quais desenvolvemos e executamos código
- usualmente GNU/Linux ou BSD e navegadores de páginas de internet (\emph{browsers}) - e
os sistemas de controle de versão dos códigos.

Com estas tecnologias em mãos, foram criados diversos aplicativos com fins artísticos,
técnicos, científicos e sociais. Como consequência, foi criada uma equipe para tratar
destas questões, o Lab Macambira, e um sistema rico de comunicação e exposição de todos
os desenvolvimentos aqui expostos.

Este trabalho tange algumas questões ainda mais gerais e difíceis.
Como abordar a atuação tecnológica e científica em uma área criativa? Como contemplar
a cultura de compartilhamento e diversidade em que esta produção tecnológica está inserida?
Enfim, é um trabalho criterioso definir quais os pontos a serem abordados em
um trabalho como este.


\section{O som e o áudio}
O som é tipicamente utilizado para fala, sinais em geral\footnote{p.ex. buzina} e música.
Caracteriza-se por ser um sentido (estudar som na wikipedia e talvez helmholtz e zuckerkandl)


  \subsection{Breve histórico e práticas atuais}
As explicações físicas do fenômeno do som no decorrer do tempo. Teorias
corpusculares, escola de Tales. Helmholtz. Surgimento do áudio. Áudio digital.


  \subsection{Campos de estudo e atuação}
som: Acúsitica, psicoacúsica, luthearia
áudio: microfones, caixas de som
áudio digital: conversores a/d d/a, programação voltada ao áudio e música


\section{Código aberto, software livre e cultura livre}
O código é repetidamente destacado como o bem mais valioso produzido em nosso tempo.
O código aberto - e sua abordagem politico-filosófica chamada de software livre -
se propõe a constituir um repertório tecnológico da humanidade, acessível por
qualquer ser humano. Esta proposta orienta o que é chamado de cultura digital,
um agregado de práticas e explorações recentes e advindas da popularização do digital.
É tido como o vetor de transformação de nossa época por excelência.
Qualquer licença de software livre é também uma licença de código aberto
(Open Source), a diferença entre as duas nomenclaturas reside essencialmente
na sua apresentação e na filosofia.


  \subsection{Cultura de compartilhamento}
  \label{sec:soft_compar}
Proposta de compartilhamento. Softwares para dar suporte a esta conectividade.


  \subsection{Licenças livres para software e para outras mídias}
  \label{sec:licencas_livres}
Principais tipos de licenças

Licenças voltadas para software.

Licenças voltadas para outras mídias.

Licenças fechadas.

  \subsubsection{Principais licenças}
  \label{sec:princ_licencas}
Licenças uma a uma. GPL, LGPL, BSD, MIT, Beerware, etc


\section{Estado das coisas}
\label{sec:context}
Aqui no Brasil: as redes existentes, carta midias livres, fluxo de informação..
No mundo: mudança de paradigma com relação a tudo que jah rolou e rolará:
protestos, modismos, entidades, leis e tentativas de leis.

  \subsection{A convergência das mídias}
  \label{sec:midiamultimidia}

Os anos de 1999 e 2000 compreendem o conhecido 'marco inicial da primeira década digital'.
Embora seja principalmente uma forma de delimitar um processo-chave já em andamento, a este período é atribuído
a delineação dos contornos da 'sociedade em rede'. Esta mudança paradigmática da nossa
sociedade é sentida agora em maior intensidade nos negócios, na comunicação e na cultura. Estes dois
últimos fatores são cruciais para o trabalho aqui apresentado dada a notoriedade do papel do
fenômeno sonoro para a comunicação e expressões culturais. Mais recentemente, podemos falar claramente
do papel do código nestas duas frentes.

O termo 'convergência das mídias' não é utilizado para indicar algum dispositivo ou produto que
lida com várias mídias. Para isso utiliza-se o termo multimídia, que é tema do próximo tópico.
A convergência das mídias se refere à mundaneidade no lidar com suportes diferentes. Procura-se
sobre um assunto e lidamos com imagens, textos, videos, áudios e até outras formas diferenciadas
de mídia, como o código de liguagem de programação.


HTML5 CSS3 e Javascript deixando interativo. 3D, conexões rápidas que possuem banda para video HD e
baixa latência. Popularização de hardware capaz destes processamentos.

Tendência:
\begin{itemize}
    \item intensa utilização de audiovisual em contraposição ao
anterior predominio de textos simples.
    \item a utilização de periféricos para resultar saída diferenciada - p.ex.
impressora - ou entrada especializada - p.ex. touch pad.
\end{itemize}


  \subsection{Multimídia}

    \subsubsection{Gargalos na interatividade e no estado do software}
    \label{sec:gargalos}
Deficiencias em disponibilização da tecnologia atual: muita
tecnologia fechada.

Comparação do software proprietário com o livre para áudio:

XXXXXXXXXXXXXXXXXXXXX

Citação da parte de video da thread 'Abrindo o Código Fechado'.

Para certas utilizações, a latência não permite a interatividade como 
a obtida em presença do objeto examinado seja ele físico ou apenas um BD.
A medida do 'agora' é de até 50 ms para humanos [Roads, Microsound] e a latência entre comunicadores é
quase sempre maior que isso [tabela de latências, média, por distância, etc].

Contribuimos na frente do software. Para isso veja [sessao dos resultados]


    \subsubsection{Dinâmica de desenvolvimento}
    \label{sec:din_dev}
A modularidade do sistema \emph{Unix type}. Codigo como substancia modular
e reativa. Colaboratividade, listas de discussão, IRC.

No caso do áudio: as compensações: jack,
facilidades de instalação e segurança, tudo hackeável.

    \subsubsection{Compartilhamento}
    \label{sec:comp_tec}
Terreno fértil -> semear que a primazia do compartilhamento se encarrega de
desencadear os desenvolvimentos ligitimos.

%%%%%%%%
    \subsection{Identificação de tarefas pouco suportadas}
    \label{sec:tarefas_n_sup}
[ESCREVER NO FINAL]

    \subsubsection{Comparacao das soluções em software livre e proprietários}
    \label{sec:sl_prop}
[ESCREVER NO FINAL E ABORDAR BREVEMENTE AS QUESTOES DE VIDEO]


\section{Objetivos}
\label{sec:objetivos}
Aqui analisamos os objetivos deste trabalho. Eles refletem os objetivos
dos empreendimentos de SL em geral e tambem das investidas no audiovisual e nas artes.

  \subsection{Difusão de tecnologias e práticas}
  \label{sec:tutoriais}
Descentralização: Fator caminhão e espalhamento de informação. Descentralização
é uma constante nas estruturas e ferramentas mais relevantes como redes
sociais, comunicadores que agora funcionam por broadcast e protocolos de
compartilhamento de dados, em especial arquivos inteiros, unidades mais
bem organizadas.

Aumento da conectividade através de avanços em telecomunicações e transportes.

    \subsubsection{Tutoriais}
    \label{sec:uso_sl}

\begin{itemize}
    \item Tutoriais para programação em python.
Screen casts e texto.

    \item Tutoriais para fazer plugins.
\end{itemize}

    \subsubsection{Estudo de caso da difusão de conteúdos livres}
    \label{sec:dif_sl}
Estamos desenvolvendo trabalhos em análise de comparação
entre sucessões de tendências filosóficas ~\cite{philome} e
musicais ~\cite{musime} e trabalhos em processamento de escrita
e fala ~\cite{rede-associacoes, complenet, enfmc, ifsc}.

Os estudos subsequentes a estes trabalhos devem desenvolver
um estudo sobre a dinâmica de geração e propagação de conteúdo na rede.
Comparando textos e utilizando medidas de similaridade separa-se
o que um texto não contém e o que é semelhante, i.e. servem de propagação
um do outro. A criatividade será enfocada também como estudos de dinâmicas
de atributos de redes complexas de palavras, como as redes de associação ~\cite{rede-associacoes}.


  \subsection{Extensão de recursos em software livre para multimídia}
  \label{sec:extensao}

  \subsubsection{Manipulação e tratamento de áudio}
  \label{sec:manip-audio}


  \subsubsection{Plugins feitos}
  \label{sec:tutoriais}

  \subsubsection{Performance em tempo real}
  \label{sec:perf}
  
  \subsubsection{Síntese de sons e estruturas musicais}
  \label{sec:sintese}

  \subsubsection{Analise}
  \label{sec:analise}