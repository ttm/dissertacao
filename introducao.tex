%% ------------------------------------------------------------------------- %%
\chapter{Introdução} %Nome do capítulo.
\label{cap:intro} %Rótulo para futura referência ao capítulo. Em qualquer lugar da tese, você poderá citar este capítulo através de ~\ref{cap:introducao}. Você escolhe o argumento de \label e pode ser qualquer coisa (Ex: \label{Procedimento_Experimental})

O som é tipicamente utilizado para fala, sinais em geral\footnote{p.ex. buzina} e música.
Caracteriza-se por ser um sentido   O livre fluxo de informação

%% ------------------------------------------------------------------------- %%

\section{O som e o áudio}

\subsection{Breve histórico e práticas atuais}

As explicações físicas do fenômeno do som no decorrer do tempo. Teorias
corpusculares, escola de Tales. Helmholtz.

\subsection{Campos de estudo e atuação}

\section{Código aberto, software livre e cultura livre} %Nome da seção.

O código é repetidamente destacado como o bem mais valioso produzido em nosso tempo.
O código aberto - e sua abordagem politico-filosófica chamada de software livre -
se propõe a constituir um repertório tecnológico da humanidade, acessível por
qualquer ser humano. Esta proposta orienta o que é chamado de cultura digital,
um agregado de práticas e explorações recentes e advindas da popularização do digital.
É tido como o vetor de transformação de nossa época por excelência.
Qualquer licença de software livre é também uma licença de código aberto (Open Source), a diferença entre as duas nomenclaturas reside essencialmente na sua apresentação e na filosofia.




\subsection{Cultura de compartilhamento} %Nome da seção.
\label{sec:soft_compar} %Rótulo para futura referência à seção. Em qualquer lugar da tese, você poderá citar esta seção através de \label{sec:consideracoes_preliminares}. O mesmo vale para todas as próximas seções e capítulos.

Softwares para dar suporte a esta conectividade.

%% ------------------------------------------------------------------------- %%
\subsection{Licenças livres para software e para outras mídias}
\label{sec:licencas_livres}

Principais tipos de licenças

Licenças voltadas para software.

Licenças voltadas para outras mídias.

Licenças fechadas.

\subsubsection{Principais licenças}
\label{sec:princ_licencas}



%%%%%%%%%%%%%%%%%%%%%%%%%%%%%%%%%%%%%%
%%%%%%%%%%%%%%%%%%%
\section{Estado das coisas}
\label{sec:context}


\subsection{A convergência das mídias}
\label{sec:midiamultimidia}

HTML5 e Javascript deixando interativo. 3D, conexões rápidas que possuem banda para video HD e
baixa latência. Popularização de hardware capaz destes processamentos.

Tendência: 
\begin{itemize}
    \item intensa utilização de audiovisual em contraposição ao
anterior predominio de textos simples.
    \item a utilização de periféricos para resultar saída diferenciada - p.ex.
impressora - ou entrada especializada - p.ex. touch pad.
\end{itemize}


\subsection{Multimídia}

\subsubsection{Gargalos na interatividade e no estado do software}
\label{sec:gargalos}

Deficiencias em disponibilização da tecnologia atual: muita
tecnologia fechada.

Comparação do software proprietário com o livre para áudio:

XXXXXXXXXXXXXXXXXXXXX

Citação da parte de video da thread 'Abrindo o Códifo Fechado'.

Para certas utilizações, a latência não permite a interatividade como 
a obtida em presença do objeto examinado seja ele físico ou apenas um BD.
A medida do 'agora' é de até 50 ms para humanos [Roads, Microsound] e a latência entre comunicadores é
quase sempre maior que isso [tabela de latências, média, por distância, etc].

Contribuimos na frente do software.


\subsubsection{Dinâmica de desenvolvimento}
\label{sec:din_dev}

A modularidade do sistema \emph{Unix type} e as compensações: jack,
facilidades de instalação e segurança, tudo hackeável.

\subsubsection{Compartilhamento}
\label{sec:comp_tec}

Terreno fértil -> semear

%%%%%%%%
\subsection{Identificação de tarefas pouco suportadas}
\label{sec:tarefas_n_sup}

\subsubsection{Comparacao das soluções em software livre e proprietários}
\label{sec:sl_prop}


\section{Objetivos}
\label{sec:objetivos}


\subsection{Difusão de tecnologias e práticas}
\label{sec:tutoriais}

Descentralização: Fator caminhão e espalhamento de informação. Descentralização
é uma constante nas estruturas e ferramentas mais relevantes como redes
sociais, comunicadores que agora funcionam por broadcast e protocolos de
compartilhamento de dados, em especial arquivos inteiros, unidades mais
bem organizadas.

Aumento da conectividade através de avanços em telecomunicações e transportes.

\subsubsection{Tutoriais}
\label{sec:uso_sl}

\begin{itemize}
    \item Tutoriais para programação em python.

Screen casts e texto.

    \item Tutoriais para fazer plugins.
\end{itemize}

\subsubsection{Estudo de caso da difusão de conteúdos livres}
\label{sec:dif_sl}

Estamos desenvolvendo trabalhos em análise de comparação
entre sucessões de tendências filosóficas ~\cite{philome} e
musicais ~\cite{musime} e trabalhos em processamento de escrita
e fala ~\cite{rede-associacoes, complenet, enfmc, ifsc}.

Os estudos subsequentes a este trabalhos devem desenvolver
um estudo sobre a dinâmica de geração e propagação de conteúdo na rede.
Comparando textos e ustilizando medidas de similaridade separa-se
o que um texto não contém e o que é semelhante, i.e. servem de propagação
um do outro. A criatividade será enfocada também como estudos de dinâmicas
de atributos de redes complexas de palavras, como as redes de associação ~\cite{rede-associacoes}.


\subsection{Extensão de recursos em software livre para multimídia}
\label{sec:extensao}

\subsubsection{Manipulação e tratamento de áudio}
\label{sec:manip-audio}


\subsubsection{Plugins feitos}
\label{sec:tutoriais}

\subsubsection{Performance em tempo real}
\label{sec:perf}
  
\subsubsection{Síntese de sons e estruturas musicais}
\label{sec:sintese}

\subsubsection{Analise}
\label{sec:analise}