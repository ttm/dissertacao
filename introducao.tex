%% ------------------------------------------------------------------------- %%
\chapter{Introdução} %Nome do capítulo.
\label{cap:intro} %Rótulo para futura referência ao capítulo. Em qualquer lugar da tese, você poderá citar este capítulo através de ~\ref{cap:introducao}. Você escolhe o argumento de \label e pode ser qualquer coisa (Ex: \label{Procedimento_Experimental})

O livre fluxo de informação

%% ------------------------------------------------------------------------- %%



\subsection{Código Aberto e Software Livre} %Nome da seção.

Qualquer licença de software livre é também uma licença de código aberto (Open Source), a diferença entre as duas nomenclaturas reside essencialmente na sua apresentação e na filosofia.


\subsection{Softwares para compartilhamento} %Nome da seção.
\label{sec:soft_compar} %Rótulo para futura referência à seção. Em qualquer lugar da tese, você poderá citar esta seção através de \label{sec:consideracoes_preliminares}. O mesmo vale para todas as próximas seções e capítulos.

Softwares para dar suporte a esta conectividade.

%% ------------------------------------------------------------------------- %%
\subsection{Licenças livres para conteúdo}
\label{sec:licencas_livres}

Principais tipos de licenças

Licenças voltadas para software.

Licenças voltadas para outras mídias.

\subsubsection{Principais licenças}
\label{sec:princ_licencas}

\section{Contextualização}
\label{sec:context}

Descentralização: Fator caminhão e espalhamento de informação. Descentralização
é uma constante nas estruturas e ferramentas mais relevantes como redes
sociais, comunicadores que agora funcionam por broadcast e protocolos de
compartilhamento de dados, em especial arquivos inteiros, unidades mais
bem organizadas.

Aumento da conectividade através de avanços em telecomunicações e transportes.

\subsubsection{Dinâmica de desenvolvimento}
\label{sec:din_dev}


\subsubsection{Compartilhamento de Tec}
\label{sec:comp_tec}

Terreno fértil -> semear



\subsection{Mutimídia} % Onde se detecta a ênfase em áudio aqui???
\label{sec:multimidia}

\subsubsection{Gargalo}
\label{sec:gargalo}

\subsubsection{Comparacao das soluções em software livre e proprietários}
\label{sec:sl_prop}

\subsection{Identificação de tarefas pouco suportadas}
\label{sec:tarefas_n_sup}



\section{Objetivos}
\label{sec:objetivos}


\subsection{Difusão de práticas}
\label{sec:difusao}

\subsubsection{Tutoriais}
\label{sec:tutoriais}

\begin{itemize}
    \item Tutoriais para programação em python.

Screen casts e texto.

    \item Tutoriais para fazer plugins.
\end{itemize}

\subsubsection{Uso de software livre}
\label{sec:uso_sl}

\subsubsection{Estudo de caso da difusão de conteúdos livres}
\label{sec:dif_sl}

Estamos desenvolvendo trabalhos em análise de comparação
entre sucessões de tendências filosóficas ~\cite{philome} e
musicais ~\cite{musime} e trabalhos em processamento de escrita
e fala ~\cite{rede-associacoes, complenet, enfmc, ifsc}.

Os estudos subsequentes a este trabalhos devem desenvolver
um estudo sobre a dinâmica de geração e propagação de conteúdo na rede.
Comparando textos e ustilizando medidas de similaridade separa-se
o que um texto não contém e o que é semelhante, i.e. servem de propagação
um do outro. A criatividade será enfocada também como estudos de dinâmicas
de atributos de redes complexas de palavras, como as redes de associação ~\cite{rede-associacoes}.


\subsection{Extensão de recursos em software livre - multimídia}
\label{sec:extensao}

\subsubsection{Manipulação e tratamento de áudio}
\label{sec:manip-audio}


\subsubsection{Plugins feitos}
\label{sec:tutoriais}

\subsubsection{Performance em tempo real}
\label{sec:perf}
  
\subsubsection{Síntese de sons e estruturas musicais}
\label{sec:sintese}

\subsubsection{Analise}
\label{sec:analise}